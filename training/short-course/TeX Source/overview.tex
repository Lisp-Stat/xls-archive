\begin{slide}{}
\chapter{Overview of Lisp-Stat}
Lisp-Stat allows you to enter data, transform data, and compute
summary statistics:
{\Large
\begin{verbatim}
> (def abrasion-loss 
       (list 372 206 175 154 136 112  55  
              45 221 166 164 113  82  32
             228 196 128  97  64 249 219
             186 155 114 341 340 284 267
             215 148))
ABRASION-LOSS
> abrasion-loss
(372 206 175 154 136 112 55 45 ...)

> (log abrasion-loss)
(5.918894 5.327876 5.164786 ...)

> (mean abrasion-loss)
175.4667
> (standard-deviation abrasion-loss)
88.12755
\end{verbatim}}
\end{slide}

\begin{slide}{}
You can construct a variety of interactive and dynamic graphs:
{\Large
\begin{verbatim}
> (histogram hardness :title "Hardness")
#<OBJECT: ...>

> (plot-points
   tensile-strength abrasion-loss
   :Variable-labels (list "T" "A"))
#<OBJECT...>

> (spin-plot 
   (list tensile-strength
         hardness
         abrasion-loss)
   :variable-labels (list "T" "H" "A"))
#<OBJECT: ...>
\end{verbatim}}
\end{slide}

\begin{slide}{}
You can fit linear regression models:
{\Large
\begin{verbatim}
> (def m (regression-model 
          (list hardness tensile-strength)
          abrasion-loss))

Least Squares Estimates:

Constant              885.5374  (61.80104)
Variable  0:         -6.573002  (0.5836548)
Variable  1:         -1.375367  (0.1944645)

R Squared:            0.840129
Sigma hat:            36.51856
Number of cases:            30
Degrees of freedom:         27

#<Object: ...>
\end{verbatim}}
\end{slide}

\begin{slide}{}
and nonlinear regression models:
{\Large
\begin{verbatim}
> (defun f (theta)
    (/ (* (select theta 0) x)
       (+ (select theta 1) x)))
F

> (def m (nreg-model #'f y (list 200 .1)))
Residual sum of squares:   7964.185
Residual sum of squares:   1593.158
...
Residual sum of squares:   1195.449

Least Squares Estimates:

Parameter 0           212.6837 (6.947153)
Parameter 1         0.06412127 (0.0082809)

R Squared:           0.9612608
Sigma hat:            10.93366
Number of cases:            12
Degrees of freedom:         10

M
\end{verbatim}}
\end{slide}

\begin{slide}{}
You can also
\begin{itemize}
\item fit generalized linear models
\item numerically maximize likelihood functions
\item compute approximate posterior moments and marginal densities.
\end{itemize}
\end{slide}

\begin{slide}{}
You can define functions of your own to implement new numerical
methods or dynamic graphical ideas of your own.

We will look at several examples:
\begin{itemize}
\item estimating a survival curve
\item Weibull regression
\item dynamic power transformation
\item variability in density estimation
\item a regression sensitivity demonstration
\item the Grand Tour
\end{itemize}

In most versions of XLISP-STAT it is also possible to incorporate and
access routines written in C or FORTRAN.
\end{slide}
