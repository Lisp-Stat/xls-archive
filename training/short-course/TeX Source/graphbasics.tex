\begin{slide}{}
\chapter{Outline of the Graphics System}
\end{slide}

\begin{slide}{}
\section{Overview}
The graphics window object tree:
\begin{center}
\begin{picture}(400,360)
\put(120,330){\framebox(160,30){\tt window-proto}}
\put(100,260){\framebox(200,30){\tt graph-window-proto}}
\put(120,190){\framebox(160,30){\tt graph-proto}}
\put(240,140){\framebox(160,30){\tt scatmat-proto}}
\put(0,140){\framebox(160,30){\tt spin-proto}}
\put(220,70){\framebox(180,30){\tt histogram-proto}}
\put(0,70){\framebox(180,30){\tt namelist-proto}}
\put(100,0){\framebox(200,30){\tt scatterplot-proto}}
\put(200,330){\line(0,-1){40}}
\put(200,260){\line(0,-1){40}}
\put(200,190){\line(0,-1){160}}
\put(200,190){\line(-6,-1){120}}
\put(200,190){\line(6,-1){120}}
\put(200,190){\line(-2,-3){60}}
\put(200,190){\line(2,-3){60}}
\end{picture}
\end{center}
\end{slide}

\begin{slide}{}
All top-level windows share certain common features:
\begin{itemize}
\item a title
\item a way to be moved
\item a way to be resized
\end{itemize}
These common features are incorporated in the window prototype
\dcode{window-proto}.

Both dialog and graphics windows inherit from the window prototype.
\end{slide}

\begin{slide}{}
\section{Graph Windows}
\subsection{Outline of the Drawing System}
A graphics window is a view onto a {\em drawing canvas}.

The dimensions of the canvas can be {\em fixed}\/ or {\em elastic}.
\begin{itemize}
\item
If a dimension is fixed, it has a scroll bar.
\item
If it is elastic, it fills the window.
\end{itemize}
The name list window uses one fixed dimension; all other standard plots
use elastic dimensions by default.
\end{slide}

\begin{slide}{}
%***** Picture *****
%\vspace{2.5in}
\begin{center}
\mbox{\psfig{figure=coords.ps}}
\end{center}

The canvas has a coordinate system.
\begin{itemize}
\item coordinate units are pixels
\item the origin is the top-left corner
\item the $x$ coordinate increases from left to right
\item the $y$ coordinate increases from top to bottom
\end{itemize}
\end{slide}

\begin{slide}{}
A number of drawing operations are available.

Drawable objects include
\begin{itemize}
\item rectangles
\item ovals
\item arcs
\item polygons
\end{itemize}
These can be
\begin{itemize}
\item framed
\item painted
\item erased
\end{itemize}
Other drawables are
\begin{itemize}
\item symbols
\item strings
\item bitmaps
\end{itemize}
\end{slide}

\begin{slide}{}
The precise effect of drawing operations depends on the {\em state} of the
drawing system.

Drawing system state components include
\begin{itemize}
\item colors -- foreground and background
\item drawing mode -- normal or XOR
\item line type -- dashed or solid
\item pen width -- an integer
\item use color -- on or off
\item buffering -- on or off
\end{itemize}
\end{slide}

\begin{slide}{}
\subsection{Animation Techniques}
There are two basic animation methods
\begin{itemize}
\item
{\em XOR drawing} -- drawing ``inverts'' the colors on the screen
\item
{\em double buffering} -- a picture is built up in a background
buffer and then copied to the screen
\end{itemize}
Some tradeoffs:
\begin{itemize}
\item
XOR drawing is usually faster
\item
XOR drawing automatically preserves the background
\item
XOR drawing distorts background and object during drawing
\item
XOR drawing inherently involves a certain amount of flicker
\item
Moving several objects by XOR causes distortion -- only one can
move at a time
\item
{\em Inverting}\/ is not well-defined on color displays
\end{itemize}
\end{slide}

\begin{slide}{}
Lisp-Stat uses
\begin{itemize}
\item XOR drawing for moving the brush rectangle
\item double buffering for rotation
\end{itemize}

As an example, let's move a highlighted symbol down the diagonal
of a window using both methods.

A new graphics window is constructed by
{\Large
\begin{verbatim}
(setf w (send graph-window-proto :new))
\end{verbatim}}
\end{slide}

\begin{slide}{}
Using XOR drawing:
{\Large
\begin{verbatim}
(let ((width (send w :canvas-width))
      (height (send w :canvas-height))
      (mode (send w :draw-mode)))
  (send w :draw-mode 'xor)
  (dotimes (i (min width height))
    (send w :draw-symbol 'disk t i i)
    (pause 2)
    (send w :draw-symbol 'disk t i i))
  (send w :draw-mode mode))
\end{verbatim}}
Using double Buffering:
{\Large
\begin{verbatim}
(let ((width (send w :canvas-width))
      (height (send w :canvas-height)))
  (dotimes (i (min width height))
    (send w :start-buffering)
    (send w :erase-window)
    (send w :draw-symbol 'disk t i i)
    (send w :buffer-to-screen)))
\end{verbatim}}
\end{slide}

\begin{slide}{}
\subsection{Handling Events}
User actions produce various {\em events}\/ that the system handles
by sending messages to the appropriate objects.

There are several types of events:
\begin{itemize}
\item resize events
\item exposure or redraw events
\item mouse events -- motion and click
\item key events
\item idle ``events''
\end{itemize}
Resize and redraw events cause \dcode{:resize} and \dcode{:redraw}
messages to be sent to the window object.

Mouse and key events produce \dcode{:do-click}, \dcode{:do-motion},
and \dcode{:do-key} messages.

In idle periods, the \dcode{:do-idle} message is sent.

The \dcode{:while-button-down} message can be used to follow the mouse
inside a click\\
(for dragging, etc.)
\end{slide}

\begin{slide}{}
\subsection{Graphics Window Menus}
Every graphics window can have a menu.

The user interface guidelines of the window system determine how
the menu is presented:
\begin{itemize}
\item
On the Macintosh, the menu is installed in the menu bar when the window
is the front window.
\item
In MS Windows, the menu is installed in the application's menu bar when
the window is the front window.
\item
In {\em SunView}, the menu is popped up when the right mouse button is
pressed in the window.
\item
Under {\em X11}, the menu is popped up when the mouse is clicked in a
\macbold{Menu} button at the top of the window.
\end{itemize}
The \dcode{:menu} message retrieves a graph window's menu
or installs a new menu.
\end{slide}

\begin{slide}{}
\subsection{An Example}
As a simple example to illustrate the handling of events, let's
construct a window that shows a single highlighted symbol at its
center.

The window is constructed by
{\Large
\begin{verbatim}
(setf w (send graph-window-proto :new))
\end{verbatim}}
We can add slots for holding the coordinates of our point:
{\Large
\begin{verbatim}
(send w :add-slot 'x
      (/ (send w :canvas-width) 2))
(send w :add-slot 'y
      (/ (send w :canvas-height) 2))

(defmeth w :x (&optional (val nil set))
  (if set (setf (slot-value 'x) (round val)))
  (slot-value 'x))

(defmeth w :y (&optional (val nil set))
  (if set (setf (slot-value 'y) (round val)))
  (slot-value 'y))
\end{verbatim}}
New coordinate values are rounded since drawing operations require
integer arguments.
\end{slide}

\begin{slide}{}
The \dcode{:resize} method positions the point at the center of
the canvas:
{\Large
\begin{verbatim}
(defmeth w :resize ()
  (let ((width (send self :canvas-width))
        (height (send self :canvas-height)))
    (send self :x (/ width 2))
    (send self :y (/ height 2))))
\end{verbatim}}
The \dcode{:redraw} method erases the window and redraws
the symbol at the location specified by the coordinate values:
{\Large
\begin{verbatim}
(defmeth w :redraw ()
  (let ((x (send self :x))
        (y (send self :y)))
    (send self :erase-window)
    (send self :draw-symbol 'disk t x y)))
\end{verbatim}}
\end{slide}

\begin{slide}{}
The \dcode{:do-click} message positions the symbol at the click and then
allows it to be dragged:
{\Large
\begin{verbatim}
(defmeth w :do-click (x y m1 m2)
  (flet ((set-sym (x y)
           (send self :x x)
           (send self :y y)
           (send self :redraw)))
    (set-symbol x y)
    (send self :while-button-down #'set-sym)))
\end{verbatim}}
The \dcode{:do-idle} method can be used to move the symbol
in a random walk:
{\Large
\begin{verbatim}
(defmeth w :do-idle ()
  (let ((x (send self :x))
        (y (send self :y)))
    (case (random 4)
      (0 (send self :x (- x 5)))
      (1 (send self :x (+ x 5)))
      (2 (send self :y (- y 5)))
      (3 (send self :y (+ y 5))))
    (send self :redraw)))
\end{verbatim}}
\end{slide}

\begin{slide}{}
We can turn the random walk on by typing
{\Large
\begin{verbatim}
(send w :idle-on t)
\end{verbatim}}
and we can turn it off with
{\Large
\begin{verbatim}
(send w :idle-on nil)
\end{verbatim}}
A better solution is to use a menu item:
{\Large
\begin{verbatim}
(setf run-item
      (send menu-item-proto :new "Run"
            :action 
            #'(lambda ()
              (send w :idle-on
                    (not (send w :idle-on))))))
\end{verbatim}}
To put a check mark on this item when the walk is running, define
an \dcode{:update} method:
{\Large
\begin{verbatim}
(defmeth run-item :update ()
   (send self :mark (send w :idle-on)))
\end{verbatim}}
\end{slide}

\begin{slide}{}
It would also be nice to have a menu item for restarting the walk:
{\Large
\begin{verbatim}
(setf restart-item
      (send menu-item-proto :new "Restart"
            :action
            #'(lambda () (send w :restart))))
\end{verbatim}}
The \dcode{:restart} method is defined as
{\Large
\begin{verbatim}
(defmeth w :restart ()
  (let ((width (send self :canvas-width))
        (height (send self :canvas-height)))
    (send self :x (/ width 2))
    (send self :y (/ height 2))
    (send self :redraw)))
\end{verbatim}}
and a menu with the two items is installed by
{\Large
\begin{verbatim}
(setf menu
      (send menu-proto :new "Random Walk"))
(send menu :append-items restart-item run-item)
(send w :menu menu)
\end{verbatim}}
\end{slide}

\begin{slide}{}
There may be some flickering when the random walk is running.

This flickering can be eliminated by modifying the \dcode{:redraw}
method to use double buffering:
{\Large
\begin{verbatim}
(defmeth w :redraw ()
  (let ((x (round (send self :x)))
        (y (round (send self :y))))
    (send self :start-buffering)
    (send self :erase-window)
    (send self :draw-symbol 'disk t x y)
    (send self :buffer-to-screen)))
\end{verbatim}}
\end{slide}

\begin{slide}{}
\section{Statistical Graphics Windows}
\dcode{graph-proto} is the statistical graphics prototype.

It inherits from \dcode{graph-window-proto}.

The graph prototype is responsible for managing the data used
by all statistical graphs.

Variations in how these data are displayed are implemented in
separate prototypes for the standard graphs.
\end{slide}

\begin{slide}{}
The graph prototype is a view into $m$ dimensional space.

It allows the display of both points and connected line segments

The default methods in this prototype implement a simple scatterplot
of two of the $m$ dimensions.

Many features of graphics windows are enhanced to simplify
adding new features to graphs.
\end{slide}

\begin{slide}{}
The graph prototype adds the following features:
\begin{itemize}
\item $m$-dimensional point and {\em line start}\/ data
\item affine transformations consisting of
\begin{itemize}
\item centering and scaling
\item a linear transformation
\end{itemize}
\item ranges for raw, scaled and canvas coordinates
\item mouse modes for controlling interaction
\item linking strategy
\item window layout management
\begin{itemize}
\item margin, content and aspect
\item background (axes)
\item overlays
\item content
\end{itemize}
\item standard menus and menu items
\end{itemize}
\end{slide}

\begin{slide}{}
\subsection{Data and Axes}
The \dcode{:isnew} method for the graph prototype
requires one argument, the number of variables:
{\Large
\begin{verbatim}
> (setf w (send graph-proto :new 4))
#<Object: 302823396, prototype = GRAPH-PROTO>
\end{verbatim}}
Using the stack loss data as an illustration, we can add data
{\Large
\begin{verbatim}
> (send w :add-points
        (list air temp conc loss))
NIL
\end{verbatim}}
and adjust scaling to make the data visible:
{\Large
\begin{verbatim}
> (send w :adjust-to-data)
NIL
\end{verbatim}}
We can also add line segments:
{\Large
\begin{verbatim}
> (send w :add-lines (list air temp conc loss))
NIL
\end{verbatim}}
\end{slide}

\begin{slide}{}
You can control whether axes are drawn with the \dcode{:x-axis} and
\dcode{:y-axis} messages:
{\Large
\begin{verbatim}
> (send w :x-axis t)
(T NIL 4)
\end{verbatim}}

The range shown can be accessed and changed:
{\Large
\begin{verbatim}
> (send w :range 0)
(50 80)
> (send w :range 1)
(17 27)
> (send w :range 1 15 30)
(15 30)
\end{verbatim}}
The function \dcode{get-nice-range} helps choosing a range and
the number of ticks:
{\Large
\begin{verbatim}
> (get-nice-range 17 27 4)
(16 28 7)
\end{verbatim}}
To remove the axis:
{\Large
\begin{verbatim}
> (send w :x-axis nil)
(NIL NIL 4)
\end{verbatim}}
\end{slide}

\begin{slide}{}
Initially, the plot shows the first two variables:
\begin{verbatim}
> (send w :current-variables)
(0 1)
\end{verbatim}
This can be changed:
\begin{verbatim}
> (send w :current-variables 2 3)
(2 3)
> (send w :current-variables 0 1)
(0 1)
\end{verbatim}
Plot data can be cleared by several messages:
\begin{verbatim}
(send w :clear-points)
(send w :clear-lines)
(send w :clear)
\end{verbatim}
If the \dcode{:draw} keyword argument is \dcode{nil} the plot
is not redrawn.

The default value is \dcode{t}.
\end{slide}

\begin{slide}{}
\subsection{Scaling and Transformations}
The scale type controls the action of the default
\dcode{:adjust-to-data} method.

The initial scale type is \dcode{nil}:
{\Large
\begin{verbatim}
> (send w :scale-type)
NIL
> (send w :range 0)
(50 80)
> (send w :scaled-range 0)
(50 80)
\end{verbatim}}
Two other scale types are \dcode{variable} and \dcode{fixed}.

For variable scaling:
{\Large
\begin{verbatim}
> (send w :scale-type 'variable)
VARIABLE
> (send w :range 0)
(35 95)
> (send w :scaled-range 0)
(-2 2)
\end{verbatim}}
The \dcode{:scale}, \dcode{:center}, and \dcode{:adjust-to-data}
messages let you build your own scale types.

\end{slide}

\begin{slide}{}
Initially there is no transformation:
{\Large
\begin{verbatim}
> (send w :transformation)
NIL
\end{verbatim}}
If the current variables are 0 and 1, a rotation can be applied to
replace \dcode{air} by \dcode{conc} and \dcode{temp} by \dcode{loss}:
{\Large
\begin{verbatim}
(send w :transformation '#2A((0  0 -1  0)
                             (0  0  0 -1)
                             (1  0  0  0)
                             (0  1  0  0)))
\end{verbatim}}
The transformation can be removed by
{\Large
\begin{verbatim}
(send w :transformation nil)
\end{verbatim}}
\end{slide}

\begin{slide}{}
A transformation can also be applied incrementally:
{\Large
\begin{verbatim}
(let* ((c (cos (/ pi 20)))
       (s (sin (/ pi 20)))
       (m (+ (* c (identity-matrix 4))
             (* s '#2A((0  0 -1  0)
                       (0  0  0 -1)
                       (1  0  0  0)
                       (0  1  0  0))))))
  (dotimes (i 10)
    (send w :apply-transformation m)))
\end{verbatim}}
A simpler message allows rotation within coordinate planes:
{\Large
\begin{verbatim}
(dotimes (i 10)
  (send w :rotate-2 0 2 (/ pi 20) :draw nil)
  (send w :rotate-2 1 3 (/ pi 20)))
\end{verbatim}}
Several messages are available for accessing data values in
raw, scaled, and screen coordinates.

Other messages are available for converting among coordinate systems.

\end{slide}

\begin{slide}{}
\subsection{Mouse Events and Mouse Modes}
The graph prototype organizes mouse interactions into {\em mouse modes}.

Each mouse mode includes
\begin{itemize}
\item a symbol for choosing the mode from a program
\item a title string used in the mode selection dialog
\item a cursor to visually identify the mode
\item mode-specific click and motion messages
\end{itemize}
It should not be necessary to override a graph's \dcode{:do-click} or
\dcode{:do-motion} methods.
\end{slide}

\begin{slide}{}
Initially there are two mouse modes, \dcode{selecting} and
\dcode{brushing}.

We can add a new mouse mode by
{\Large
\begin{verbatim}
(send w :add-mouse-mode 'identify
      :title "Identify"
      :click :do-identify
      :cursor 'finger)
\end{verbatim}}
The \dcode{:do-identify} method can be defined as
{\Large
\begin{verbatim}
(defmeth w :do-identify (x y m1 m2)
  (let* ((cr (send self :click-range))
         (p (first
             (send self :points-in-rect 
                   (- x 2) (- y 2) 4 4))))
    (if p
        (let ((mode (send self :draw-mode))
              (lbl (send self :point-label p)))
          (send self :draw-mode 'xor)
          (send self :draw-string lbl x y)
          (send self :while-button-down
                #'(lambda (x y) nil))
          (send self :draw-string lbl x y)
          (send self :draw-mode mode)))))
\end{verbatim}}
\end{slide}

\begin{slide}{}
The button down action does nothing; it just waits.

An alternative is to allow the label to be dragged, perhaps to make
it easier to read:
{\large
\begin{verbatim}
(defmeth w :do-identify (x y m1 m2)
  (let* ((cr (send self :click-range))
         (p (first
             (send self :points-in-rect 
                   (- x 2) (- y 2) 4 4))))
    (if p
        (let ((mode (send self :draw-mode))
              (lbl (send self :point-label p)))
          (send self :draw-mode 'xor)
          (send self :draw-string lbl x y)
          (send self :while-button-down
                #'(lambda (new-x new-y)
                    (send self :draw-string lbl x y)
                    (setf x new-x)
                    (setf y new-y)
                    (send self :draw-string lbl x y)))
          (send self :draw-string lbl x y)
          (send self :draw-mode mode)))))
\end{verbatim}}
\end{slide}

\begin{slide}{}
\subheading{Standard Mouse Modes and Linking}
The click and motion methods of the two standard modes
use a number of messages.

In \dcode{selecting} mode, a click
\begin{itemize}
\item
Sends \dcode{:unselect-all-points}, unless the extend modifier is used.
\item
Sends \dcode{:adjust-points-in-rect} with click $x$ and $y$
coordinates, width and height returned by \dcode{:click-range},
and the symbol \dcode{selected} as arguments.
\item
While the button is down, a dashed rectangle is stretched from the
click to the mouse.

When the button is released, \dcode{:adjust-points-in-rect} is sent
with the rectangle coordinates and \dcode{selected} as arguments.
\end{itemize}
\end{slide}

\begin{slide}{}
In \dcode{brushing} mode, a click
\begin{itemize}
\item
Sends \dcode{:unselect-all-points} unless the extend modifier is used.
\item
While the mouse is dragged, sends \dcode{:adjust-points-in-rect} with
the brush rectangle and \dcode{selected} as arguments.
\end{itemize}
In \dcode{brushing} mode, moving the mouse
\begin{itemize}
\item
Sends \dcode{:adjust-points-in-rect} with the brush rectangle and
\dcode{hilited} as arguments.
\end{itemize}
\end{slide}

\begin{slide}{}
Points can be in four states:
\begin{itemize}
\item[] \dcode{invisible}
\item[] \dcode{normal}
\item[] \dcode{hilited}
\item[] \dcode{selected}.
\end{itemize}
Linking is based on a {\em loose linking}\/ model:
\begin{itemize}
\item points are related by index number
\item only point states are adjusted
\end{itemize}
The system uses two messages to determine which plots are linked:
\begin{itemize}
\item
\dcode{:links} returns a list of plots linked to the plot (possibly
including the plot itself)
\item \dcode{:linked} determines if the plot is linked, and
turns linking on and off.
\end{itemize}
\end{slide}

\begin{slide}{}
When a point's state is changed in a plot,
\begin{itemize}
\item
Each linked plot (and the plot itself) is sent the
\dcode{:adjust-screen-point} message with the index as argument.
\item
The action taken by the method for \dcode{:adjust-screen-point} may
depend on both current and previous states.
\end{itemize}
Since it is not always feasible to redraw single points,
\begin{itemize}
\item
the \dcode{:needs-adjusting} method can be used to check or set a flag
\item
the \dcode{:adjusting-screen} method can redraw the entire plot if the
flag is set
\end{itemize}
The easiest, though not necessarily the most efficient, way to augment
standard mouse modes is to define a new \dcode{:adjust-screen} method
\end{slide}

\begin{slide}{}
Some useful messages:
\begin{itemize}
\item[] For points specified by index:
\begin{itemize}
\item[] \dcode{:point-showing}
\item[] \dcode{:point-hilited}
\item[] \dcode{:point-selected}
\end{itemize}
For sets of indices:
\begin{itemize}
\item[] \dcode{:selection} or \dcode{:points-selected}
\item[] \dcode{:points-hilited}
\item[] \dcode{:points-showing}
\end{itemize}
Other operations:
\begin{itemize}
\item[] \dcode{:erase-selection}
\item[] \dcode{:show-all-points}
\item[] \dcode{:focus-on-selection}
\item[] \dcode{:adjust-screen}
\end{itemize}
Some useful predicates:
\begin{itemize}
\item[] \dcode{:any-points-selected-p}
\item[] \dcode{:all-points-showing-p} 
\end{itemize}
\end{itemize}
\end{slide}

\begin{slide}{}
\heading{Window Layout and Redrawing}
The \dcode{:resize} method maintains a margin and a content rectangle
\begin{center}
\mbox{\psfig{figure=layout.ps,height=3.4in}}
\end{center}

\begin{itemize}
\item
The plot is surrounded by a margin, used by plot controls.
\item
The content rectangle can use a fixed or a variable aspect ratio.
\item
The size of the content depends on the aspect ratio and the axes.
\item
The plot can be covered by overlays, resized with \dcode{:resize-overlays}.
\end{itemize}
\end{slide}

\begin{slide}{}
The aspect type used can be changed by
\begin{verbatim}
(send w :fixed-aspect t)
\end{verbatim}
The \dcode{:redraw} method sends three messages:
\begin{itemize}
\item
\dcode{:redraw-background} -- erases the canvas and draws the axes
\item 
\dcode{:redraw-overlays} -- sends each overlay the \dcode{:redraw} message
\item
\dcode{:redraw-content} -- redraws points, lines, etc.
\end{itemize}
Many methods, like \dcode{:rotate-2}, also send \dcode{:redraw-content}.

\end{slide}

\begin{slide}{}
Plot overlays are useful for holding controls.
\begin{itemize}
\item Overlays inherit from \dcode{graph-overlay-proto}.
\item Overlays are like transparent sheets of plastic.
\item Overlays are drawn from the bottom up.
\item Overlays can intercept mouse clicks.
\item Clicks are processed from the top down:
\begin{itemize}
\item
Each overlay is sent the \dcode{:do-click} message until one returns a
non-\dcode{nil} result.
\item
Only if no overlay accepts a click is the click passed to the current
mouse mode.
\end{itemize}
\end{itemize}
The controls of a rotating plot are implemented as an overlay.

Other examples of overlays are in \dcode{plotcontrols.lsp} in the
\macbold{Examples} folder.
\end{slide}

\begin{slide}{}
\subsection{Menus and Menu Items}
To help construct standard menus
\begin{itemize}
\item \dcode{:menu-title} returns the title to use
\item \dcode{:menu-template} returns a list of items or symbols
\item \dcode{:new-menu} constructs and installs the new menu
\end{itemize}
The \dcode{:isnew} method sends the plot the \dcode{:new-menu}
message when it is created.
\end{slide}

\begin{slide}{}
Standard items can be specified as symbols in the template:
\begin{itemize}
\item \dcode{color}
\item \dcode{dash}
\item \dcode{focus-on-selection}
\item \dcode{link}
\item \dcode{mouse}
\item \dcode{options}
\item \dcode{redraw}
\item \dcode{erase-selection}
\item \dcode{rescale}
\item \dcode{save-image}
\item \dcode{selection}
\item \dcode{show-all}
\item \dcode{showing-labels}
\item \dcode{symbol}
\end{itemize}
\end{slide}

\begin{slide}{}
\section{Standard Statistical Graphs}
Each of the standard plot prototypes needs only a few new
methods.

The main additional or changes methods are:
\begin{description}\normalsize
\item[]
\dcode{scatterplot-proto}:

Overrides: \dcode{:add-points}, \dcode{:add-lines},
\dcode{:adjust-to-data}.

New messages: \dcode{:add-boxplot}, \dcode{:add-function-contours},
\dcode{:add-surface-contour}, \dcode{:add-surface-contours}.
\item[]
\dcode{scatmat-proto}:

Overrides: \dcode{:add-lines},
\dcode{:add-points}, \dcode{:adjust-points-in-rect},
\dcode{:adjust-screen-point}, \dcode{:do-click}, \dcode{:do-motion},
\dcode{:redraw-background}, \dcode{:redraw-content}, \dcode{:resize}.
\item[]
\dcode{spin-proto}:

Overrides: \dcode{:adjust-to-data},
\dcode{:current-variables}, \dcode{:do-idle}, \dcode{:isnew},
\dcode{:resize}, \dcode{:redraw-content}.

New methods: \dcode{:abcplane}, \dcode{:add-function},
\dcode{:add-surface}, \dcode{:angle}, \dcode{:content-variables},
\dcode{:depth-cuing}, \dcode{:draw-axes}, \dcode{:rotate},
\dcode{:rotation-type}, \dcode{:showing-axes} 
\item[]
\dcode{histogram-proto}:

Overrides: \dcode{:add-points},
\dcode{:adjust-points-in-rect}, \dcode{:adjust-screen},
\dcode{:adjust-screen-point}, \dcode{:adjust-to-data},
\dcode{:clear-points}, \dcode{:drag-point}, \dcode{:isnew},
\dcode{:redraw-content}, \dcode{:resize}.

New methods: \dcode{:num-bins}, \dcode{:bin-counts}.
\item[]
\dcode{name-list-proto}:

Overrides: \dcode{:add-points},
\dcode{:adjust-points-in-rect}, \dcode{:adjust-screen-point},
\dcode{:redraw-background}, \dcode{:redraw-content}.
\end{description}
\end{slide}
