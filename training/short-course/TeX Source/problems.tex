\documentstyle[handout,12pt]{report}
\input{psfig}

% This command enables hyphenation if \tt mode by changed \hyphencharacter
% in the 10 point typewriter font. To work in other point sizes it would
% have to be redefined. It may be Bator to just make the change globally 
% and have it apply to anything that is set in \tt mode
\newcommand{\dcode}[1]{{\tt \hyphenchar\tentt="2D #1\hyphenchar\tentt=-1}}

\instructor{Tierney}
\course{ENAR, Spring 1992}

\begin{document}
\heading{Suggested Exercises 1}
\begin{enumerate}
\item
For each of the following expressions try to predict what the interpreter
will return. Then type them in, see what happens, and try to explain any
differences.
\begin{enumerate}
\item \dcode{(+ (- 1 2) 3)}
\item \dcode{'(+ 3 5 6)}
\item \dcode{(mean (list 1 2 3))}
\item \dcode{(mean '(1 2 3))}
\end{enumerate}

\item
Draw a plot of the following functions:
\begin{enumerate}
\item
$f(x) = 2x + x^{2}$ over the range between -2 and 3.
\item
$f(x) = \cos(x^{2})/(1+x^{2})$ over the range between $-\pi$ and $\pi$.
\end{enumerate}

\item
Generate 50 normal, beta, and gamma variates and plot them using
histograms and quantile plots. The quantile functions are
\dcode{normal-quant}, \dcode{beta-quant}, and \dcode{gamma-quant}.
The \dcode{sort-data} function sorts a sequence.

\item
Modify the histogram animation example to change only one observation
at a time. This produces more continuous changes that may be easier to
watch. For a positive integer argument the function \dcode{random}
returns a random nonnegative integer less than its argument.

\item
Look at the examples above the line in the {\bf Demos} menu.
See if you can
\begin{enumerate}
\item see the sine curve in the {\bf Sine} example
\item find parallel planes in the {\bf Randu} example
\item tell which of the two spheres in the {\bf Spheres} example is hollow.
\end{enumerate}
\end{enumerate}

\newpage
\heading{Hints and Solutions}
\begin{enumerate}
\item
\begin{enumerate}
\item \dcode{2}
\item \dcode{(+ 3 5 6)}
\item \dcode{2}
\item \dcode{2}
\end{enumerate}

\item
\begin{enumerate}
\item
\begin{verbatim}
(def x (rseq -2 3 30))
(plot-lines x (+ (* 2 x) (^ x 2)))
\end{verbatim}
\item
\begin{verbatim}
(def x (rseq (- pi) pi 30))
(plot-lines x (/ (cos (^ x 2)) (+ 1 (^ x 2))))
\end{verbatim}
\end{enumerate}

\item
For normal variates,
\begin{verbatim}
(def x (normal-rand 50))
(histogram x)
(plot-points (normal-quant (/ (iseq 1 50) 51)) (sort-data x))
\end{verbatim}
For gamma variables, the gamma exponent and for beta variables the two
beta exponents are required as additional arguments to the generator
and quantile functions.

\item
\begin{verbatim}
(def x (normal-rand 20))
(setf h (histogram x))

(dotimes (i 50)
  (send h :clear :draw nil)
  (setf (select x (random 20)) (select (normal-rand 20) 0))
  (send h :add-points x))
\end{verbatim}

\item
\begin{enumerate}
\item
Resize the plot to have a long $x$ axis and a short $y$ axis.
\item
Click a few times on the left {\bf Yaw} button.
\item
Try selecting a slice and see what happens as the slice is rotated.
\end{enumerate}
\end{enumerate}

\newpage
\heading{Suggested Exercises 2}
\begin{enumerate}
\item
Define a function that computes the statistic
\begin{displaymath}
\frac{\overline{X}-\mu_{0}}{S/\sqrt{n}}
\end{displaymath}
for a given sample and value of $\mu_{0}$.

\item
Define a function that takes survival and censoring times as arguments
and returns a list of observed times and censoring indicators. The
functions \dcode{pmin} and \dcode{if-else} may be useful.

\item
The survival data objects should be loaded on your computer. Modify
the \dcode{:plot} method to take an optional argument to choose the
Kaplan-Meier or Fleming-Harrington estimator.

\item
Modify the survival object prototype's \dcode{:plot} method to add
$\pm2$ standard error bars. The \dcode{:add-points} method can be
given the keyword argument \dcode{:type} with value \dcode{dashed} to
produce dashed lines; the keyword argument \dcode{:color} can be given
with symbols like \dcode{red}, \dcode{green}, or \dcode{blue}.  Don't
forget to quote the symbols.

\item
\begin{enumerate}
\item
Construct a scatterplot of \dcode{tensile-strength} against
\dcode{abrasion-loss}.
\item
Use the \dcode{lowess} or \dcode{kernel-smooth} function to add a
smoothed line to the plot.
\item
Both of these functions take a smoothing parameter as a keyword
argument.  Write a method for the plot that replaces the smoother with
one using a specified parameter.
\item
Make a slider dialog for controlling the parameter.
\end{enumerate}
\end{enumerate}

\newpage
\heading{Hints and Solutions}
\begin{enumerate}
\item
\begin{verbatim}
(defun t-stat (x mu-0)
  (let ((n (length x))
        (x-bar (mean x))
        (s (standard-deviation x)))
    (/ (- x-bar mu-0) (/ s (sqrt n)))))
\end{verbatim}

\item
\begin{verbatim}
(defun make-surv-data (surv cens)
  (list (pmin surv cens)
        (if-else (<= surv cens) 1 0)))
\end{verbatim}

\item
\begin{verbatim}
(defmeth survival-proto :plot (&optional (use-km t))
  (let ((udt (send self :death-times)))
    (if use-km
        (plot-lines (make-steps udt
                                (send self :km-estimator)))
        (plot-lines (make-steps udt
                                (send self :fh-estimator))))))
\end{verbatim}

\item
\begin{verbatim}
(defmeth survival-proto :plot ()
  (let* ((udt (send self :death-times))
         (km (send self :km-estimator))
         (se (send self :greenwood-se))
         (p (plot-lines (make-steps udt km))))
    (send p :add-lines udt (+ km (* 2 se)) :type 'dashed)
    (send p :add-lines udt (- km (* 2 se)) :type 'dashed)))
\end{verbatim}

\item
\begin{enumerate}
\item \dcode{(setf p (plot-points tensile-strength abrasion-loss))}
\item
\begin{verbatim}
(send p :add-lines (lowess tensile-strength abrasion-loss))
\end{verbatim}
\item
\begin{verbatim}
(defmeth p :lowess (&optional (f 0.25))
  (send self :clear-lines :draw nil)
  (send self :add-lines
        (lowess tensile-strength abrasion-loss :f f)))
\end{verbatim}
\item
\begin{verbatim}
(interval-slider-dialog '(0 1) :action
                        #'(lambda (f) (send p :lowess f)))
\end{verbatim}
\end{enumerate}
\end{enumerate}

\newpage
\heading{Suggested Exercises 3}
\begin{enumerate}
\item
Construct and install a menu. You need to go through the following
steps:
\begin{itemize}
\item
Create the menu and send it the \dcode{:install} message.
\item
Create a menu item or two that do something simple, such as put
up a message dialog.
\item
Append the item or items to the menu.
\end{itemize}
Remove the menu when you are finished.

\item
Reconstruct the plot with a smoother you set up earlier. Add a menu
item that turns the smoother on and off. Have the menu item show a
check mark if the smoother is in the plot.

\item
Add a menu item to a survival function plot that adds or removes
$\pm2$ standard error lines.

\item
Use the \dcode{:x-axis}, \dcode{:y-axis}, and \dcode{:range} messages
to modify the axes and ranges of a plot.
\end{enumerate}

\newpage
\heading{Hints and Solutions}
\begin{enumerate}
\item
\begin{verbatim}
(setf m (send menu-proto :new "My Menu"))
(send m :install)
(let ((i (send menu-item-proto :new "Hello" :action
               #'(lambda () (message-dialog "Hello")))))
  (send m :append-items i))
(send m :remove)
\end{verbatim}

\item
If the plot is \dcode{p}, use a slot to hold a flag to indicate if the
smoother is present:
\begin{verbatim}
(send p :add-slot 'has-smooth)

(defmeth p :has-smooth (&optional (new nil set))
  (if set (setf (slot-value 'has-smooth) new))
  (slot-value 'has-smooth))

(send p :clear-lines)

(let ((i (send menu-item-proto :new "Smoother")))
  (defmeth i :do-action ()
    (let ((has (send p :has-smooth)))
      (if has
          (send p :clear-lines)
          (send p :add-lines
                (lowess tensile-strength abrasion-loss)))
      (send p :has-smooth (not has))))
  (defmeth i :update ()
    (send self :mark (if (send p :has-smooth) t nil)))
  (send (send p :menu) :append-items i))
\end{verbatim}

%\item

%\item
\end{enumerate}

\newpage
\heading{Suggested Exercises 4}
\begin{enumerate}
\item
The {\bf Regression} item in the {\bf Demos} menu sets up the
regression example in which points can be moved to show changes in the
regression line. The plot is called \dcode{p}. Modify one or more
methods of this plot to have it show a smoother instead of a
regression line. Add a slider dialog to control the smoother.
\item
Try the plot interpolation examples in the {\bf Demos} menu.
\item
Try the tour examples in the {\bf Demos} menu.
\end{enumerate}

\newpage
\heading{Hints and Solutions}
\begin{enumerate}
\item
Change the \dcode{:set-regression-line} method to
\begin{verbatim}
(defmeth p :set-regression-line ()
  (let* ((i (iseq (send self :num-points)))
         (x (send self :point-coordinate 0 i))
         (y (send self :point-coordinate 1 i)))
    (send self :clear-lines :draw nil)
    (send self :add-lines (kernel-smooth x y))))
\end{verbatim}

%\item

%\item
\end{enumerate}
\end{document}
