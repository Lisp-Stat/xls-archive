\begin{slide}{}
\section{What is Lisp-Stat}
Lisp-Stat is an extensible environment for statistical computing and
dynamic graphics.

Some of its features are
\begin{itemize}
\item
a variety of numerical statistical operations
\item
a variety of interactive and dynamic graphical methods
\item 
a very high level programming language (Lisp) that can be used to
\begin{itemize}
\item simplify combinations of calculations
\item adapt methods to specific problems
\item add new capabilities to the system
\end{itemize}
%\item
%interface to C and Fortran code
\item
an extensible graphics system through
\begin{itemize}
\item access to graphical interface tools\\(Menus and Dialogs)
\item standard graphs as building blocks
\item use of object-oriented programming
\end{itemize}
\end{itemize}
\end{slide}

\begin{slide}{}
Some things Lisp-Stat does not do:
\begin{itemize}
\item
incorporate a large collection of tools for specialized analyses
\begin{itemize}
\item[]
Such tools can be implemented within the system or, if C or Fortran
implementations are already available, they can be accessed through
dynamic or static loading.
\end{itemize}
\item
provide extensive support for presentation graphics
\begin{itemize}
\item[]
Other systems on workstations and PC's already provide a wide range of
tools for preparing presentation graphics.
\end{itemize}
\end{itemize}
\end{slide}

\begin{slide}{}
Implementation and Portability:
\begin{itemize}
\item Lisp-Stat is a general specification
\item XLISP-STAT is a first implementation
\item XLISP-STAT runs on
\begin{itemize}
\item Apple Macintosh
\item MS Windows
\item Commodore Amiga (J. Lindsey)
\item {\em Sun}\/ workstations using {\em SunView}
\item BSD UNIX workstations running {\em X11}
\end{itemize}
\item A Kyoto Common Lisp-based implementation may be available soon
\end{itemize}
\end{slide}

\begin{slide}{}
\section{Obtaining XLISP-STAT}
XLISP-STAT source code and executables are available free of charge.

Source code for the UNIX, Macintosh, and MS Windows versions is
available by anonymous {\em ftp}\/ from \dcode{umnstat.stat.umn.edu}
and by email from \dcode{statlib}.

\begin{itemize}
\item[]
If you have access to the {\em internet}\/ but have never used {\em
ftp}, there is probably someone at your site who can help you.
\end{itemize}

Executables for Macintosh and MS Windows are also available for
anonymous {\em ftp}\/ from \dcode{umnstat.stat.umn.edu} and some other
sites.
\end{slide}

\begin{slide}{}
If you do not have access to {\em ftp}, you can get a copy of the
Macintosh or MS Windows executables by sending disks and a self-addressed,
stamped mailer to
\begin{center}
\parbox{4in}{\raggedright
Luke Tierney\\
School of Statistics\\
University of Minnesota\\
Minneapolis, MN 55455}
\end{center}
The number of disks required is
\begin{itemize}
\item[] Two 800K disks for the Macintosh version
\item[] One $3\frac{1}{4}$'' high-density disk for the MS Windows version
\end{itemize}
\end{slide}

\begin{slide}{}
\section{Documentation}
A tutorial introduction is available as a technical report:

\refitem{{\sc Tierney, L.}, (1989), ``XLISP-STAT: A statistical
environment based on the XLISP language,'' U of M Tech. Rep. 528.}

The \LaTeX\ source for this report is available for {\em ftp}\/ from
\dcode{umnstat.stat.umn.edu} or by email from \dcode{statlib}.

More complete documentation is available as a book:

\refitem{{\sc Tierney, L.}, (1990), {\em Lisp-Stat: An Object-Oriented
Environment for Statistical Computing and Dynamic Graphics}, New York,
NY: Wiley.}

The technical report corresponds to Chapter 2 of the book.
\end{slide}

\begin{slide}{}
\heading{Some Historical Notes}
\begin{itemize}
\item
Several extensible statistical environments have been based on high
level languages:
\begin{itemize}
\item The New {\em S}\/ language
\item APL-based systems (Anscombe)
\item Tools for the {\em Gauss}\/ system
\end{itemize}
\item
Several researchers have based systems on the Lisp language:
\begin{itemize}
\item McDonald and Pedersen
\item Stuetzle
\item Oldford and Peters
\item Buja and Hurley
\end{itemize}
\item
A strong argument in favor of Lisp is that Lisp provides excellent
support for {\em experimental programming}.
\end{itemize}
\end{slide}

\begin{slide}{}
\section{Course outline}
\begin{enumerate}
\item Overview of Lisp-Stat
\item A Tutorial Introduction
\item Some Lisp Programming
\item Objects
\item Outline of the Graphics System
\item Some Dynamic Graphics Examples
\end{enumerate}
\end{slide}

\begin{slide}{}
\section{Course Objectives}
After completing the course, you should be able to
\begin{itemize}
\item
use Lisp-Stat with your own data for basic statistical calculations
and graphics.
\item
develop some simple numerical and graphical tools for your own
applications
\item
understand the basics of Lisp programming and object-oriented
programming
\end{itemize}
You should also be well prepared to learn more about Lisp-Stat by working
through the book.
\end{slide}




