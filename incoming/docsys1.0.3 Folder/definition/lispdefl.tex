% TeX macros for use with the almond hacked version of definitions
%% Created 10/17/91 RGA
%%  File Name          :  lispdef.tex
%%  Version Number     :  2.0
%%  Last Changed Date  :  98/10/14 
%%  Created Date       :  11/14/95
%%%%%%%%%%%%%%%%%%%%%%%%%%%%%%%%%%%%%%%%%%%%%%%%%%%%%%%%%%%%%%%%%%%%%%
%% Stylistic Clause Definitions

%%% Version 2.0 now includes definitions for the Project-Doc proto
%%% macros as well as the definition objects

%%  This is a modified version which is meant to work with LaTeX
%%  instead of Plain TeX.  

%%  I've commented out the automatic page numbering.  Somebody else
%%  will need to do that.

%% Also, it doesn't use any of the LaTeX document style macros.  

%% 
\def\gotit{\relax} % for defining extent of a code argument

%% Definition Objects
\outer\def\beginDefinition{\vskip 0pt plus .3\vsize\penalty -500
  \vskip 0pt plus -.3\vsize\bigskip\vskip\parskip}
\newskip\BetweenDefSkip
\outer\def\endDefinition{\vskip\BetweenDefSkip}

%%Name Box Objects
\newdimen\NameBoxWd
\newskip\NameBoxSkip
\def\DefNameBox#1#2{\par % #1 = Name   #2 = Entity Type
\vbox{\hrule\hbox{\vrule\kern 3pt\vbox{\kern3pt
  \hbox to\NameBoxWd{ #1 \hfil {\it #2\/}}\kern3pt}\kern3pt\vrule}\hrule}
\markboth{#1}{#1}
\vskip\NameBoxSkip}

%% Usage
\newskip\BeforeUsage
\newskip\AfterUsage
\newdimen\Usageindent
\def\Usage#1{\par\noindent{\bf Usage:\/}%\hfil\par 
\vskip\BeforeUsage
\begingroup\tt\rightskip=0pt plus6em minus3em\relax
\Usageindent=#1em\multiply\Usageindent by 11\divide\Usageindent by 20
\hangindent\Usageindent}
\def\endUsage{\par\endgroup
\vskip\AfterUsage}

%% Arguments
\newskip\BeforeArgument
\newskip\BetweenArgument
\newskip\AfterArgument
\def\beginArguments{\par\noindent{\bf Arguments:}\begingroup
  \vskip\BeforeArgument }
\def\argument{\hang\indent\begingroup\it}
\def\typeArg{\endgroup \ --- \begingroup\tt}
\def\endArg{\endgroup\par
  \vskip\BetweenArgument}
\def\endArguments{\par\endgroup\vskip\AfterArgument}

%% Slots (follows arguments)
\def\beginSlots{\par\noindent{\bf Slots:}\begingroup
  \vskip\BeforeArgument }
\def\slot{\hang\indent\begingroup\it}
\def\initargsSlot{\endgroup\ \  {\bf (}\begingroup\tt}
\def\typeSlot{\endgroup {\bf )}\ --- \begingroup\tt}
\def\docSlot{\endgroup \ --- \begingroup\rm}
\def\endSlot{\endgroup\par
  \vskip\BetweenArgument}
\def\endSlots{\par\endgroup\vskip\AfterArgument}

%% Parameters (KR input slots)
\def\beginProperties{\par\noindent{\bf Properties:}\begingroup
  \vskip\BeforeArgument }
\def\beginParameters{\par\noindent{\bf Parameters:}\begingroup
  \vskip\BeforeArgument }
\def\beginOutputs{\par\noindent{\bf Outputs:}\begingroup
  \vskip\BeforeArgument }
\def\KRSlot{\hang\indent\begingroup\tt}
\def\typeKRSlot{\endgroup \ --- \begingroup\it}
\def\docKRSlot{\endgroup \ --- \begingroup\rm}
\def\endKRSlot{\endgroup\par
  \vskip\BetweenArgument}
\def\endOutputs{\par\endgroup\vskip\AfterArgument}
\def\endParameters{\par\endgroup\vskip\AfterArgument}
\def\endProperties{\par\endgroup\vskip\AfterArgument}

%% Methods and Signals
\newif\ifFirstinList
\def\beginMethods{\par\noindent{\bf Methods:}\begingroup
  \vskip\BeforeArgument\hang\indent\tt\FirstinListtrue }
\def\beginSignals{\par\noindent{\bf Signals:}\begingroup
  \vskip\BeforeArgument\tt\FirstinListtrue }
\def\nextInList{\ifFirstinList\relax\FirstinListfalse\else , \fi}
\def\endMethods{\par\endgroup\vskip\AfterArgument}
\def\endSignals{\par\endgroup\vskip\AfterArgument}


%%Returns
\newif\ifBreakSingleReturn
\newif\ifFirstValueReturn
\newskip\BeforeReturn
\newskip\BetweenReturn
\newskip\AfterReturn
\def\beginReturn{\par\noindent{\bf Returns:}\FirstValueReturntrue}
\def\itemReturn{\ifFirstValueReturn\par\vskip\BeforeReturn\else
\par\vskip\BetweenReturn\fi\hang\indent\FirstValueReturnfalse}
\def\singleReturn{\ifBreakSingleReturn\relax\itemReturn\else\qquad\fi}
\def\simpleReturn{\begingroup\it}
\def\endsReturn{\endgroup}
\def\typeReturn{{\rm Object of Type:} \begingroup \tt }
\def\endtReturn{\endgroup}
\def\complexReturn{\begingroup \it}
\def\midReturn{\endgroup --- \begingroup\tt }
\def\endcReturn{\endgroup}
\def\endReturn{\par\vskip\AfterReturn}


%% Parents
\newskip\BeforeParent
\newskip\AfterParent
\def\Parents{\par\noindent{\bf Parents:}\hskip\BeforeParent
\begingroup \rm\FirstinListtrue }
\def\endParent{\endgroup \par\vskip\AfterParent}

%% Children
\newskip\BeforeChildren
\newskip\AfterChildren
\def\Children{\par\noindent{\bf Children:}\hskip\BeforeChildren
\begingroup \rm\FirstinListtrue }
\def\endChildren{\endgroup \par\vskip\AfterChildren}



%% Documentation
\newskip\BeforeDocumentation
\newskip\AfterDocumentation
\def\beginDocumentation{\par\noindent{\tt Documentation:\/}\par
	\begingroup\narrower
	\parskip=1pt plus 2pt}
\def\endDocumentation{\par\endgroup\vskip\AfterDocumentation}

%% Source
\newskip\Sourceskip
\def\Source#1{\par\noindent{\it Source file:\/} {\tt #1}\par
   \vskip\Sourceskip}

%%%%%%%%%%%%%%%%%%%%%%%%%%%%%%%%%%%%%%%%%%%%%%%%%%%%%%%%%%%%%%%%%%%%%%%
%%  Project Documentation Objects

\def\beginProject#1{ % #1 == Project Name
  \vskip 0pt plus .6\vsize\penalty -1000\vskip 0pt plus -.6\vsize	
  \markboth{#1}{#1}}
\def\endProject{\vfill\eject}


\newskip\afterTitle
\def\projectTitle#1{% #1 = Title
\centerline{{\bf #1\/}}
\vskip\afterTitle
}

\def\projectAuthor#1#2#3{ % #1 Name, #2 Affliiation, #3 Email
\centerline{\it\quad #1,#2\quad \tt #3}
}

\def\projectVersion#1{% #1 = Version String
\vskip\afterTitle
\centerline{#1}
\vskip\afterTitle
}

%% Functional Description
\newskip\BeforeDesc
\newskip\AfterDesc
\def\beginFunctionalDesc{\par\noindent{\bf Functional Description:}\hskip\BeforeDesc
\begingroup \rm }
\def\endFunctionalDesc{\endgroup \par\vskip\AfterDesc}



%% Copyright
\def\projectCopyright#1{ % #1 == Copyright info
\par\noindent{\bf Copyright and Licence:}\hskip\BeforeDesc
\begingroup\rm #1\endgroup \par\vskip\AfterDesc}

%% Reviewer
\newskip\BeforeValue
\newskip\AfterValue
\def\projectReviewer#1#2{ % #1 Name #2 Email
\par\noindent{\bf Reviewed by:}\hskip\BeforeValue #1 \quad ({\tt #2})
\par\vskip\AfterValue}

%% Motivation Description
\def\beginMotivationDesc{\par\noindent{\bf Motivation:}\hskip\BeforeDesc
\begingroup \rm }
\def\endMotivationDesc{\endgroup \par\vskip\AfterDesc}


%% Stat Description
\def\beginStatDesc{\par\noindent{\bf Statistical Description:}\hskip\BeforeDesc
\begingroup \rm }
\def\endStatDesc{\endgroup \par\vskip\AfterDesc}

%%Instructions
\def\beginInstructions{\par\noindent{\bf Instructions:}\hskip\BeforeDesc
\begingroup \rm }
\def\endInstructions{\endgroup \par\vskip\AfterDesc}

%% References
\newskip\BetweenRefs
\newskip\AfterRefs
\def\beginRefs{\par\noindent{\bf References:}\par}
\def\projectRef#1{ % #1 Ref
\vskip\BetweenRefs\par\hang\hskip-\parindent
#1 \par}
\def\endRefs{\vskip\AfterRefs}

%%Examples
\newskip\BetweenExs
\newskip\AfterExs
\newcount\projectExNo
\def\beginExamples{\par\noindent{\bf Examples:}\par\projectExNo=0}
\def\projectEx#1{ % #1 Ref
\advance\projectExNo by 1
\vskip\BetweenExs\par\hang\textindent{\the\projectExNo} #1 \par}
\def\endExamples{\vskip\AfterExs}

%%Warnings
\def\beginWarnings{\par\noindent{\bf Warnings:}\par}
\def\projectWarn#1{ % #1 Ref
\vskip\BetweenExs\par\hang\textindent{{\bf !}} #1 \par}
\def\endWarnings{\vskip\AfterExs}

%%Change History
\def\beginChanges{\par\noindent{\bf Change History:}\par}
\def\projectChange#1{ % #1 Ref
\vskip\BetweenExs\par\hang\textindent{$\bullet$} #1 \par}
\def\endChanges{\vskip\AfterExs}

%%See Also
\def\beginSeeAlsos{\par\noindent{\bf See Also:}\par}
\def\projectSee#1{ % #1 Ref
\vskip\BetweenExs\par\hang\textindent{$\bullet$} #1 \par}
\def\endSeeAlsos{\vskip\AfterExs}


%% Compatibility Info
\def\projectXLSVersion#1{ % #1 == Version Compatibility Info
\par\noindent{\bf {XLISP-STAT} Compatibility Issues:}\hskip\BeforeDesc
\begingroup\rm #1\endgroup \par\vskip\AfterDesc}
\def\projectOSVersion#1{ % #1 == OS Compatibility Info
\par\noindent{\bf Platform Compatibility Issues:}\hskip\BeforeDesc
\begingroup\rm #1\endgroup \par\vskip\AfterDesc}


%%Package
\def\projectPackage#1#2{ % #1 Package-Name
\par\noindent{\bf Package:}\hskip\BeforeValue {\tt #1} ({\tt #2\/})
\par\vskip\AfterValue}

%% List Style Things
%% This next macro is to get around problems with long filenames not
%% breaking properly.
\def\hfudge{\hskip 0pt plus .5\hsize\penalty -500
\hskip 0pt plus -.5\hsize\quad}
\def\beginDepList{\par\noindent{\bf Project depends on the following
additional packages:}\par\begingroup \rm }
\def\endProjectList{\endgroup\par\vskip\AfterDesc}
\def\beginFileList{\par\noindent{\bf Project source files:}
\begingroup \rm }
\def\beginObjectList{\par\noindent{\bf Exported Objects:}
\begingroup \rm }
\def\beginFunList{\par\noindent{\bf Exported Functions and Messages:}
\begingroup \rm }
\def\beginVarList{\par\noindent{\bf Global Variables and Parameters:}
\begingroup \rm }

%% Full Definition List Style
\outer\def\beginObjects{\vskip 0pt plus .3\vsize\penalty -500
  \vskip 0pt plus -.3\vsize\bigskip
\noindent{\bf Exported Classes and Prototype Objects}
\par\vskip\parskip}

\outer\def\beginProjectFuns{\vskip 0pt plus .3\vsize\penalty -500
  \vskip 0pt plus -.3\vsize\bigskip
\noindent{\bf Exported Functions, Generic Functions and Messages}
\par\vskip\parskip}

\outer\def\beginProjectVars{\vskip 0pt plus .3\vsize\penalty -500
  \vskip 0pt plus -.3\vsize\bigskip
\noindent{\bf Global Variables, Parameters and Constants}
\par\vskip\parskip}


%%%%%%%%%%%%%%%%%%%%%%%%%%%%%%%%%%%%%%%%%%%%%%%%%%%%%%%%%%%%%%%%%%%%%%%%
%%%%  Headlines
%% *** Fix Me***
%% Somebody could modify this to work with LaTeX
%\nopagenumbers
%\headline={\ifodd\pageno\rightheadline \else\leftheadline\fi}
%\def\rightheadline{\tenrm\folio \hfill \botmark}
%\def\leftheadline{\tenrm\firstmark \hfill \folio}
			%\topmark
\pagestyle{myheadings}
\voffset=2\baselineskip


%%%%%%%%%%%%%%%%%%%%%%%%%%%%%%%%%%%%%%%%%%%%%%%%%%%%%%%%%%%%%%%%%%%%%%
%%%% Stupid Verbatum Stuff

% Allow reasonally normal use of < and > in (documentation) text

\def\>{{$\rangle$\hskip-\lastskip}}\catcode`\>=\active\global\let>=\>
\def\<{{$\langle$\hskip-\lastskip}}\catcode`\<=\active\global\let<=\<
\def\{{{\tt\char"7B}}%
\def\}{{\tt\char"7D}}%



{\def\:{:\negthinspace}\catcode`\:=\active\global\let:=\:}
% fiddle with : spacing

\def\littchars{
\catcode`\:=\active%
\def\{{{\tt\char"7B}}%
\def\}{{\tt\char"7D}}%
\def\\{{\tt\char"5C}}%
\def\^{{\tt\char"5E}}%
\def\~{{\tt\char"7E}}%
\catcode`\_=12\catcode`\^=12%
\catcode`\&=12\catcode`\#=12%
\catcode`\$=12\catcode`\%=12\relax
}
% The littchars macro should take care of all literal character
% definitions for lisp input/output.  \,{ and } retain their TeX
% meanings but do the obvious thing (in \tt font) when quoted.

\def\cd{\begingroup\littchars\frenchspacing\obeyspaces\excd}
% switch on littchars mode
\def\excd#1{#1\endgroup}

% \cd will cause the next token or group to be read in littchars mode.
% To use it as an argument in a macro you will need to use the
% \expandafter{ trick, thus:
% \formattingmacro\expandafter{\cd{...}}

%% \verb is set up to use the Dirty Trick on p382 of the TeX book.
\def\uncatcodespecials{\def\do##1{\catcode`##1=12 }\dospecials\littchars}
\def\setupverbatim{\uncatcodespecials\obeyspaces}
\def\verb{\begingroup\setupverbatim\doverbatim}
\def\doverbatim#1{\def\next##1#1{##1\endgroup}\next}

\def\\{{\tt\char"5C}}%
\def\^{{\tt\char"5E}}%
\def\~{{\tt\char"7E}}%

%%%%%%%%%%%%%%%%%%%%%%%%%%%%%%%%%%%%%%%%%%%%%%%%%%%%%%%%%%%%%%%%%%%%%%%
%%%%%% Skip Settings
\BetweenDefSkip=\bigskipamount   % Between Definitions
\NameBoxWd=\hsize		 % Box width
\advance\NameBoxWd by -.5 in
\NameBoxSkip=\medskipamount	% Skip after NameBox
\BeforeUsage=1em 		% Skip between Usage: and usage
\AfterUsage=\smallskipamount	% Skip after Usage.
\BeforeArgument=0pt 		% Skip between Arguments: and args
\BetweenArgument=0pt plus 1pt	% Skip between Args
\AfterArgument=\medskipamount	% Skip after arguments section
\BreakSingleReturnfalse
\BeforeReturn=0pt 		% Skip between Returns: and args
\BetweenReturn=0pt 		% Skip between Args
\AfterReturn=\medskipamount	% Skip after Returns section
\BeforeParent=1em		% (Horizontal) skip between Parents:
				% and list
\AfterParent=\smallskipamount	% skip after Parents
\BeforeChildren=1em		% (Horizontal) skip between Children:
				% and list
\AfterChildren=\smallskipamount	% skip after Children
\BeforeDocumentation=\parskip	% After Documentation: before text
\AfterDocumentation=\bigskipamount	% After documentation text
\Sourceskip=\smallskipamount		% After source
\parskip=0pt 

%%% Documentation flags
\BeforeDesc = \smallskipamount
\AfterDesc = \smallskipamount
\BeforeValue = \smallskipamount
\AfterValue = \smallskipamount
\BetweenRefs = \smallskipamount
\AfterRefs = \smallskipamount
\BetweenExs = \smallskipamount
\AfterExs = \smallskipamount
\afterTitle = \medskipamount

