This paper describes in detail the software we have written for
graphically simulating some Markov Chains.  The original motivation
for this work came from a question posed by the second author while a
student in a course on Probability and Stochastic Processes taught by
the first author.  Therefore, we suspect that this software will
probably find its best use as a pedagogical tool in introductory
courses on Probability and Stochastic Processes. 

We had some objectives in writing such a detailed paper.  We
hope that such detailed description will enable people to use our
software effectively and isolate bugs, if any; that this would provide
first-time users with a non-trivial example of the power of {\tt
  Lisp-Stat}\cite{luke} and the object-oriented paradigm inherent in
it; and that 
other users can build upon our design and add features or other
methods that they may find lacking, in which case, we would be glad to
hear about such.

This software is free and can be obtained by anybody from {\tt
  statlib} by anonymous {\tt ftp} or by e-mail. A sample {\tt ftp}
session is given below.
\begin{verbatim}
% ftp lib.stat.cmu.edu
Connected to TEMPER.STAT.CMU.EDU.
220- temper.stat.cmu.edu FTP server ready.
220 StatLib users please send e-mail address as password.
Name (lib.stat.cmu.edu:naras): statlib
331 Please send your email address, for bug reports and logs.
Password: (Type your e-mail address here)
230 Guest login ok, access restrictions apply.
ftp> cd xlispstat
250 CWD command successful.
ftp> bin
200 Type set to I.
ftp> get markov.shar
ftp> quit
221 Goodbye.
%
\end{verbatim}

For more information on the {\tt statlib} server, see the list of
answers to ``Frequently Asked Questions about S'' maintained by Shanti
Gomatam\cite{shanti}. 

This paper is organized as follows.  In section~\ref{sec:codedesc}, we
describe in detail each of the modules that make up the software. In
section~\ref{sec:applications}, we present three applications that
demonstrate the use of our software. Finally, we make some remarks and
touch upon future developments in section~\ref{sec:remarks}.













