The file {\tt states.lsp} contains the code for the state button
objects.  These buttons are slightly different from the control
buttons because they are oval in shape, and moreover, they perform
different actions. It is possible to have one object prototype for
both the control and the state buttons and distinguish between them
using a slot that indicates the nature of the button, but we decided
to do it separately. When the user clicks the mouse on any 
state button, the user gets a summary of the number of visits to that
state. 

\begin{verbatim}
(defproto state-proto '(title state-no loc-and-size no-of-visits 
                              pvec))
\end{verbatim}
The {\tt state-proto} prototype is defined with slot names {\tt
  title}, to hold the name of the state, {\tt state-no}, to hold the
number corresponding to the state, {\tt loc-and-size}, to hold the
location and size of the state button in the {\tt dmc-proto} window,
{\tt no-of-visits}, to hold the number of times the Markov chain has
been to the state, and {\tt pvec}, to hold the list of transition
probabilities in each state.  The location and size are stored as a
list of four values $x$, $y$, width and height. Recall the $x$ value
increases horizontally to the right while the $y$ value increases
vertically down in a graph window.  

\subsubsection{The {\tt :isnew} method}
\label{subsubsec:state-isnew}
\begin{verbatim}
(defmeth state-proto :isnew (name sno vec)
  (let ((n (length vec)))
    (dolist (j (iseq 1 (- n 1)))
            (setf (elt vec j) (+ (elt vec (- j 1)) (elt vec j))))
    (setf (slot-value 'pvec) vec)
    (setf (elt (slot-value 'pvec) (- n 1)) 1))
  (setf (slot-value 'no-of-visits) 0)
  (setf (slot-value 'title) name)
  (setf (slot-value 'state-no) sno))
\end{verbatim}
A new instance of {\tt state-proto} is returned by this method.  The
method is called with arguments {\tt name}, a title, {\tt sno}, the
state number and {\tt vec}, the transition probability vector.
To make it easier for us to generate the states to go to from this
state, we store the distribution function.  Since we are dealing with
floating point numbers, a precautionary measure is taken by setting
the last entry to $1$. 

\subsubsection{The {\tt :loc-and-size} method}
\label{subsubsec:states-loc-and-size}
\begin{verbatim}
(defmeth state-proto :loc-and-size (&optional loc-size)
  (if loc-size
      (setf (slot-value 'loc-and-size) loc-size)
    (slot-value 'loc-and-size)))
\end{verbatim}
This method can be used to set or retrieve the location and size of
the state in the window where it resides.

\subsubsection{The {\tt prob-vec} method}
\label{subsubsec:prob-vec}
\begin{verbatim}
(defmeth state-proto :prob-vec ()
  (let ((n (length (slot-value 'pvec)))
        (tmp (copy-vector (slot-value 'pvec))))
    (dolist (j (iseq (- n 1) 1))
            (setf (elt tmp j) (- (elt tmp j) (elt tmp (- j 1)))))
    tmp))
\end{verbatim}
This method recreates the transition probability vector from the
distribution function.  The method is usually used only for
printing the transition matrix when the user so desires.


\subsubsection{The {\tt :no-of-visits} method}
\label{subsubsec:no-of-visits}
\begin{verbatim}
(defmeth state-proto :no-of-visits ()
  (slot-value 'no-of-visits))
\end{verbatim}
This is an accessor method for the {\tt no-of-visits} slot.

\subsubsection{The {\tt :you-are-current-state} method}
\label{subsubsec:you-are-current-state}
\begin{verbatim}
(defmeth state-proto :you-are-current-state (w)
  (let ((nv (slot-value 'no-of-visits)))
    (setf (slot-value 'no-of-visits) (+ nv 1)))
  (send self :redraw w))
\end{verbatim}
When the Markov chain goes to a new state, that state is sent this
message.  Therefore, this method increments the number of visits and
asks itself to be redrawn. The net effect of this is to highlight this
state as the current state in the window.


\subsubsection{The {\tt you-are-not-current-state} method}
\label{subsubsec:you-are-not-current-state}
\begin{verbatim}
(defmeth state-proto :you-are-not-current-state (w)
  (send self :redraw w))
\end{verbatim}
When the markov chain makes a transition from state $l$ to state $k$,
the state $l$ is sent this message.  Thus, this message restores the
state to a normal state color.  This is accomplished by just sending
itself a {\tt :redraw} message, since the {\tt :redraw} method checks
whether a state is the current state or not before drawing.

\subsubsection{The {\tt :redraw} method}
\label{subsubsec:state-redraw}
\begin{verbatim}
(defmeth state-proto :redraw (window)
  (let* ((dc (send window :draw-color))
         (cs (eql (send window :current-state) 
                  (slot-value 'state-no)))
         (tc (if cs
                 *current-state-text-color*
               *normal-state-text-color*))
         (bc (if cs
                 *current-state-color*
               *normal-state-color*))
defmeth         (ls (send self :loc-and-size))
         (x (first ls))
         (y (second ls))
         (w (third ls))
         (h (fourth ls))
         (cx (+ x (round (* .5 w))))
         (cy (+ y (round (* .5 h)))))
    (send window :draw-color bc)
    (send window :paint-oval x y w h)
    (send window :draw-color tc)
    (send window :draw-text 
          (format nil "~d" (slot-value 'state-no))
          cx (+ cy (round (* .5 (send window :text-ascent))))
          1 0)
    (send window :draw-color dc)
    (send window :frame-oval x y w h)
    nil))
\end{verbatim}
This method is in charge of redrawing the state button in the window
appropriately. We have to determine whether the state is the current
state or not.  If it is the current state, then the state has to be
redrawn with its proper colors. Otherwise, it has to be redrawn with
the usual colors.  The appropriate drawing color is chosen and an oval
is painted at the location in the window where the state button
resides. The state title, which is usually the state number,  is also
redrawn in the appropriate text color in the center of the state
button.  The default drawing color is restored afterwards.  


\subsubsection{The {\tt :my-click} method}
\label{subsubsec:my-click}
\begin{verbatim}
( state-proto :my-click (x y m1 m2)
  (let* ((ls (send self :loc-and-size))
         (left-x (first ls))
         (bot-y (second ls))
         (rt-x (+ left-x (third ls)))
         (top-y (+ bot-y (fourth ls))))
    (if (and (< left-x x rt-x) (< bot-y y top-y))
        t
      nil)))
\end{verbatim}
The method {\tt :my-click} checks if the mouse was clicked returns
{\tt t} if the mouse is clicked in a state's button.  The coordinates
of where the mouse click occurred are passed as {\tt x} and {\tt y}.
The method merely checks whether the click was in a rectangle
encompassing the state button by using information about its location
and size in the window.  If so, it returns {\tt t} else it returns
{\tt nil}. 


\subsubsection{The {\tt :do-click} method}
\label{subsubsec:states-do-click}
\begin{verbatim}
(defmeth state-proto :do-click (w)
  (let ((nv (send self :no-of-visits))
        (time (send w :time)))
    (if (eql time 0)
        (message-dialog (format nil 
                                "State ~s~%No. of visits ~d~%"
                                (send self :title) nv))
      (message-dialog (format nil 
                              "State ~s~%No. of visits ~d~%~
                                Proportion ~7,5f~%"
                                (send self :title) 
                                nv (/ nv time))))))
\end{verbatim}
If the mouse is clicked on a state button, a message dialog pops up
informing the user of the number of visits to the state. The user has
to dismiss the message dialog before he or she can continue further.

\subsubsection{The {\tt :reset} method}
\label{subsubsec:states-reset}
\begin{verbatim}
(defmeth state-proto :reset ()
  (setf (slot-value 'no-of-visits) 0))
\end{verbatim}
This method resets the state to its original state at the start of the
simulation, which only involves setting the slot value {\tt
  no-of-visits} to zero. 

\subsubsection{The {\tt :title} method}
\label{subsubsec:states-title}
\begin{verbatim}
(defmeth state-proto :title (&optional title)
  (if title 
      (setf (slot-value 'title) title)
  (slot-value 'title)))
\end{verbatim}
This method sets or retrieves the title of the state.


\subsubsection{The {\tt :describe} method}
\label{subsubsec:states-describe}
\begin{verbatim}
(defmeth state-proto :describe (&optional time)
   (let ((nv (send self :no-of-visits)))
    (if (> nv 0)
        (if time
            (format t "~3d         ~7d       ~1,5f~%"
                    (slot-value 'title) nv (/ nv time))
          (format t "State ~a, No of visits: ~g~%" 
                  (slot-value 'state-no) nv)))))
 \end{verbatim}
This method prints the number of visits to the state and the
proportion of time the markov chain spent in the state if time is
greater than 0.  Note that if the state has not been visited yet, this
method prints nothing.


\subsubsection{The {\tt :next-state} method}
\label{subsubsec:next-state}
\begin{verbatim}
(defmeth state-proto :next-state ()
  (let* ((u (select (uniform-rand 1) 0))
         (pvec (slot-value 'pvec))
         (state (catch 'index
                  (dotimes (j n)
                           (if (< u (select pvec j))
                               (throw 'index j))))))
    state))
\end{verbatim}
This method will return the next state that the Markov chain will go to from
this state.  A uniform random variable, {\tt u}, is generated.  Then each
element of {\tt pvec} is checked to see if it is less than {\tt u}.  
The first item that is corresponds to the next state that the Markov chain 
will go to.  This algorithm is quite naive, but to use more efficient
algorithms, one just needs to modify this method suitably. 
