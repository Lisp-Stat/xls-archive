This scenario deals with a gambler who has a probability $p$ of winning one 
more dollar with each play of the game, and a probability $q$ of losing a 
dollar.  If successive plays of the game are independent, then, the
collection of random variables $\{X_n,\ n>0\}$, where $X_n$ is the
gambler's fortune at time $n$, is a Markov Chain.
Given a starting amount of $i$ dollars, we can calculate the
probability that the gambler will win $N$ dollars before going broke.
The following code will give a visual representation of the gambler's
fortune.  Starting with $i$ dollars, the probability $P_{i}$ that the
gambler will eventually have $N$ dollars is given by: 
\[ P_{i}= \left\{
   \begin{array}{ll}
   \frac{1-(q/p)^{i}}{1-(q/p)^{N}}, & \mbox{if $p\not=\frac{1}{2}$} \\
   \frac{i}{N}, & \mbox{if $p=\frac{1}{2}$}
   \end{array}
\right. \]
For a detailed explanation of the derivation of this formula, see
Ross\cite{rosstext}.  

This can be illustrated by means of our code.  The file {\tt
  gambler.lsp} contains the following. 
\begin{verbatim}
(defvar n 26)
(defvar *prob-win* .5)
\end{verbatim}
These two lines define the gambler's maximum fortune, $26-1=\$25$, and
the probability of winning each game, $p$, is  $0.45$. 

Next, since we are going to be playing a lot of games, we would like
to speed up things as much as possible.  One easy candidate for the
speedup is the {\tt :next-state} method for the {\tt state-proto}
object. We do that as follows.
\begin{verbatim}
(defmeth state-proto :next-state ()
  (let ((cs (slot-value 'state-no)))
    (if (or (eql cs 0) (eql cs (- n 1)))
	cs
      (let ((u (select (uniform-rand 1) 0)))
	(if (< u *prob-win*)
	    (+ cs 1)
	  (- cs 1))))))
\end{verbatim}
This method takes into account the fact that the transition matrix is
sparse.  Note that the value of {\tt n} is used in the code, which means
that {\tt n} should not be changed in the course of the run. 

Now, we are ready to create the Markov Chain. 
\begin{verbatim}
(defvar m (identity-matrix n))
(dotimes (i (- n 1))
         (when (> i 0)
               (setf (aref m i i) 0)
               (setf (aref m i (+ i 1)) *prob-win*)
               (setf (aref m i (- i 1)) (- 1 *prob-win*))))
(def initial-state 12)
(setf z (send dmc-proto :new m initial-state))
\end{verbatim}
The transition matrix for our Markov Chain is:
\[ \begin{array}{ll}
	P_{00}=P_{NN}=1 & \\
   	P_{i,i+1}=p=1-P_{i,i-1}=1-q, & i=1,2,\ldots,N-1.
   \end{array} \]
In other words, the probability of the Markov chain moving to the next
state, or, of the gambler winning this play of the game, is $p$, as
was mentioned earlier.  Similarly, the probability of the Markov chain
moving back one state, or, of the gambler losing a dollar, is $q=1-p$.
By defining the initial state of the Markov chain to be 12, we are giving
the gambler twelve dollars to start with.

The last line above creates a new instance of the {\tt dmc-proto}
object using the transition matrix that was just defined, beginning in
our chosen initial state.  Note that although the transition matrix is
used in creating the Markov Chain, it is not used afterwards, since we
have overridden the {\tt :next-state} method for {\tt state-proto}. 

Upon executing the code, the user gets a window like the one shown in
figure~\ref{fig:gambler}.  The current State is highlighted with a
different color. In the figure, we have also
taken the liberty to show the sample paths window and the pop-up
message dialog box when the mouse is clicked on one of the state
buttons.  Note that the sample path window is not enabled
automatically, but has to be requested from the menu in the
Markov Chain window.

Now, how do we run the Markov Chain a large number of times?  The
following code will help us do that.
\begin{verbatim}
(defun test-gambler (number)
  (let ((w 0)
	(l 0)
	(tmp nil))
    (dotimes (j number)
             (format t "Game: ~d~%" j)
	     (send z :reset)
	     (send z :run-dmc-silently 500)
	     (setf tmp (send z :current-state))
	     (if (eql tmp (- n 1))
		 (setf w (+ w 1))
	       (if (eql tmp 0)
		   (setf l (+ l 1)))))
    (format t "Wins: ~d, Losses: ~d, ~
               Decided games: ~d, ~
               Proportion wins: ~d~%"
	    w l (+ w l) (/ w (+ w l)))))
\end{verbatim}
This function will play {\tt number} games, where {\tt number} is
given as an argument to this function.  Note that after each game, we
determine the current state, so that we may detect whether the gambler
won or lost, or whether the game did not end.  When the games are all
over, the number of wins and losses is printed along with the
proportion of wins. 

\begin{figure}[tbhp]
  \caption{The Markov Chain window created by {\tt gambler.lsp}}
  \label{fig:gambler}
  \epsffile{gambler.ps}

% Turn off fancy PStricks stuff.
%     \psellipse[fillstyle=solid,fillcolor=white](2,7)(1.5,0.5)
%     \rput(2,7){Control Buttons}
%     \psline{->}(3.5,7)(4.5,7)
%     \psellipse[fillstyle=solid,fillcolor=white](2,5)(1.5,0.5)
%     \rput(2,5){State Buttons} 33 
%     \psline{->}(3.5,5)(4.5,5)     

\end{figure}
In a test run, we called the function {\tt test-gambler} with an
argument 1000. After 1000 games, the gambler's fortune reached \$25
474 times in 976 decided games, yielding a winning percentage of
48.57\%.  According to the formula, since the probability of winning
each turn was 0.5, the gambler's winning percentage was expected to be
$12/25$, or 48\%.  The computer-generated Markov chain has yielded
results very close to the ones predicted theoretically.
It must be remarked that this simulation took quite a while even on a 
loaded workstation.  We can think of many things that can be
optimized, but that is a subject that doesn't concern us now. 

Now suppose, the probability of winning is not $.5$, but say, $0.45$,
and the gambler's initial fortune is \$15. 
All one needs to do is to change the values of {\tt *prob-win*} and 
repeat the whole process using $15$ as the initial state. A run of
1000 games yielded  126 wins in 997 
decided games, a winning percentage of $0.126379$.  This is close to
the the theoretical value of 
$$
\frac{1-(0.55/0.45)^{15}}{1-(0.55/0.45)^{25}}= 0.128657.
$$	











