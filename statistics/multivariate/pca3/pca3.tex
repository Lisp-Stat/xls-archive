\documentclass[12pt]{amsart}
\usepackage{verbatim}
\begin{document}
\title{PCA of three variables in Xlisp-Stat}
\author{Jan de Leeuw}
\date{\today}
\maketitle
Suppose $u_{i}=(x_{i},y_{i},z_{i})$ are $n$ points in $\mathbb{R}^{3}$.
We want to find direction cosines $w=(w_{1},w_{2},w_{3})$ and
an intercept $a=(a_{1},a_{2},a_{3})$ such that the
line $\mathcal{L}=\{y\mid y=a+\lambda w\}$ approximates the $u_{i}$
as closely as possible.\par
%
The loss function we use is
\begin{equation*}
\sigma(a,w)=\sum_{i=1}^{n}\min_{\lambda_{i}}\ (u_{i}-a-\lambda_{i}w)'
(u_{i}-a-\lambda_{i}w).
\end{equation*}
This means that we project the $u_{i}$ perpendicularly on the line,
and measure the sum of squared distances of the $u_{i}$ and the
projections $\hat u_{i}$. Then choose $a$ and $w$, with $w'w=1$,
such that $\sigma(a,w)$ is minimized.\par
%
It is clear that this problem is identical to the problem of minimizing
\begin{equation*}
\sigma(a,w,\lambda)=\sum_{i=1}^{n} (u_{i}-a-\lambda_{i}w)'(u_{i}-a-\lambda_{i}w)
\end{equation*}
over all three sets of variables, with constraint $w'w=1$. This show 
that we may also require, without loss of generality, that the 
$\lambda_{i}$ sum to zero.\par
%
We first minimize of $a$ for given $w$ and $\lambda$. This gives
$\hat a=u_{\bullet},$
with superscript $\bullet$ indicating mean. Thus
\begin{equation*}
\min_{a}\ \sigma(a,w,\lambda)=\sum_{i=1}^{n} (\tilde u_{i}-\lambda_{i}w)'
(\tilde u_{i}-\lambda_{i}w),
\end{equation*}
with tildes over symbols indicating deviations from the mean. Now 
minimize of the $\lambda_{i}$, which must add up to
zero. The solution is $\hat\lambda_{i}=w'\tilde u_{i}$, which indeed
adds up to zero. We now find
\begin{equation*}
\min_{\lambda}\min_{a}\ \sigma(a,w,\lambda)=\sum_{i=1}^{n}\{\tilde 
u_{i}'\tilde u_{i}^{}-\hat\lambda_{i}^{2}\}.
\end{equation*}
In the final step of our minimization problem, we \textit{maximize}
the sum of squares of the $\hat\lambda_{i}$ over $w$ with $w'w=1$. But
\begin{equation*}
\sum_{i=1}^{n}\hat\lambda_{i}^{2}=w'\{\sum_{i=1}^{n}\tilde u_{i}^{}
\tilde u_{i}'\}w=w'Cw,
\end{equation*}
where $C=\tilde U'\tilde U$ is the $3\times 3$ cross-product matrix of
the $\tilde u_{i}$. It follows that $\hat w$ is the normalized
eigenvector corresponding with the dominant eigenvalue of $C$, or
equivalently of the covariance matrix of the $u_{i}$. Suppose this
dominant eigenvalue is $\omega$. We then draw the line between
$\hat a-\frac{1}{2}\sqrt{\omega}\hat w$ and
$\hat a+\frac{1}{2}\sqrt{\omega}\hat w$, which has length $\omega$.
\par
%
This procedure can be repeated for the other two eigenvalues and
eigenvectors, that capture the other dimensions of variation. \par
%
In Xlisp-Stat we simply add a PCA method to the spin-proto. Sending
an instance of the spin-proto the PCA message will draw the three
principal axis, with squared length equal to the eigenvalues. The 
code is given below
%
%
\begin{verbatim}

(defmeth spin-proto :pca ()
(let* ((n (send self :num-points))
       (x (send self :point-coordinate 0 (iseq n)))
       (y (send self :point-coordinate 1 (iseq n)))
       (z (send self :point-coordinate 2 (iseq n)))
       (c (* (1- n) (covariance-matrix x y z)))
       (m (list (mean x) (mean y) (mean z)))
       (g (eigen c))
       (e (second g))
       (f (sqrt (first g))))
(send self :dircos m (first e) (/ (elt  f 0) 2))
(send self :dircos m (second e) (/ (elt f 1) 2))
(send self :dircos m (third e) (/ (elt f 2) 2))
(print e)
(print f)
))

(defmeth spin-proto :dircos (m w u)
	(send self :abline (+ m (* u w)) (- m (* u w)) )
)

(defmeth spin-proto :abline (a b) 
    (send self :add-lines (make-pairs a b))
)

(defun make-pairs (x y)
  (let ((n (length x)))
     (mapcar #'(lambda (z) (list (elt x z) (elt y z))) (iseq n))
))
\end{verbatim}
\end{document}


