% Format BigAMSLaTeX2e
\documentclass[12pt]{amsart}
\usepackage{uclastat,float}
%\input xypic
%
\newtheorem{theorem}{Theorem}[section]
\newtheorem{lemma}[theorem]{Lemma}
\newtheorem{corollary}[theorem]{Corollary}
%
\theoremstyle{definition}
\newtheorem{definition}[theorem]{Definition}
\newtheorem{example}[theorem]{Example}
%
\theoremstyle{remark}
\newtheorem{remark}[theorem]{Remark}
\newtheorem{assumption}[theorem]{Assumption}
%
\numberwithin{equation}{section}
%
\begin{document}
\title[Symmetric Functions]{Symmetric Functions in Lisp}
\author{Jan de Leeuw}
\date{\today}
\address{UCLA Statistics Program\\
8118 Mathematical Sciences Building\\
University of California at Los Angeles}
\email{deleeuw@@stat.ucla.edu}
\maketitle
%\tableofcontents
%\begin{abstract}
%\end{abstract}
\section{Introduction}
Suppose $\eps=\{\eps_1,\cdots,\eps_n\} is a sequence of real numbers. The elementary 
symmetric functions of these numbers are defined by the recursion
\begin{equation}
\gamma_r(\eps_1,\cdots,\eps_t)=\gamma_r(\eps_1,\cdots,\eps_{t-1})+
\eps_n\gamma_{r-1}(\eps_1,\cdots,\eps_{t-1}),
\end{equation}
for $0\leq r\leq t$ and $1\leq t\leq n$. The initial condition is
\begin{equation}
\gamma_0(\eps)\ident 1.
\end{equation}
For convenience we set $\gamma_1
It follows that
\begin{align}
\gamma_1(\eps)&=\sum_{i=1}^n \eps_i,\\
\gamma_2(\eps)&=\sum_{i<j}^n \eps_i\eps_j,\\
\gamma_3(\eps)&=\sum_{i<j<k}^n \eps_i\eps_j\eps_k,
\end{align}
and so on. In general
\begin{equation}
\gamma_r(\eps)=\sum_{\#(K)=r}^{\binom{n}{r}}\prod_{k\in K}\eps_k
\end{equation}
\section{Programming the Symmetric Functions}
\end{document}