\magnification\magstep1
\parskip 10pt
\parindent 0pt
\input eplain 
\centerline{\bf INTRODUCTION}

The {\tt XLISP-STAT} package lacks a number of procedures 
which would make it more
suitable for routine data handling and data analysis. 
As examples we mention recoding variables, making cross tables, 
printing nice tables and publication-quality graphics, and the reading of
formatted input (as in {\tt FORTRAN} format strings and the
{\tt scanf} function in {\tt C.} On can argue, of course, that this
is not necessarily a bad thing. The {\tt UNIX} philosophy tells us that tools
should be highly specialized. Tools which
can do everything, don't do anything right.

On the other hand there is not reason not to use {\tt XLISP-STAT}
as a convenient interface to other special-purpose {\tt UNIX} tools,
or to add {\tt LISP} functions that people can either use or ignore. 

Making tables and graphs will be discussed in another note. Here we
discuss some {\tt XLISP-STAT} functions to read formatted input.
Code is listed below, and will also be available on {\tt statlib}
in {\tt ftp.stat.cmu.edu.}
\vskip .5cm\centerline{\bf EXAMPLE}

Consider the following small input file, which is called {\tt foo.dat.}
\hfill\break
{\tt 001aap -1.235555}\hfill\break
{\tt 002mens +.154444}\hfill\break
We read this into a list of lists, each list containing one record,
each record haveing the same number of data items. For this we use
the function\hfill\break
{\tt (read-file "foo.dat" ffor),}\hfill\break
with {\tt ffor} a format string. In this case the most obvious format 
string is\hfill\break
{\tt '("i3" "a4" "f5.2" "4i1")}\hfill\break
which creates\hfill\break
{\tt ((1 "aap " -1.23 5 5 5 5) (2 "mens" 0.15 4 4 4 4))}\hfill\break
If we want to, we can also use {\tt "f3.0"} for {\tt "i3",}
but generally i-formats are somewhat more efficient.

Other formats are possible as well. We can use
\hfill\break
{\tt '("i3" "a4" "x5" "4i1")}\hfill\break
which creates\hfill\break
{\tt ((1 "aap " 5 5 5 5) (2 "mens" 4 4 4 4))}\hfill\break
or
\hfill\break
{\tt '("i3" "x4" "f5.2" "t9" "a4" "x5" "4i1")}\hfill\break
which creates\hfill\break
{\tt ((1 -1.23 "aap " 5 5 5 5) (2 0.15 "mens" 4 4 4 4))}\hfill\break
Thus using x-formats allows us to skip characters, while the t-format
allows us to back up. Combining x-formats and t-formats makes it
possible to read the data on a record in arbitrary order. To 
conclude the example:\hfill\break
{\tt '("f3.0" "3a1"  "x3" "f3.2" "2i2")}\hfill\break
gives\hfill\break
{\tt ((1 "a" "a" "p" 0.23 55 55) (2 "m" "e" "n" 0.15 44 44))},\hfill\break
and\hfill\break 
{\tt '("a8" "f2.0" "i6")}\hfill\break
gives
{\tt (("001aap -" 1 235555) ("002mens " 0 154444))}.\hfill\break
This seems to indicate a considerable level of flexibility.
\vskip .5cm\centerline{\bf SYNTAX}

\unorderedlist
\li The i-formats, a-formats, and f-formats can have repeat factors
(as in {\tt "4f7.5"}), the x-format (skip next n places) and
the t-format (go n places back) do not have repeats (in a sense,
they are repeats). 
\li We do not allow {\tt "4(f10.4,x2)"} etc.
\li The string read by the a-format can contain all {\tt ascii}
characters. 
\li The i-format can read characters 0-9, spaces, and signs,
with the spaces first, then either a $+$ or a $-$, and then
a string of digits (which can have leading zeroes). Spaces
and signs can be missing.
\li The f-format can read characters 0-9, point, spaces, and signs,
with the spaces first, then either a $+$ or a $-$, and then
a string of digits (which can have leading zeroes), then the
decimal point, and then another string of digits (which can have
trailing zeroes). Spaces and signs can be missing. The decimal
point and the digit string following it can be missing. The
decimal string before the decimal point can be missing. We even
allow both decimal strings to be missing, in which case the number
is interpreted as zero.
\endunorderedlist
\vfill\eject
\centerline{\bf LISTING}

\listing{read-format.lsp}
\bye
