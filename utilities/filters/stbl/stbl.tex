\documentstyle[psfonts,11pt,ncs,noweb,pstricks,pst-plot,multicol,epsf]{article}
\input nowebmargins
\pagestyle{noweb}
\noweboptions{}
\begin{document}

\def\nw/{{\tt noweb}}
\def\xhdvi/{{\tt XHDvi}}
\def\xls/{{\tt Lisp-Stat}}
\def\web/{{\tt WEB}}
\def\htex/{hyper-\TeX{}}
\def\[{\ifhmode\ \fi$[\mkern-2mu[$}
\def\]{$]\mkern-2mu]$}
\def\xls/{{\tt Lisp-Stat}}
\def\lisp/{{\tt Lisp}}
\def\mosaic/{{\tt Mosaic}}


\title{A Literate Program for a Hyper-Document}
\author{B. Narasimhan\\
  Department of Mathematics\\   
  Penn State Erie, The Behrend College\\
  Erie, PA 16563}

\date{Draft of \today}

\maketitle

\begin{abstract} 

  We describe a literate program for a hyper-document using the \nw/
  literate programming tools. The program essentially consists of a
  hyper-document and the associated code that implements the
  hyper-links. The code uses the \xls/ environment. The hyper-document
  can be \LaTeX{}ed and viewed with \xhdvi/, a \htex/ dvi previewer.
  This is but a small part of a free hyper-text for introductory
  statistics that is being developed by a group of us.

\end{abstract} 

\filename{stbl.nw}\nwbegindocs{1}\section{Introduction} 
\label{sec:intro}
This paper describes a literate program for a hyper-document. The
hyper-document is a \LaTeX{} file with embedded hyper-links to \xls/
code. Using a \htex/ viewer like \xhdvi/ one can view the document and
have \xls/ automatically invoked on the hyper-links.  

Before we go any further, it must be remarked that all the tools
mentioned earlier are freely available. Here is a list that will get
you started.

The \nw/ tools are available from all {\tt CTAN} sites in {\tt
  /web/noweb}. As a last recourse, it is also available at {\tt
  ftp://bellcore.com:/pub/norman}. \xls/ is available from {\tt
  ftp://stat.umn.edu/pub/xlispstat}.  \xhdvi/ is available from {\tt
  http://xxx.lanl.gov/hypertex/}.  Several pre-compiled binaries are
also available.  \xhdvi/ is still under beta-test, and actually
contains bugs.  For example, it is able to process only one hyper-link
per invocation, and dumps core on the second, at least on my SGI
machine. However, such problems are bound to be fixed in the near
future.

After one has \LaTeX{}ed the document, one needs to invoke \xhdvi/ as
shown below.
\begin{verbatim}
% xhdvi -browser 'xterm -e xlispstat' hyperdoc &
\end{verbatim}
One can also, of course, set the {\tt .mailcap} and {\tt .mime.types}
entries, but they are not really necessary for this simple
hyper-document.

\begin{figure}[hbtp]
  \begin{center}
    \leavevmode
      \epsfxsize=3in
      \epsfysize=3in
      \epsffile{pic.ps}
  \end{center}
  \caption{Statistical Tables in \xls/}
  \label{fig:example}
\end{figure}

Some users might be interested in using just the statistical tables. 
The figure~\ref{fig:example} shows how the code works. To invoke the code,
one does the following.
\begin{verbatim}
% xlispstat stbls
\end{verbatim}

Here is our entire document.

\nwenddocs{}\nwbegincode{2}\sublabel{NWstb7-LitG-1}\nwmargintag{{\nwtagstyle{}\subpageref{NWstb7-LitG-1}}}\moddef{Literate Program~{\nwtagstyle{}\subpageref{NWstb7-LitG-1}}}\endmoddef
\LA{}Hyper-document stuff~{\nwtagstyle{}\subpageref{NWstb7-HypK-1}}\RA{}
\LA{}Makefile~{\nwtagstyle{}\subpageref{NWstb7-Mak8-1}}\RA{}
\LA{}Readme file~{\nwtagstyle{}\subpageref{NWstb7-ReaB-1}}\RA{}
\nwnotused{Literate\ Program}\nwendcode{}\nwbegindocs{3}\nwdocspar

Let us begin with {\tt{}Hyper-document\ stuff}. There are two main
parts. The first is {\tt{}Hyper-document} which contains textual matter
with embedded hyper-links. The second is the {\tt{}Code} that implements
the actions invoked when a hyper-link is used.

\nwenddocs{}\nwbegincode{4}\sublabel{NWstb7-HypK-1}\nwmargintag{{\nwtagstyle{}\subpageref{NWstb7-HypK-1}}}\moddef{Hyper-document stuff~{\nwtagstyle{}\subpageref{NWstb7-HypK-1}}}\endmoddef
\LA{}Hyper-document~{\nwtagstyle{}\subpageref{NWstb7-HypE-1}}\RA{}
\LA{}Code~{\nwtagstyle{}\subpageref{NWstb7-Cod4-1}}\RA{}
\nwused{\\{NWstb7-LitG-1}}\nwendcode{}\nwbegindocs{5}\nwdocspar

\section{The Hyper-document}
\label{sec:hyper-doc}
So, what is our hyper-document about? The document should essentially
teach students how to use statistical tables for various
distributions. As this literate program is already sizeable in terms
of printed pages, we shall keep the document brief and save a few
trees.

This hyper-document assumes students know about histograms. The
material in this document is a compressed excerpt from my notes from
Moore and McCabe\cite{moore} for my introductory statistics
class.

\nwenddocs{}\nwbegincode{6}\sublabel{NWstb7-HypE-1}\nwmargintag{{\nwtagstyle{}\subpageref{NWstb7-HypE-1}}}\moddef{Hyper-document~{\nwtagstyle{}\subpageref{NWstb7-HypE-1}}}\endmoddef
\\documentstyle\{article\}
\\def\\href#1#2\{\\special\{html:<a href="#1">\}\{#2\}\\special\{html:</a>\}\}
\\begin\{document\}
\\section\{Density Functions\}
Consider a data set of $1000$ numbers. We already know to pictorially
summarize the distribution of these numbers by means of histograms.
Here, for \\href\{file:hist.lsp\}\{example\}, is a histogram of $1000$
numbers. We can \\href\{file:smhist.lsp\}\{approximate\} the histogram by a
smooth curve that displays the shape of the distribution after ironing
out some of the raggedness. Such smoothing loses some details in the
histogram and therefore can be thought of as an idealized form of the
histogram.  We can also say that the unevenness in the histogram is a
consequence of the classes we have chosen and so the curve is really a
better description of the data.  In a relative frequency histogram,
the areas of the bars are proportional to the relative frequency of
the classes. And if we add together all the relative frequencies, we
get $1$. Therefore, it is natural to ask that our idealization of the
histogram, the smooth curve have total area $1$ underneath it. The area
under the curve between any two values on the $x$-axis is then equal
to the proportion of observations falling between the two values. This
curve is called the density curve of the distribution of the data.

There are many density curves. Let us study an important one which is
called the normal density.

\\subsection\{The Normal Density\}
Let us first take a look at the normal density. Please click
\\href\{file:stbl.lsp\}\{here\} and continue reading this document for
further instructions. 

Now, if all went well, you should have a menu with the word \{\\bf
  Tables\} on it. Press your mouse on the word \{\\bf Tables\} and drag it
on to the \{\\bf Normal Distribution\} menu item and release the mouse
button.

Do you see the density? It should have a bell-shape with a single
peak. Notice how the density is \{\\em symmetric\\/\} about $0$; that is,
the shape to the left of $0$ is the same as the shape to the right of
$0$. The exact density of a normal curve is described by giving
information about two quantities, the mean $\\mu$, where the peak
occurs and the standard deviation $\\sigma$, which specifies how widely
spread the curve is. The curve that you see now has $\\mu=0$ and
$\\sigma=1$ and is called the standard normal distribution. The exact
formula for a normal density curve with mean $\\mu$ and standard
deviation $\\sigma$ is given by 
\\begin\{equation\}
\\phi(x) = \\frac\{1\}\{\\sigma\\sqrt\{2\\pi\}\}
\\exp\{-\\frac\{1\}\{2\}\\biggl(\\frac\{x-\\mu\}\{\\sigma\}\\biggr)^2\} 
\\label\{eq:normal-density\}
\\end\{equation\}

The normal density curve has the following property, which is often
referred to as the empirical rule. 

\\begin\{center\}
\\fbox\{
  \\parbox[b]\{4in\}\{
    \\paragraph\{The $68$-$95$-$99$ Rule\}
    \\begin\{itemize\}
    \\item \\href\{file:68.lsp\}\{$68$\\%\} of the observations fall within
      $\\sigma$ of the mean $\\mu$.
    \\item \\href\{file:95.lsp\}\{$95$\\%\} of the observations fall within
      $2\\sigma$ of the mean $\\mu$.
    \\item \\href\{file:99.lsp\}\{$99.7$\\%\} of the observations fall within
      $3\\sigma$ of the mean $\\mu$.
    \\end\{itemize\}
    \}
  \}
\\end\{center\}
\\end\{document\}
\nwidentuses{\\{{hist.lsp}{hist.lsp}}\\{{68.lsp}{68.lsp}}\\{{95.lsp}{95.lsp}}\\{{99.lsp}{99.lsp}}\\{{smhist.lsp}{smhist.lsp}}\\{{stbl}{stbl}}}\nwindexuse{hist.lsp}{hist.lsp}{NWstb7-HypE-1}\nwindexuse{68.lsp}{68.lsp}{NWstb7-HypE-1}\nwindexuse{95.lsp}{95.lsp}{NWstb7-HypE-1}\nwindexuse{99.lsp}{99.lsp}{NWstb7-HypE-1}\nwindexuse{smhist.lsp}{smhist.lsp}{NWstb7-HypE-1}\nwindexuse{stbl}{stbl}{NWstb7-HypE-1}\nwused{\\{NWstb7-HypK-1}}\nwendcode{}\nwbegindocs{7}\nwdocspar
Notice how this document needs a few \xls/ programs for the
hyper-links. Section~\ref{sec:programs} deals with them.

\section{The Code} 
\label{sec:code}
Let's first get the copyright out of the way.
\nwenddocs{}\nwbegincode{8}\sublabel{NWstb7-CopI-1}\nwmargintag{{\nwtagstyle{}\subpageref{NWstb7-CopI-1}}}\moddef{Copyright for code~{\nwtagstyle{}\subpageref{NWstb7-CopI-1}}}\endmoddef
;;;
;;; @(#)$Header$
;;;
;;; Copyright (C) 1994 B. Narasimhan, naras@euler.bd.psu.edu
;;;
;;; This program is free software; you can redistribute it and/or modify
;;; it under the terms of the GNU General Public License as published by
;;; the Free Software Foundation; either version 2 of the License, or
;;; (at your option) any later version.
;;;
;;; This program is distributed in the hope that it will be useful,
;;; but WITHOUT ANY WARRANTY; without even the implied warranty of
;;; MERCHANTABILITY or FITNESS FOR A PARTICULAR PURPOSE.  See the
;;; GNU General Public License for more details.
;;;
;;; You should have received a copy of the GNU General Public License
;;; along with this program; if not, write to the Free Software
;;; Foundation, Inc., 675 Mass Ave, Cambridge, MA 02139, USA.
;;;

\nwindexdefn{Copyright}{Copyright}{NWstb7-CopI-1}\nwindexdefn{GNU}{GNU}{NWstb7-CopI-1}\nwindexdefn{FSF}{FSF}{NWstb7-CopI-1}\eatline
\nwidentdefs{\\{{Copyright}{Copyright}}\\{{FSF}{FSF}}\\{{GNU}{GNU}}}\nwused{\\{NWstb7-Cod4-1}\\{NWstb7-ProG-1}\\{NWstb7-ProI-1}\\{NWstb7-ProE-1}\\{NWstb7-ProE.2-1}\\{NWstb7-ProE.3-1}\\{NWstb7-UtiE-1}\\{NWstb7-ButJ-1}\\{NWstb7-DisN-1}\\{NWstb7-DisI-1}\\{NWstb7-StaN-1}}\nwendcode{}\nwbegindocs{9}\nwdocspar
So, what must our code do? It should provide a user-friendly interface to
the built-in distribution functions in \xls/. \xls/ has many of the
common distribution functions; however, one has to write \lisp/
phrases to use them.  Since our intended audience is introductory
statistics students, we want to insulate them from \lisp/. Our code
should provide a menu via which the user can choose a particular
distribution as shown in figure\ref{fig:example}. Once a distribution
is chosen, it should display a graph of the density along with button
controls for calculating probabilities or quantiles.  Inputs from the
user should be solicited via informative dialogs.  Answers to queries
must be displayed in the margin of the graph along with parameters
identifying the distribution.  Finally, the code must also be
extensible---if new distributions are added, it must be easy to make
them available too.

Here is the main structure of our program. 
\nwenddocs{}\nwbegincode{10}\sublabel{NWstb7-Cod4-1}\nwmargintag{{\nwtagstyle{}\subpageref{NWstb7-Cod4-1}}}\moddef{Code~{\nwtagstyle{}\subpageref{NWstb7-Cod4-1}}}\endmoddef
\LA{}Copyright for code~{\nwtagstyle{}\subpageref{NWstb7-CopI-1}}\RA{}
\LA{}Program hist.lsp~{\nwtagstyle{}\subpageref{NWstb7-ProG-1}}\RA{}
\LA{}Program smhist.lsp~{\nwtagstyle{}\subpageref{NWstb7-ProI-1}}\RA{}
\LA{}Utility functions~{\nwtagstyle{}\subpageref{NWstb7-UtiH-1}}\RA{}
\LA{}Additional methods for built-in prototypes~{\nwtagstyle{}\subpageref{NWstb7-Addg-1}}\RA{}
\LA{}Implementation constants~{\nwtagstyle{}\subpageref{NWstb7-ImpO-1}}\RA{}
\LA{}Distribution prototype definition~{\nwtagstyle{}\subpageref{NWstb7-DisX-1}}\RA{}
\LA{}Distribution prototype methods~{\nwtagstyle{}\subpageref{NWstb7-DisU-1}}\RA{}
\LA{}Button overlay prototype definition~{\nwtagstyle{}\subpageref{NWstb7-ButZ-1}}\RA{}
\LA{}Button overlay prototype methods~{\nwtagstyle{}\subpageref{NWstb7-ButW-1}}\RA{}
\LA{}Normal distribution~{\nwtagstyle{}\subpageref{NWstb7-NorJ-1}}\RA{}
\LA{}T distribution~{\nwtagstyle{}\subpageref{NWstb7-T*dE-1}}\RA{}
\LA{}Chi-square distribution~{\nwtagstyle{}\subpageref{NWstb7-ChiN-1}}\RA{}
\LA{}F distribution~{\nwtagstyle{}\subpageref{NWstb7-F*dE-1}}\RA{}
\LA{}Main function~{\nwtagstyle{}\subpageref{NWstb7-MaiD-1}}\RA{}
\LA{}Invoke main function~{\nwtagstyle{}\subpageref{NWstb7-InvK-1}}\RA{}
\nwused{\\{NWstb7-HypK-1}}\nwendcode{}\nwbegindocs{11}\nwdocspar

\subsection{Utility functions}
\label{sec:utils}
Here are some simple but useful functions. I won't even bother to
explain them, since they are pretty self-documenting. 

\nwenddocs{}\nwbegincode{12}\sublabel{NWstb7-UtiH-1}\nwmargintag{{\nwtagstyle{}\subpageref{NWstb7-UtiH-1}}}\moddef{Utility functions~{\nwtagstyle{}\subpageref{NWstb7-UtiH-1}}}\endmoddef
(defun probability-p (x)
" Method args: (x)
Returns true if x is a number between 0 and 1, end-points included."
  (and (>= x 0.0) (<= x 1.0)))

(defun strict-probability-p (x)
" Method args: (x)
Returns true if x is a number between 0 and 1, end-points not included."
  (and (> x 0.0) (< x 1.0)))

(defun nonzero-probability-p (x)
" Method args: (x)
Returns true if x is a number between 0 and 1, 0 not included."
  (and (> x 0.0) (<= x 1.0)))

(defun nonunit-probability-p (x)
" Method args: (x)
Returns true if x is a number between 0 and 1, 1 not included."
  (and (>= x 0.0) (< x 1.0)))
\nwindexdefn{probability-p}{probability-p}{NWstb7-UtiH-1}\nwindexdefn{strict-probability-p}{strict-probability-p}{NWstb7-UtiH-1}\nwindexdefn{nonzero-probability-p}{nonzero-probability-p}{NWstb7-UtiH-1}\eatline
\nwidentdefs{\\{{nonzero-probability-p}{nonzero-probability-p}}\\{{probability-p}{probability-p}}\\{{strict-probability-p}{strict-probability-p}}}\nwalsodefined{\\{NWstb7-UtiH-2}\\{NWstb7-UtiH-3}\\{NWstb7-UtiH-4}\\{NWstb7-UtiH-5}\\{NWstb7-UtiH-6}\\{NWstb7-UtiH-7}}\nwused{\\{NWstb7-Cod4-1}\\{NWstb7-UtiE-1}}\nwendcode{}\nwbegindocs{13}\nwdocspar
The next function {\tt{}new-xlispstat} is useful for writing code that is
compatible with older versions of \xls/.
\nwenddocs{}\nwbegincode{14}\sublabel{NWstb7-UtiH-2}\nwmargintag{{\nwtagstyle{}\subpageref{NWstb7-UtiH-2}}}\moddef{Utility functions~{\nwtagstyle{}\subpageref{NWstb7-UtiH-1}}}\plusendmoddef
(defun new-xlispstat ()
" Method args: none
Returns true if the version of xlispstat is 3.xx or greater."
  (and (boundp 'xls-major-release) (>= xls-major-release 3)))
\nwindexdefn{new-xlispstat}{new-xlispstat}{NWstb7-UtiH-2}\eatline
\nwidentdefs{\\{{new-xlispstat}{new-xlispstat}}}\nwendcode{}\nwbegindocs{15}\nwdocspar
Often, we need to get the value of a string returned by a
dialog.
\nwenddocs{}\nwbegincode{16}\sublabel{NWstb7-UtiH-3}\nwmargintag{{\nwtagstyle{}\subpageref{NWstb7-UtiH-3}}}\moddef{Utility functions~{\nwtagstyle{}\subpageref{NWstb7-UtiH-1}}}\plusendmoddef
(defun val-of (str)
"Method args: (str)
Returns the value of str. If empty or invalid string, returns nil."
  (if (= (length str) 0)
      nil
    (if (new-xlispstat)
        (ignore-errors (read-from-string str))
      (unwind-protect
          (read (make-string-input-stream str))))))
\nwindexdefn{val-of}{val-of}{NWstb7-UtiH-3}\eatline
\nwidentdefs{\\{{val-of}{val-of}}}\nwidentuses{\\{{new-xlispstat}{new-xlispstat}}}\nwindexuse{new-xlispstat}{new-xlispstat}{NWstb7-UtiH-3}\nwendcode{}\nwbegindocs{17}\nwdocspar
The function {\tt{}get-values-from} takes a list of {\tt{}text-items} and
returns a list of values from each item in the list. If any item has
invalid values, no error is signalled, but a {\tt{}nil} is returned for
that entry.

\nwenddocs{}\nwbegincode{18}\sublabel{NWstb7-UtiH-4}\nwmargintag{{\nwtagstyle{}\subpageref{NWstb7-UtiH-4}}}\moddef{Utility functions~{\nwtagstyle{}\subpageref{NWstb7-UtiH-1}}}\plusendmoddef
(defun get-values-from (list)
" Method args: (list)
List should be list of edit-text-items. A list of values from each of
them is returned."
  (mapcar #'(lambda(x) (val-of (send x :text))) list))
\nwindexdefn{get-values-from}{get-values-from}{NWstb7-UtiH-4}\eatline
\nwidentdefs{\\{{get-values-from}{get-values-from}}}\nwidentuses{\\{{val-of}{val-of}}}\nwindexuse{val-of}{val-of}{NWstb7-UtiH-4}\nwendcode{}\nwbegindocs{19}\nwdocspar
The function {\tt{}get-numbers-from} is similar, but will signal an error
if any value is not a number.
\nwenddocs{}\nwbegincode{20}\sublabel{NWstb7-UtiH-5}\nwmargintag{{\nwtagstyle{}\subpageref{NWstb7-UtiH-5}}}\moddef{Utility functions~{\nwtagstyle{}\subpageref{NWstb7-UtiH-1}}}\plusendmoddef
(defun get-numbers-from (list)
" Method args: (list)
List should be list of edit-text-items. A list of values from each of
them is returned.  If any of them is invalid, an error is signalled."
  (let ((vals (mapcar #'(lambda(x) (val-of (send x :text))) list)))
     (unless (every #'numberp vals)
        (error "Non-numeric values in some items."))
     vals))
\nwindexdefn{get-numbers-from}{get-numbers-from}{NWstb7-UtiH-5}\eatline
\nwidentdefs{\\{{get-numbers-from}{get-numbers-from}}}\nwidentuses{\\{{val-of}{val-of}}}\nwindexuse{val-of}{val-of}{NWstb7-UtiH-5}\nwendcode{}\nwbegindocs{21}\nwdocspar
The functions {\tt{}get-values-from} and {\tt{}get-numbers-from} are useful
in dialogs. 

\subsection{Additional Useful Methods}
\label{sec:additional-methods}
The following additional methods for the built-in \xls/ prototypes
turn out be surprisingly useful in writing good looking dialogs.

\nwenddocs{}\nwbegincode{22}\sublabel{NWstb7-Addg-1}\nwmargintag{{\nwtagstyle{}\subpageref{NWstb7-Addg-1}}}\moddef{Additional methods for built-in prototypes~{\nwtagstyle{}\subpageref{NWstb7-Addg-1}}}\endmoddef
(defmeth text-item-proto :width (&optional width)
"Method args: (&optional wid)
Sets or retrieves the width of a text-item."
  (if width
      (let ((sz (slot-value 'size)))
        (setf (slot-value 'size) (list width (select sz 1))))
    (select (slot-value 'size) 0)))
\nwindexdefn{:width}{:colwidth}{NWstb7-Addg-1}\eatline
\nwidentdefs{\\{{:width}{:colwidth}}}\nwalsodefined{\\{NWstb7-Addg-2}\\{NWstb7-Addg-3}}\nwused{\\{NWstb7-Cod4-1}\\{NWstb7-UtiE-1}}\nwendcode{}\nwbegindocs{23}\nwdocspar
\nwenddocs{}\nwbegincode{24}\sublabel{NWstb7-Addg-2}\nwmargintag{{\nwtagstyle{}\subpageref{NWstb7-Addg-2}}}\moddef{Additional methods for built-in prototypes~{\nwtagstyle{}\subpageref{NWstb7-Addg-1}}}\plusendmoddef
(defmeth edit-text-item-proto :width (&optional width)
"Method args: (&optional wid)
Sets or retrieves the width of a edit-text-item."
  (if width
      (let ((sz (slot-value 'size)))
        (setf (slot-value 'size) (list width (select sz 1))))
    (select (slot-value 'size) 0)))
\nwindexdefn{:width}{:colwidth}{NWstb7-Addg-1}\eatline
\nwidentdefs{\\{{:width}{:colwidth}}}\nwendcode{}\nwbegindocs{25}\nwdocspar
\nwenddocs{}\nwbegincode{26}\sublabel{NWstb7-Addg-3}\nwmargintag{{\nwtagstyle{}\subpageref{NWstb7-Addg-3}}}\moddef{Additional methods for built-in prototypes~{\nwtagstyle{}\subpageref{NWstb7-Addg-1}}}\plusendmoddef
(defmeth interval-scroll-item-proto :width (&optional width)
"Method args: (&optional width)
Sets or retrieves the width of an interval-scroll-item."
  (if width
      (let ((sz (slot-value 'size)))
        (setf (slot-value 'size) (list width (select sz 1))))
    (select (slot-value 'size) 0)))
\nwindexdefn{:width}{:colwidth}{NWstb7-Addg-1}\eatline
\nwidentdefs{\\{{:width}{:colwidth}}}\nwendcode{}\nwbegindocs{27}\nwdocspar

\subsection{The {\tt{}dist-plot-proto} Prototype}
\label{sec:distproto}
The prototype for defining various distributions is
{\tt{}dist-plot-proto}. The documentation string in the code provides
some help for users of the code. Let us discuss the slots in detail.
(You may want to look at the actual definition of the prototype below
now.)  The three slots {\tt{}dens}, {\tt{}cdf} and {\tt{}icdf} respectively
hold functions that return the density, cumulative distribution, and
inverse cumulative distribution functions.  We shall henceforth refer
to these functions are $f$, $F$ and $F^{-1}$ respectively. The slot
{\tt{}params} holds the parameters of the distribution. Slots {\tt{}l-point}
and {\tt{}r-point} hold two abscissa values between which the plot is
shaded using {\tt{}shade-color}. The plot is shaded only when at least
one of them is non-{\tt{}nil}. If {\tt{}l-point} is non-{\tt{}nil}, then the
plot is shaded to the left of {\tt{}l-point}. When {\tt{}r-point} is
non-{\tt{}nil}, the plot is shaded to the right of {\tt{}r-point}. If both
are non-{\tt{}nil}, then the plot is shaded between the two points. The
slot {\tt{}num-points} will indicate the number of points used in
plotting.  Every time a density plot is drawn, we want to display the
parameters of the distribution on the plot so as to uniquely identify
the distribution.  The slot {\tt{}params-print-format} will hold a string
that indicates the format in which the parameters are to be printed.
The slot {\tt{}params-display-loc} will indicate where on the plot the
parameters should be displayed. The slots {\tt{}answer} and
{\tt{}answer-display-loc} hold analagous information for the answer
string.  Finally, since a distribution might have infinite support, we
have to impose practical limits on the maximum and minimum
probabilities.  The density will only be drawn between $F^{-1}(p_0)$
and $F^{-1}(p_1)$ where $p_0$ and $p_1$ are the values contained in
the slots {\tt{}min-probability} and {\tt{}max-probability} respectively.

Note that {\tt{}dist-plot-proto} inherits from {\tt{}scatterplot-proto} and
so we have all the methods of {\tt{}scatterplot-proto} available for
{\tt{}dist-plot-proto}. In particular, much of the graph drawing is
really the responsibility of {\tt{}scatterplot-proto}. 

\nwenddocs{}\nwbegincode{28}\sublabel{NWstb7-DisX-1}\nwmargintag{{\nwtagstyle{}\subpageref{NWstb7-DisX-1}}}\moddef{Distribution prototype definition~{\nwtagstyle{}\subpageref{NWstb7-DisX-1}}}\endmoddef
(defproto dist-plot-proto 
          '(dens cdf icdf params l-point r-point shade-color
                 num-points important-abscissae
                 params-print-format params-display-loc
                 answer answer-display-loc 
                 min-probability max-probability)
           () scatterplot-proto
"The distribution plot prototype. The slots dens, cdf, and icdf
respectively hold the function that calculate the density, 
cumulative distribution and inverse of the cumulative distribution
functions. The slot params holds the parameters for the 
distribution. L-point and r-point hold values between which the
plot must be shaded under the density with shade-color. Num-points
indicates how many points must be used to plot the density. The slot
important-abscissae is a list of points that need to be highlighted.
The strings params-print-format and answer indicate how
the parameters and answers should be printed on the plot while the
loc-slots hold the location on the plot where they should be 
printed. Min-probability and max-probability indicate the
practical support of the density.")
\nwindexdefn{dist-plot-proto}{dist-plot-proto}{NWstb7-DisX-1}\eatline
\nwindexdefn{dens}{dens}{NWstb7-DisX-1}\nwindexdefn{cdf}{cdf}{NWstb7-DisX-1}\nwindexdefn{icdf}{icdf}{NWstb7-DisX-1}\nwindexdefn{params}{params}{NWstb7-DisX-1}\nwindexdefn{l-point}{l-point}{NWstb7-DisX-1}\nwindexdefn{r-point}{r-point}{NWstb7-DisX-1}\nwindexdefn{shade-color}{shade-color}{NWstb7-DisX-1}\eatline
\nwindexdefn{num-points}{num-points}{NWstb7-DisX-1}\nwindexdefn{important-abscissae}{important-abscissae}{NWstb7-DisX-1}\eatline
\nwindexdefn{params-print-format}{params-print-format}{NWstb7-DisX-1}\nwindexdefn{params-display-loc}{params-display-loc}{NWstb7-DisX-1}\eatline
\nwindexdefn{answer}{answer}{NWstb7-DisX-1}\nwindexdefn{answer-display-loc}{answer-display-loc}{NWstb7-DisX-1}\eatline
\nwindexdefn{min-probability}{min-probability}{NWstb7-DisX-1}\nwindexdefn{max-probability}{max-probability}{NWstb7-DisX-1}\eatline
\nwidentdefs{\\{{answer}{answer}}\\{{answer-display-loc}{answer-display-loc}}\\{{cdf}{cdf}}\\{{dens}{dens}}\\{{dist-plot-proto}{dist-plot-proto}}\\{{icdf}{icdf}}\\{{important-abscissae}{important-abscissae}}\\{{l-point}{l-point}}\\{{max-probability}{max-probability}}\\{{min-probability}{min-probability}}\\{{num-points}{num-points}}\\{{params}{params}}\\{{params-display-loc}{params-display-loc}}\\{{params-print-format}{params-print-format}}\\{{r-point}{r-point}}\\{{shade-color}{shade-color}}}\nwused{\\{NWstb7-Cod4-1}\\{NWstb7-DisN-1}}\nwendcode{}\nwbegindocs{29}\nwdocspar

\subsection{Methods for {\tt{}dist-plot-proto}}
\label{sec:distprotomethods}
The methods for using {\tt{}dist-plot-proto} can be categorized as
follows.

\nwenddocs{}\nwbegincode{30}\sublabel{NWstb7-DisU-1}\nwmargintag{{\nwtagstyle{}\subpageref{NWstb7-DisU-1}}}\moddef{Distribution prototype methods~{\nwtagstyle{}\subpageref{NWstb7-DisU-1}}}\endmoddef
\LA{}Distribution prototype accessor and modifier methods~{\nwtagstyle{}\subpageref{NWstb7-Disq-1}}\RA{}
\LA{}Other useful methods for distribution prototype~{\nwtagstyle{}\subpageref{NWstb7-Othl-1}}\RA{}
\LA{}Distribution prototype :isnew method~{\nwtagstyle{}\subpageref{NWstb7-Disa-1}}\RA{}
\LA{}Distribution prototype redrawing methods~{\nwtagstyle{}\subpageref{NWstb7-Dise-1}}\RA{}
\nwused{\\{NWstb7-Cod4-1}\\{NWstb7-DisN-1}}\nwendcode{}\nwbegindocs{31}\nwdocspar

\subsubsection{Accessor and Modifier Methods.}
\label{sec:accessors}
The following methods provide access to the slots of {\tt{}dist-plot-proto}.

\nwenddocs{}\nwbegincode{32}\sublabel{NWstb7-Disq-1}\nwmargintag{{\nwtagstyle{}\subpageref{NWstb7-Disq-1}}}\moddef{Distribution prototype accessor and modifier methods~{\nwtagstyle{}\subpageref{NWstb7-Disq-1}}}\endmoddef
(defmeth dist-plot-proto :params (&optional params)
  "Method args: (&optional params)
Sets or retrieves the parameters for the distribution."
  (if params
      (setf (slot-value 'params) params)
    (slot-value 'params)))
\nwindexdefn{:params}{:colparams}{NWstb7-Disq-1}\eatline
\nwidentdefs{\\{{:params}{:colparams}}}\nwidentuses{\\{{dist-plot-proto}{dist-plot-proto}}\\{{params}{params}}}\nwindexuse{dist-plot-proto}{dist-plot-proto}{NWstb7-Disq-1}\nwindexuse{params}{params}{NWstb7-Disq-1}\nwalsodefined{\\{NWstb7-Disq-2}\\{NWstb7-Disq-3}\\{NWstb7-Disq-4}\\{NWstb7-Disq-5}\\{NWstb7-Disq-6}\\{NWstb7-Disq-7}\\{NWstb7-Disq-8}\\{NWstb7-Disq-9}\\{NWstb7-Disq-A}\\{NWstb7-Disq-B}\\{NWstb7-Disq-C}\\{NWstb7-Disq-D}\\{NWstb7-Disq-E}\\{NWstb7-Disq-F}}\nwused{\\{NWstb7-DisU-1}}\nwendcode{}\nwbegindocs{33}\nwdocspar
\nwenddocs{}\nwbegincode{34}\sublabel{NWstb7-Disq-2}\nwmargintag{{\nwtagstyle{}\subpageref{NWstb7-Disq-2}}}\moddef{Distribution prototype accessor and modifier methods~{\nwtagstyle{}\subpageref{NWstb7-Disq-1}}}\plusendmoddef
(defmeth dist-plot-proto :answer (&optional str)
  "Method args: (&optional str)
Sets or retrieves the answer string."
  (if str
      (setf (slot-value 'answer) str)
    (slot-value 'answer)))
\nwindexdefn{:answer}{:colanswer}{NWstb7-Disq-2}\eatline
\nwidentdefs{\\{{:answer}{:colanswer}}}\nwidentuses{\\{{answer}{answer}}\\{{dist-plot-proto}{dist-plot-proto}}}\nwindexuse{answer}{answer}{NWstb7-Disq-2}\nwindexuse{dist-plot-proto}{dist-plot-proto}{NWstb7-Disq-2}\nwendcode{}\nwbegindocs{35}\nwdocspar
\nwenddocs{}\nwbegincode{36}\sublabel{NWstb7-Disq-3}\nwmargintag{{\nwtagstyle{}\subpageref{NWstb7-Disq-3}}}\moddef{Distribution prototype accessor and modifier methods~{\nwtagstyle{}\subpageref{NWstb7-Disq-1}}}\plusendmoddef
(defmeth dist-plot-proto :answer-display-loc (&optional loc)
  "Method args: (&optional loc)
Sets or retrieves the answer-display-loc slot."
  (if loc
     (setf (slot-value 'answer-display-loc) loc)
  (slot-value 'answer-display-loc)))
\nwindexdefn{:answer-display-loc}{:colanswer-display-loc}{NWstb7-Disq-3}\eatline
\nwidentdefs{\\{{:answer-display-loc}{:colanswer-display-loc}}}\nwidentuses{\\{{:answer}{:colanswer}}\\{{answer}{answer}}\\{{answer-display-loc}{answer-display-loc}}\\{{dist-plot-proto}{dist-plot-proto}}}\nwindexuse{:answer}{:colanswer}{NWstb7-Disq-3}\nwindexuse{answer}{answer}{NWstb7-Disq-3}\nwindexuse{answer-display-loc}{answer-display-loc}{NWstb7-Disq-3}\nwindexuse{dist-plot-proto}{dist-plot-proto}{NWstb7-Disq-3}\nwendcode{}\nwbegindocs{37}\nwdocspar
\nwenddocs{}\nwbegincode{38}\sublabel{NWstb7-Disq-4}\nwmargintag{{\nwtagstyle{}\subpageref{NWstb7-Disq-4}}}\moddef{Distribution prototype accessor and modifier methods~{\nwtagstyle{}\subpageref{NWstb7-Disq-1}}}\plusendmoddef
(defmeth dist-plot-proto :params-print-format (&optional str)
  "Method args: (&optional str)
Sets or retrieves the parameter-print-format string."
  (if str
      (setf (slot-value 'params-print-format) str)
    (slot-value 'params-print-format)))
\nwindexdefn{:params-print-format}{:colparams-print-format}{NWstb7-Disq-4}\eatline
\nwidentdefs{\\{{:params-print-format}{:colparams-print-format}}}\nwidentuses{\\{{dist-plot-proto}{dist-plot-proto}}\\{{:params}{:colparams}}\\{{params}{params}}\\{{params-print-format}{params-print-format}}}\nwindexuse{dist-plot-proto}{dist-plot-proto}{NWstb7-Disq-4}\nwindexuse{:params}{:colparams}{NWstb7-Disq-4}\nwindexuse{params}{params}{NWstb7-Disq-4}\nwindexuse{params-print-format}{params-print-format}{NWstb7-Disq-4}\nwendcode{}\nwbegindocs{39}\nwdocspar
\nwenddocs{}\nwbegincode{40}\sublabel{NWstb7-Disq-5}\nwmargintag{{\nwtagstyle{}\subpageref{NWstb7-Disq-5}}}\moddef{Distribution prototype accessor and modifier methods~{\nwtagstyle{}\subpageref{NWstb7-Disq-1}}}\plusendmoddef
(defmeth dist-plot-proto :params-display-loc (&optional loc)
  "Method args: (&optional loc)
Sets or retrieves the params-display-loc slot."
  (if loc
     (setf (slot-value 'params-display-loc) loc)
  (slot-value 'params-display-loc)))
\nwindexdefn{:params-display-loc}{:colparams-display-loc}{NWstb7-Disq-5}\eatline
\nwidentdefs{\\{{:params-display-loc}{:colparams-display-loc}}}\nwidentuses{\\{{dist-plot-proto}{dist-plot-proto}}\\{{:params}{:colparams}}\\{{params}{params}}\\{{params-display-loc}{params-display-loc}}}\nwindexuse{dist-plot-proto}{dist-plot-proto}{NWstb7-Disq-5}\nwindexuse{:params}{:colparams}{NWstb7-Disq-5}\nwindexuse{params}{params}{NWstb7-Disq-5}\nwindexuse{params-display-loc}{params-display-loc}{NWstb7-Disq-5}\nwendcode{}\nwbegindocs{41}\nwdocspar
\nwenddocs{}\nwbegincode{42}\sublabel{NWstb7-Disq-6}\nwmargintag{{\nwtagstyle{}\subpageref{NWstb7-Disq-6}}}\moddef{Distribution prototype accessor and modifier methods~{\nwtagstyle{}\subpageref{NWstb7-Disq-1}}}\plusendmoddef
(defmeth dist-plot-proto :dens (&optional dens)
  "Method args: (&optional dens)
Sets or retrieves the density."
  (if dens
      (setf (slot-value 'dens) dens)
    (slot-value 'dens)))
\nwindexdefn{:dens}{:coldens}{NWstb7-Disq-6}\eatline
\nwidentdefs{\\{{:dens}{:coldens}}}\nwidentuses{\\{{dens}{dens}}\\{{dist-plot-proto}{dist-plot-proto}}}\nwindexuse{dens}{dens}{NWstb7-Disq-6}\nwindexuse{dist-plot-proto}{dist-plot-proto}{NWstb7-Disq-6}\nwendcode{}\nwbegindocs{43}\nwdocspar
\nwenddocs{}\nwbegincode{44}\sublabel{NWstb7-Disq-7}\nwmargintag{{\nwtagstyle{}\subpageref{NWstb7-Disq-7}}}\moddef{Distribution prototype accessor and modifier methods~{\nwtagstyle{}\subpageref{NWstb7-Disq-1}}}\plusendmoddef
(defmeth dist-plot-proto :cdf (&optional cdf)
  "Method args: (&optional cdf)
Sets or retrieves the CDF."
  (if cdf
      (setf (slot-value 'cdf) cdf)
    (slot-value 'cdf)))
\nwindexdefn{:cdf}{:colcdf}{NWstb7-Disq-7}\eatline
\nwidentdefs{\\{{:cdf}{:colcdf}}}\nwidentuses{\\{{cdf}{cdf}}\\{{dist-plot-proto}{dist-plot-proto}}}\nwindexuse{cdf}{cdf}{NWstb7-Disq-7}\nwindexuse{dist-plot-proto}{dist-plot-proto}{NWstb7-Disq-7}\nwendcode{}\nwbegindocs{45}\nwdocspar
\nwenddocs{}\nwbegincode{46}\sublabel{NWstb7-Disq-8}\nwmargintag{{\nwtagstyle{}\subpageref{NWstb7-Disq-8}}}\moddef{Distribution prototype accessor and modifier methods~{\nwtagstyle{}\subpageref{NWstb7-Disq-1}}}\plusendmoddef
(defmeth dist-plot-proto :icdf (&optional icdf)
  "Method args: (&optional icdf)
Sets or retrieves the Inverse CDF."
  (if icdf
      (setf (slot-value 'icdf) icdf)
    (slot-value 'icdf)))
\nwindexdefn{:icdf}{:colicdf}{NWstb7-Disq-8}\eatline
\nwidentdefs{\\{{:icdf}{:colicdf}}}\nwidentuses{\\{{dist-plot-proto}{dist-plot-proto}}\\{{icdf}{icdf}}}\nwindexuse{dist-plot-proto}{dist-plot-proto}{NWstb7-Disq-8}\nwindexuse{icdf}{icdf}{NWstb7-Disq-8}\nwendcode{}\nwbegindocs{47}\nwdocspar
\nwenddocs{}\nwbegincode{48}\sublabel{NWstb7-Disq-9}\nwmargintag{{\nwtagstyle{}\subpageref{NWstb7-Disq-9}}}\moddef{Distribution prototype accessor and modifier methods~{\nwtagstyle{}\subpageref{NWstb7-Disq-1}}}\plusendmoddef
(defmeth dist-plot-proto :num-points (&optional num-points)
  "Method args: (&optional num-points)
Sets or retrieves the slot num-points."
  (if num-points
      (setf (slot-value 'num-points) num-points)
    (slot-value 'num-points)))
\nwindexdefn{:num-points}{:colnum-points}{NWstb7-Disq-9}\eatline
\nwidentdefs{\\{{:num-points}{:colnum-points}}}\nwidentuses{\\{{dist-plot-proto}{dist-plot-proto}}\\{{num-points}{num-points}}}\nwindexuse{dist-plot-proto}{dist-plot-proto}{NWstb7-Disq-9}\nwindexuse{num-points}{num-points}{NWstb7-Disq-9}\nwendcode{}\nwbegindocs{49}\nwdocspar
\nwenddocs{}\nwbegincode{50}\sublabel{NWstb7-Disq-A}\nwmargintag{{\nwtagstyle{}\subpageref{NWstb7-Disq-A}}}\moddef{Distribution prototype accessor and modifier methods~{\nwtagstyle{}\subpageref{NWstb7-Disq-1}}}\plusendmoddef
(defmeth dist-plot-proto :important-abscissae (&optional list)
  "Method args: (&optional list)
Sets or retrieves the list of important abscissae."
  (if list
      (setf (slot-value 'important-abscissae) list)
    (slot-value 'important-abscissae)))
\nwindexdefn{:important-abscissae}{:colimportant-abscissae}{NWstb7-Disq-A}\eatline
\nwidentdefs{\\{{:important-abscissae}{:colimportant-abscissae}}}\nwidentuses{\\{{dist-plot-proto}{dist-plot-proto}}\\{{important-abscissae}{important-abscissae}}}\nwindexuse{dist-plot-proto}{dist-plot-proto}{NWstb7-Disq-A}\nwindexuse{important-abscissae}{important-abscissae}{NWstb7-Disq-A}\nwendcode{}\nwbegindocs{51}\nwdocspar
\nwenddocs{}\nwbegincode{52}\sublabel{NWstb7-Disq-B}\nwmargintag{{\nwtagstyle{}\subpageref{NWstb7-Disq-B}}}\moddef{Distribution prototype accessor and modifier methods~{\nwtagstyle{}\subpageref{NWstb7-Disq-1}}}\plusendmoddef
(defmeth dist-plot-proto :shade-color (&optional color)
  "Method args: (&optional color)
Sets or retrieves the shading color."
  (if color
      (setf (slot-value 'shade-color) color)
    (slot-value 'shade-color)))
\nwindexdefn{:shade-color}{:colshade-color}{NWstb7-Disq-B}\eatline
\nwidentdefs{\\{{:shade-color}{:colshade-color}}}\nwidentuses{\\{{dist-plot-proto}{dist-plot-proto}}\\{{shade-color}{shade-color}}}\nwindexuse{dist-plot-proto}{dist-plot-proto}{NWstb7-Disq-B}\nwindexuse{shade-color}{shade-color}{NWstb7-Disq-B}\nwendcode{}\nwbegindocs{53}\nwdocspar
\nwenddocs{}\nwbegincode{54}\sublabel{NWstb7-Disq-C}\nwmargintag{{\nwtagstyle{}\subpageref{NWstb7-Disq-C}}}\moddef{Distribution prototype accessor and modifier methods~{\nwtagstyle{}\subpageref{NWstb7-Disq-1}}}\plusendmoddef
(defmeth dist-plot-proto :l-point (&optional (point nil supplied-p))
  "Method args: (&optional (point nil supplied-p))
Sets or retrieves the slot l-point."
  (if supplied-p
      (setf (slot-value 'l-point) point)
    (slot-value 'l-point)))
\nwindexdefn{:l-point}{:coll-point}{NWstb7-Disq-C}\eatline
\nwidentdefs{\\{{:l-point}{:coll-point}}}\nwidentuses{\\{{dist-plot-proto}{dist-plot-proto}}\\{{l-point}{l-point}}}\nwindexuse{dist-plot-proto}{dist-plot-proto}{NWstb7-Disq-C}\nwindexuse{l-point}{l-point}{NWstb7-Disq-C}\nwendcode{}\nwbegindocs{55}\nwdocspar
\nwenddocs{}\nwbegincode{56}\sublabel{NWstb7-Disq-D}\nwmargintag{{\nwtagstyle{}\subpageref{NWstb7-Disq-D}}}\moddef{Distribution prototype accessor and modifier methods~{\nwtagstyle{}\subpageref{NWstb7-Disq-1}}}\plusendmoddef
(defmeth dist-plot-proto :r-point (&optional (point nil supplied-p))
  "Method args: (&optional (point nil supplied-p))
Sets or retrieves the slot r-point."
  (if supplied-p
      (setf (slot-value 'r-point) point)
    (slot-value 'r-point)))
\nwindexdefn{:r-point}{:colr-point}{NWstb7-Disq-D}\eatline
\nwidentdefs{\\{{:r-point}{:colr-point}}}\nwidentuses{\\{{dist-plot-proto}{dist-plot-proto}}\\{{r-point}{r-point}}}\nwindexuse{dist-plot-proto}{dist-plot-proto}{NWstb7-Disq-D}\nwindexuse{r-point}{r-point}{NWstb7-Disq-D}\nwendcode{}\nwbegindocs{57}\nwdocspar
\nwenddocs{}\nwbegincode{58}\sublabel{NWstb7-Disq-E}\nwmargintag{{\nwtagstyle{}\subpageref{NWstb7-Disq-E}}}\moddef{Distribution prototype accessor and modifier methods~{\nwtagstyle{}\subpageref{NWstb7-Disq-1}}}\plusendmoddef
(defmeth dist-plot-proto :min-probability (&optional val) 
  "Method args: (&optional val) 
Sets or retrieves the minimum probability for which quantiles 
can be calculated."  
  (if val 
      (setf (slot-value 'min-probability) val)
    (slot-value 'min-probability)))
\nwindexdefn{:min-probability}{:colmin-probability}{NWstb7-Disq-E}\eatline
\nwidentdefs{\\{{:min-probability}{:colmin-probability}}}\nwidentuses{\\{{dist-plot-proto}{dist-plot-proto}}\\{{min-probability}{min-probability}}}\nwindexuse{dist-plot-proto}{dist-plot-proto}{NWstb7-Disq-E}\nwindexuse{min-probability}{min-probability}{NWstb7-Disq-E}\nwendcode{}\nwbegindocs{59}\nwdocspar
\nwenddocs{}\nwbegincode{60}\sublabel{NWstb7-Disq-F}\nwmargintag{{\nwtagstyle{}\subpageref{NWstb7-Disq-F}}}\moddef{Distribution prototype accessor and modifier methods~{\nwtagstyle{}\subpageref{NWstb7-Disq-1}}}\plusendmoddef
(defmeth dist-plot-proto :max-probability (&optional val) 
  "Method args: (&optional val) 
Sets or retrieves the maximum probability for which quantiles 
can be calculated."  
  (if val
      (setf (slot-value 'max-probability) val)
    (slot-value 'max-probability)))
\nwindexdefn{:max-probability}{:colmax-probability}{NWstb7-Disq-F}\eatline
\nwidentdefs{\\{{:max-probability}{:colmax-probability}}}\nwidentuses{\\{{dist-plot-proto}{dist-plot-proto}}\\{{max-probability}{max-probability}}}\nwindexuse{dist-plot-proto}{dist-plot-proto}{NWstb7-Disq-F}\nwindexuse{max-probability}{max-probability}{NWstb7-Disq-F}\nwendcode{}\nwbegindocs{61}\nwdocspar

\subsubsection{Other Useful Methods.}
\label{sec:otherusefulmethods}
The density must be drawn on $[x_{min}, x_{max}]$, where
$x_{min}=F^{-1}(p_0)$ and $x_{max}=F^{-1}(p_1)$, $p_0$ and $p_1$ being
the practical limits on the minimum and maximum probabilities.  So it
is useful to have two methods that return $x_{min}$ and $x_{max}$.

\nwenddocs{}\nwbegincode{62}\sublabel{NWstb7-Othl-1}\nwmargintag{{\nwtagstyle{}\subpageref{NWstb7-Othl-1}}}\moddef{Other useful methods for distribution prototype~{\nwtagstyle{}\subpageref{NWstb7-Othl-1}}}\endmoddef
(defmeth dist-plot-proto :xmin ()
  "Method args: None
Returns the minimum value of x used in plotting."
  (apply (slot-value 'icdf) (send self :min-probability) 
         (send self :params)))
\nwindexdefn{:xmin}{:colxmin}{NWstb7-Othl-1}\eatline
\nwidentdefs{\\{{:xmin}{:colxmin}}}\nwidentuses{\\{{dist-plot-proto}{dist-plot-proto}}\\{{:min-probability}{:colmin-probability}}\\{{min-probability}{min-probability}}\\{{:params}{:colparams}}\\{{params}{params}}}\nwindexuse{dist-plot-proto}{dist-plot-proto}{NWstb7-Othl-1}\nwindexuse{:min-probability}{:colmin-probability}{NWstb7-Othl-1}\nwindexuse{min-probability}{min-probability}{NWstb7-Othl-1}\nwindexuse{:params}{:colparams}{NWstb7-Othl-1}\nwindexuse{params}{params}{NWstb7-Othl-1}\nwalsodefined{\\{NWstb7-Othl-2}\\{NWstb7-Othl-3}\\{NWstb7-Othl-4}\\{NWstb7-Othl-5}\\{NWstb7-Othl-6}\\{NWstb7-Othl-7}\\{NWstb7-Othl-8}\\{NWstb7-Othl-9}\\{NWstb7-Othl-A}\\{NWstb7-Othl-B}\\{NWstb7-Othl-C}}\nwused{\\{NWstb7-DisU-1}}\nwendcode{}\nwbegindocs{63}\nwdocspar
\nwenddocs{}\nwbegincode{64}\sublabel{NWstb7-Othl-2}\nwmargintag{{\nwtagstyle{}\subpageref{NWstb7-Othl-2}}}\moddef{Other useful methods for distribution prototype~{\nwtagstyle{}\subpageref{NWstb7-Othl-1}}}\plusendmoddef
(defmeth dist-plot-proto :xmax ()
  "Method args: None
Returns the maximum value of x used in plotting."
  (apply (slot-value 'icdf) (send self :max-probability) 
         (send self :params)))
\nwindexdefn{:xmax}{:colxmax}{NWstb7-Othl-2}\eatline
\nwidentdefs{\\{{:xmax}{:colxmax}}}\nwidentuses{\\{{dist-plot-proto}{dist-plot-proto}}\\{{:max-probability}{:colmax-probability}}\\{{max-probability}{max-probability}}\\{{:params}{:colparams}}\\{{params}{params}}}\nwindexuse{dist-plot-proto}{dist-plot-proto}{NWstb7-Othl-2}\nwindexuse{:max-probability}{:colmax-probability}{NWstb7-Othl-2}\nwindexuse{max-probability}{max-probability}{NWstb7-Othl-2}\nwindexuse{:params}{:colparams}{NWstb7-Othl-2}\nwindexuse{params}{params}{NWstb7-Othl-2}\nwendcode{}\nwbegindocs{65}\nwdocspar
We will also need to calculate the density, {\tt{}cdf}, and
{\tt{}icdf} at a point.
\nwenddocs{}\nwbegincode{66}\sublabel{NWstb7-Othl-3}\nwmargintag{{\nwtagstyle{}\subpageref{NWstb7-Othl-3}}}\moddef{Other useful methods for distribution prototype~{\nwtagstyle{}\subpageref{NWstb7-Othl-1}}}\plusendmoddef
(defmeth dist-plot-proto :dens-at (x)
  "Method args: x 
Returns the value of the density at x."
  (apply (slot-value 'dens) x (send self :params)))
\nwindexdefn{:dens-at}{:coldens-at}{NWstb7-Othl-3}\eatline
\nwidentdefs{\\{{:dens-at}{:coldens-at}}}\nwidentuses{\\{{:dens}{:coldens}}\\{{dens}{dens}}\\{{dist-plot-proto}{dist-plot-proto}}\\{{:params}{:colparams}}\\{{params}{params}}}\nwindexuse{:dens}{:coldens}{NWstb7-Othl-3}\nwindexuse{dens}{dens}{NWstb7-Othl-3}\nwindexuse{dist-plot-proto}{dist-plot-proto}{NWstb7-Othl-3}\nwindexuse{:params}{:colparams}{NWstb7-Othl-3}\nwindexuse{params}{params}{NWstb7-Othl-3}\nwendcode{}\nwbegindocs{67}\nwdocspar
\nwenddocs{}\nwbegincode{68}\sublabel{NWstb7-Othl-4}\nwmargintag{{\nwtagstyle{}\subpageref{NWstb7-Othl-4}}}\moddef{Other useful methods for distribution prototype~{\nwtagstyle{}\subpageref{NWstb7-Othl-1}}}\plusendmoddef
(defmeth dist-plot-proto :cdf-at (x)
  "Method args: x 
Returns the value of the cdf at x."
  (apply (slot-value 'cdf) x (send self :params)))
\nwindexdefn{:cdf-at}{:colcdf-at}{NWstb7-Othl-4}\eatline
\nwidentdefs{\\{{:cdf-at}{:colcdf-at}}}\nwidentuses{\\{{:cdf}{:colcdf}}\\{{cdf}{cdf}}\\{{dist-plot-proto}{dist-plot-proto}}\\{{:params}{:colparams}}\\{{params}{params}}}\nwindexuse{:cdf}{:colcdf}{NWstb7-Othl-4}\nwindexuse{cdf}{cdf}{NWstb7-Othl-4}\nwindexuse{dist-plot-proto}{dist-plot-proto}{NWstb7-Othl-4}\nwindexuse{:params}{:colparams}{NWstb7-Othl-4}\nwindexuse{params}{params}{NWstb7-Othl-4}\nwendcode{}\nwbegindocs{69}\nwdocspar
\nwenddocs{}\nwbegincode{70}\sublabel{NWstb7-Othl-5}\nwmargintag{{\nwtagstyle{}\subpageref{NWstb7-Othl-5}}}\moddef{Other useful methods for distribution prototype~{\nwtagstyle{}\subpageref{NWstb7-Othl-1}}}\plusendmoddef
(defmeth dist-plot-proto :icdf-at (x)
  "Method args: x 
Returns the value of the icdf at x."
  (apply (slot-value 'icdf) x (send self :params)))
\nwindexdefn{:icdf-at}{:colicdf-at}{NWstb7-Othl-5}\eatline
\nwidentdefs{\\{{:icdf-at}{:colicdf-at}}}\nwidentuses{\\{{dist-plot-proto}{dist-plot-proto}}\\{{:icdf}{:colicdf}}\\{{icdf}{icdf}}\\{{:params}{:colparams}}\\{{params}{params}}}\nwindexuse{dist-plot-proto}{dist-plot-proto}{NWstb7-Othl-5}\nwindexuse{:icdf}{:colicdf}{NWstb7-Othl-5}\nwindexuse{icdf}{icdf}{NWstb7-Othl-5}\nwindexuse{:params}{:colparams}{NWstb7-Othl-5}\nwindexuse{params}{params}{NWstb7-Othl-5}\nwendcode{}\nwbegindocs{71}\nwdocspar

\subsubsection{The :isnew Method.}
\label{sec:distisnewmethod}
We are now ready to write our {\tt{}:isnew} method. This method
determines how we create an instance of the {\tt{}dist-plot-proto}
prototype. The mandatory arguments to this method are of course $f$,
$F$, and $F^{-1}$ specified via keyword arguments along with the
parameters of the distribution, and two functions {\tt{}prob-dialog} and
{\tt{}quant-dialog} which handle distribution specific dialogs. See also
sections~\ref{sec:normal}, \ref{sec:t}, \ref{sec:chisq}, and
\ref{sec:f} for more on these dialogs. But first, some sensible
defaults for $p_0$ and $p_1$.

\nwenddocs{}\nwbegincode{72}\sublabel{NWstb7-ImpO-1}\nwmargintag{{\nwtagstyle{}\subpageref{NWstb7-ImpO-1}}}\moddef{Implementation constants~{\nwtagstyle{}\subpageref{NWstb7-ImpO-1}}}\endmoddef
(defparameter *min-probability* 1e-7)
(defparameter *max-probability* (- 1 1e-7))
\nwindexdefn{*min-probability*}{*min-probability*}{NWstb7-ImpO-1}\eatline
\nwindexdefn{*max-probability*}{*max-probability*}{NWstb7-ImpO-1}\eatline
\nwidentdefs{\\{{*max-probability*}{*max-probability*}}\\{{*min-probability*}{*min-probability*}}}\nwidentuses{\\{{max-probability}{max-probability}}\\{{min-probability}{min-probability}}}\nwindexuse{max-probability}{max-probability}{NWstb7-ImpO-1}\nwindexuse{min-probability}{min-probability}{NWstb7-ImpO-1}\nwalsodefined{\\{NWstb7-ImpO-2}\\{NWstb7-ImpO-3}\\{NWstb7-ImpO-4}}\nwused{\\{NWstb7-Cod4-1}\\{NWstb7-DisN-1}}\nwendcode{}\nwbegindocs{73}\nwdocspar
to be the default $p_0$ and $p_1$ respectively. Let us also make the
default number of points to be plotted to be $50$ and the default
shade~color as {\tt{}magenta}.
\nwenddocs{}\nwbegincode{74}\sublabel{NWstb7-ImpO-2}\nwmargintag{{\nwtagstyle{}\subpageref{NWstb7-ImpO-2}}}\moddef{Implementation constants~{\nwtagstyle{}\subpageref{NWstb7-ImpO-1}}}\plusendmoddef
(defparameter *num-points* 50)
(defparameter *shade-color* 
  (if (screen-has-color)
     'magenta
    'black))
\nwindexdefn{*num-points*}{*num-points*}{NWstb7-ImpO-2}\eatline
\nwindexdefn{*shade-color*}{*shade-color*}{NWstb7-ImpO-2}\eatline
\nwidentdefs{\\{{*num-points*}{*num-points*}}\\{{*shade-color*}{*shade-color*}}}\nwidentuses{\\{{num-points}{num-points}}\\{{shade-color}{shade-color}}}\nwindexuse{num-points}{num-points}{NWstb7-ImpO-2}\nwindexuse{shade-color}{shade-color}{NWstb7-ImpO-2}\nwendcode{}\nwbegindocs{75}\nwdocspar
Note that the {\tt{}:isnew} method uses another prototype
{\tt{}button-overlay-proto} which is described in
section\ref{sec:button-proto}. Here is the beginning of the
{\tt{}:isnew} method with the arguments. 
\nwenddocs{}\nwbegincode{76}\sublabel{NWstb7-Disa-1}\nwmargintag{{\nwtagstyle{}\subpageref{NWstb7-Disa-1}}}\moddef{Distribution prototype :isnew method~{\nwtagstyle{}\subpageref{NWstb7-Disa-1}}}\endmoddef
(defmeth dist-plot-proto :isnew (&key dens cdf icdf params 
         prob-dialog quant-dialog
         (shade-color *shade-color*)
         (params-print-format "Parameters: ~g")
         (num-points *num-points*)
         (max-probability *max-probability*)
         (min-probability *min-probability*)
         (title "Probability Density"))
\nwidentuses{\\{{cdf}{cdf}}\\{{dens}{dens}}\\{{dist-plot-proto}{dist-plot-proto}}\\{{icdf}{icdf}}\\{{:isnew}{:colisnew}}\\{{*max-probability*}{*max-probability*}}\\{{max-probability}{max-probability}}\\{{*min-probability*}{*min-probability*}}\\{{min-probability}{min-probability}}\\{{*num-points*}{*num-points*}}\\{{num-points}{num-points}}\\{{params}{params}}\\{{params-print-format}{params-print-format}}\\{{*shade-color*}{*shade-color*}}\\{{shade-color}{shade-color}}}\nwindexuse{cdf}{cdf}{NWstb7-Disa-1}\nwindexuse{dens}{dens}{NWstb7-Disa-1}\nwindexuse{dist-plot-proto}{dist-plot-proto}{NWstb7-Disa-1}\nwindexuse{icdf}{icdf}{NWstb7-Disa-1}\nwindexuse{:isnew}{:colisnew}{NWstb7-Disa-1}\nwindexuse{*max-probability*}{*max-probability*}{NWstb7-Disa-1}\nwindexuse{max-probability}{max-probability}{NWstb7-Disa-1}\nwindexuse{*min-probability*}{*min-probability*}{NWstb7-Disa-1}\nwindexuse{min-probability}{min-probability}{NWstb7-Disa-1}\nwindexuse{*num-points*}{*num-points*}{NWstb7-Disa-1}\nwindexuse{num-points}{num-points}{NWstb7-Disa-1}\nwindexuse{params}{params}{NWstb7-Disa-1}\nwindexuse{params-print-format}{params-print-format}{NWstb7-Disa-1}\nwindexuse{*shade-color*}{*shade-color*}{NWstb7-Disa-1}\nwindexuse{shade-color}{shade-color}{NWstb7-Disa-1}\nwalsodefined{\\{NWstb7-Disa-2}\\{NWstb7-Disa-3}\\{NWstb7-Disa-4}\\{NWstb7-Disa-5}\\{NWstb7-Disa-6}}\nwused{\\{NWstb7-DisU-1}}\nwendcode{}\nwbegindocs{77}\nwdocspar
Now, on to the basic tasks the {\tt{}:isnew} method must perform. Clearly, we
need to store the supplied arguments in the respective slots.  
\nwenddocs{}\nwbegincode{78}\sublabel{NWstb7-Disa-2}\nwmargintag{{\nwtagstyle{}\subpageref{NWstb7-Disa-2}}}\moddef{Distribution prototype :isnew method~{\nwtagstyle{}\subpageref{NWstb7-Disa-1}}}\plusendmoddef
  (setf (slot-value 'dens) dens)
  (setf (slot-value 'cdf) cdf)
  (setf (slot-value 'icdf) icdf)
  (setf (slot-value 'shade-color) shade-color)
  (setf (slot-value 'params) params)
  (setf (slot-value 'params-print-format) params-print-format)
  (setf (slot-value 'num-points) num-points)
  (setf (slot-value 'min-probability) min-probability)
  (setf (slot-value 'max-probability) max-probability)
\nwidentuses{\\{{cdf}{cdf}}\\{{dens}{dens}}\\{{icdf}{icdf}}\\{{max-probability}{max-probability}}\\{{min-probability}{min-probability}}\\{{num-points}{num-points}}\\{{params}{params}}\\{{params-print-format}{params-print-format}}\\{{shade-color}{shade-color}}}\nwindexuse{cdf}{cdf}{NWstb7-Disa-2}\nwindexuse{dens}{dens}{NWstb7-Disa-2}\nwindexuse{icdf}{icdf}{NWstb7-Disa-2}\nwindexuse{max-probability}{max-probability}{NWstb7-Disa-2}\nwindexuse{min-probability}{min-probability}{NWstb7-Disa-2}\nwindexuse{num-points}{num-points}{NWstb7-Disa-2}\nwindexuse{params}{params}{NWstb7-Disa-2}\nwindexuse{params-print-format}{params-print-format}{NWstb7-Disa-2}\nwindexuse{shade-color}{shade-color}{NWstb7-Disa-2}\nwendcode{}\nwbegindocs{79}\nwdocspar
Then, we need to invoke the {\em inherited\/} {\tt{}:isnew} method of
{\tt{}scatterplot-proto} to actually draw the graphical window.
\nwenddocs{}\nwbegincode{80}\sublabel{NWstb7-Disa-3}\nwmargintag{{\nwtagstyle{}\subpageref{NWstb7-Disa-3}}}\moddef{Distribution prototype :isnew method~{\nwtagstyle{}\subpageref{NWstb7-Disa-1}}}\plusendmoddef
  (call-next-method 2 :title title)
  (when (screen-has-color) (send self :use-color t))
\nwendcode{}\nwbegindocs{81}\nwdocspar
We add two button overlays to the plot for calculating
probabilities and quantiles. We need to figure out the
locations of these buttons on the window.  A look at 
figure~\ref{fig:example} might help. We shall arrange it so that the
two buttons are on the left and right side of the window
respectively. The {\tt{}:isnew} method for {\tt{}button-overlay-proto} just
requires the coordinates of the left upper corner of a rectangular box
surrounding the button along with the string to be displayed and a
function which is called when the button is pressed. 
\nwenddocs{}\nwbegincode{82}\sublabel{NWstb7-Disa-4}\nwmargintag{{\nwtagstyle{}\subpageref{NWstb7-Disa-4}}}\moddef{Distribution prototype :isnew method~{\nwtagstyle{}\subpageref{NWstb7-Disa-1}}}\plusendmoddef
  (let* ((em (send self :text-width "m"))
         (ascent (send self :text-ascent))
         (descent (send self :text-descent))
         (prob (send button-overlay-proto :new em ascent
                     "Find Probability" 
                     #'(lambda() (funcall prob-dialog self))))
         (cw (send self :canvas-width))
         (qtw (send self :text-width "Find Quantile"))
         (quant (send button-overlay-proto :new (- cw em qtw em) ascent
                      "Find Quantile"
                       #'(lambda() (funcall quant-dialog self)))))
\nwendcode{}\nwbegindocs{83}\nwdocspar
Before we add the overlays to the plot, we need to allow enough space
in the margins for the display of the buttons, parameters and the
answer. So we make the window slightly bigger to accommodate
these quantities. Each line of text will occupy at most
$\mbox{{\tt{}ascent}} + \mbox{{\tt{}descent}}$ vertical space. For example,
the buttons will occupy $y + \mbox{{\tt{}ascent}} + \mbox{{\tt{}descent}} +
\mbox{{\tt{}em}}$ vertical space where $y$ is the y-coordinate of the
top-left corner of the button. (The extra {\tt{}em} is due to the fact
that there is a gap of $0.5\times\mbox{{\tt{}em}}$ all around the text in
the box). Let us also assume that each line of text takes up
$1.5\times(\mbox{{\tt{}ascent}} + \mbox{{\tt{}descent}})$ vertical space. So
we can calculate how much space we need in the margin at the top. We
also need some space in the bottom for drawing arrows that identify
the quantiles. Taking all these things into consideration, we are led
to the following code.
\nwenddocs{}\nwbegincode{84}\sublabel{NWstb7-Disa-5}\nwmargintag{{\nwtagstyle{}\subpageref{NWstb7-Disa-5}}}\moddef{Distribution prototype :isnew method~{\nwtagstyle{}\subpageref{NWstb7-Disa-1}}}\plusendmoddef
    (let* ((sz (send self :size))
           (ht (select sz 1))
           (top-margin (round (+ ascent ascent descent em
                                 (* 1.5 (+ ascent descent))
                                 (* 1.5 (+ ascent descent)))))
           (bot-margin (round (* 1.5 (+ ascent descent)))))
       (send self :size (select sz 0) (+ ht top-margin bot-margin))
       (send self :margin 0 top-margin 0 bot-margin))
    (send self :add-overlay prob)
    (send self :add-overlay quant)
\nwendcode{}\nwbegindocs{85}\nwdocspar
Finally, we need to determine the locations where the parameters and
the answer must be displayed. The answer string will be displayed
centered horizontally in the window, and so its location will just be
the $y$ coordinate.
\nwenddocs{}\nwbegincode{86}\sublabel{NWstb7-Disa-6}\nwmargintag{{\nwtagstyle{}\subpageref{NWstb7-Disa-6}}}\moddef{Distribution prototype :isnew method~{\nwtagstyle{}\subpageref{NWstb7-Disa-1}}}\plusendmoddef
    (send self :params-display-loc 
          (list em (+ ascent ascent descent em ascent descent)))
    (setf (slot-value 'answer-display-loc) 
          (+ ascent ascent descent em 
             (round (* 1.5 (+ ascent descent)))
             ascent descent))))
\nwindexdefn{:isnew}{:colisnew}{NWstb7-Disa-6}\eatline
\nwidentdefs{\\{{:isnew}{:colisnew}}}\nwidentuses{\\{{:params}{:colparams}}\\{{params}{params}}\\{{:params-display-loc}{:colparams-display-loc}}\\{{params-display-loc}{params-display-loc}}}\nwindexuse{:params}{:colparams}{NWstb7-Disa-6}\nwindexuse{params}{params}{NWstb7-Disa-6}\nwindexuse{:params-display-loc}{:colparams-display-loc}{NWstb7-Disa-6}\nwindexuse{params-display-loc}{params-display-loc}{NWstb7-Disa-6}\nwendcode{}\nwbegindocs{87}\nwdocspar

\subsubsection{Redrawing Methods.}
\label{sec:distredrawing}
To make sure that the plot redraws itself when it is moved around on
the screen, we need to write a {\tt{}:redraw} method.  Several sub-tasks
have to be addressed. These involve shading under the density,
displaying the parameters, and displaying the answer. In addition, we
may have to highlight certain quantiles along the $x$-axis. (By
highlighting, we mean drawing arrows that point to the values.) It is
best to write these sub-tasks as methods. The methods for displaying
the parameters and the answer are easy.
\nwenddocs{}\nwbegincode{88}\sublabel{NWstb7-Othl-6}\nwmargintag{{\nwtagstyle{}\subpageref{NWstb7-Othl-6}}}\moddef{Other useful methods for distribution prototype~{\nwtagstyle{}\subpageref{NWstb7-Othl-1}}}\plusendmoddef
(defmeth dist-plot-proto :display-params ()
  "Method args: (None)
Displays the parameters on the plot."
   (when (send self :params-display-loc)
      (let ((str (apply #'format nil (send self :params-print-format)
                        (send self :params))))
        (apply #'send self :draw-string str 
               (send self :params-display-loc)))))
\nwindexdefn{:display-params}{:coldisplay-params}{NWstb7-Othl-6}\eatline
\nwidentdefs{\\{{:display-params}{:coldisplay-params}}}\nwidentuses{\\{{dist-plot-proto}{dist-plot-proto}}\\{{:params}{:colparams}}\\{{params}{params}}\\{{:params-display-loc}{:colparams-display-loc}}\\{{params-display-loc}{params-display-loc}}\\{{:params-print-format}{:colparams-print-format}}\\{{params-print-format}{params-print-format}}}\nwindexuse{dist-plot-proto}{dist-plot-proto}{NWstb7-Othl-6}\nwindexuse{:params}{:colparams}{NWstb7-Othl-6}\nwindexuse{params}{params}{NWstb7-Othl-6}\nwindexuse{:params-display-loc}{:colparams-display-loc}{NWstb7-Othl-6}\nwindexuse{params-display-loc}{params-display-loc}{NWstb7-Othl-6}\nwindexuse{:params-print-format}{:colparams-print-format}{NWstb7-Othl-6}\nwindexuse{params-print-format}{params-print-format}{NWstb7-Othl-6}\nwendcode{}\nwbegindocs{89}\nwdocspar
\nwenddocs{}\nwbegincode{90}\sublabel{NWstb7-Othl-7}\nwmargintag{{\nwtagstyle{}\subpageref{NWstb7-Othl-7}}}\moddef{Other useful methods for distribution prototype~{\nwtagstyle{}\subpageref{NWstb7-Othl-1}}}\plusendmoddef
(defmeth dist-plot-proto :display-answer ()
  "Method args: (None)
Displays the answer centered horizontally on the plot."
  (let ((y (send self :answer-display-loc))
        (answer (send self :answer)))
     (when answer
        (let ((x (round (* 0.5 (- (send self :canvas-width)
                                  (send self :text-width answer))))))
           (send self :draw-string answer x y)))))
\nwindexdefn{:display-answer}{:coldisplay-answer}{NWstb7-Othl-7}\eatline
\nwidentdefs{\\{{:display-answer}{:coldisplay-answer}}}\nwidentuses{\\{{:answer}{:colanswer}}\\{{answer}{answer}}\\{{:answer-display-loc}{:colanswer-display-loc}}\\{{answer-display-loc}{answer-display-loc}}\\{{dist-plot-proto}{dist-plot-proto}}}\nwindexuse{:answer}{:colanswer}{NWstb7-Othl-7}\nwindexuse{answer}{answer}{NWstb7-Othl-7}\nwindexuse{:answer-display-loc}{:colanswer-display-loc}{NWstb7-Othl-7}\nwindexuse{answer-display-loc}{answer-display-loc}{NWstb7-Othl-7}\nwindexuse{dist-plot-proto}{dist-plot-proto}{NWstb7-Othl-7}\nwendcode{}\nwbegindocs{91}\nwdocspar
It is also straight-forward to write a method for drawing a vertical
arrow from point $(a,b)$ to $(a,c)$.
\nwenddocs{}\nwbegincode{92}\sublabel{NWstb7-Othl-8}\nwmargintag{{\nwtagstyle{}\subpageref{NWstb7-Othl-8}}}\moddef{Other useful methods for distribution prototype~{\nwtagstyle{}\subpageref{NWstb7-Othl-1}}}\plusendmoddef
(defmeth dist-plot-proto :draw-vert-arrow (a b c)
  "Method args: (a b c)
Draws a vertical arrow from (a b) ending at (a c). The arrow head will
be at (a c). The coordinates must be canvas coordinates."
  (send self :draw-line a b a c)
  (let* ((p (+ b (round (* 0.8 (- c b)))))
         (x1 (- a 10))
         (x2 (+ a 10)))
    (send self :paint-poly (list (list x1 p) (list x2 p) (list a c)))))
\nwindexdefn{:draw-vert-arrow}{:coldraw-vert-arrow}{NWstb7-Othl-8}\eatline
\nwidentdefs{\\{{:draw-vert-arrow}{:coldraw-vert-arrow}}}\nwidentuses{\\{{dist-plot-proto}{dist-plot-proto}}}\nwindexuse{dist-plot-proto}{dist-plot-proto}{NWstb7-Othl-8}\nwendcode{}\nwbegindocs{93}\nwdocspar

Using the method for drawing vertical arrows, we can now highlight
important $x$-values. We need the following constant.
\nwenddocs{}\nwbegincode{94}\sublabel{NWstb7-ImpO-3}\nwmargintag{{\nwtagstyle{}\subpageref{NWstb7-ImpO-3}}}\moddef{Implementation constants~{\nwtagstyle{}\subpageref{NWstb7-ImpO-1}}}\plusendmoddef
(defparameter *quantile-print-format* "~,3f")
\nwindexdefn{*quantile-print-format*}{*quantile-print-format*}{NWstb7-ImpO-3}\eatline
\nwidentdefs{\\{{*quantile-print-format*}{*quantile-print-format*}}}\nwendcode{}\nwbegindocs{95}\nwdocspar
Not surprisingly, we need a similar one for probability later; we
might as well define it here.
\nwenddocs{}\nwbegincode{96}\sublabel{NWstb7-ImpO-4}\nwmargintag{{\nwtagstyle{}\subpageref{NWstb7-ImpO-4}}}\moddef{Implementation constants~{\nwtagstyle{}\subpageref{NWstb7-ImpO-1}}}\plusendmoddef
(defparameter *probability-print-format* "~,3f")
\nwindexdefn{*probability-print-format*}{*probability-print-format*}{NWstb7-ImpO-4}\eatline
\nwidentdefs{\\{{*probability-print-format*}{*probability-print-format*}}}\nwendcode{}\nwbegindocs{97}\nwdocspar
So here is our method for highlighting some abscissae.
\nwenddocs{}\nwbegincode{98}\sublabel{NWstb7-Othl-9}\nwmargintag{{\nwtagstyle{}\subpageref{NWstb7-Othl-9}}}\moddef{Other useful methods for distribution prototype~{\nwtagstyle{}\subpageref{NWstb7-Othl-1}}}\plusendmoddef
(defmeth dist-plot-proto :highlight-important-abscissae ()
  "Method args: None
Draws a vertical arrows highlighting the abscissae in the slot
important-abscissae." 
  (when (send self :important-abscissae)
        (let* ((list (send self :important-abscissae))
               (ascent (send self :text-ascent))
               (descent (send self :text-descent))
               (ht (send self :canvas-height))
               (arrow-start-y (- ht (round (* 1.5 (+ ascent descent)))))
               (str-start-y (- ht (round (* 0.35 (+ ascent descent))))))
           (dolist (val list)
             (let* ((coord (send self :real-to-canvas val 0.0))
                    (x (select coord 0))
                    (y (select coord 1))
                    (str (format nil *quantile-print-format* val))
                    (str-wid (send self :text-width str)))
               (send self :draw-string str 
                 (- x (round (* 0.5 str-wid))) str-start-y)
               (send self :draw-vert-arrow x arrow-start-y y))))))
\nwindexdefn{:highlight-important-abscissae}{:colhighlight-important-abscissae}{NWstb7-Othl-9}\eatline
\nwidentdefs{\\{{:highlight-important-abscissae}{:colhighlight-important-abscissae}}}\nwidentuses{\\{{dist-plot-proto}{dist-plot-proto}}\\{{:draw-vert-arrow}{:coldraw-vert-arrow}}\\{{:important-abscissae}{:colimportant-abscissae}}\\{{important-abscissae}{important-abscissae}}\\{{*quantile-print-format*}{*quantile-print-format*}}}\nwindexuse{dist-plot-proto}{dist-plot-proto}{NWstb7-Othl-9}\nwindexuse{:draw-vert-arrow}{:coldraw-vert-arrow}{NWstb7-Othl-9}\nwindexuse{:important-abscissae}{:colimportant-abscissae}{NWstb7-Othl-9}\nwindexuse{important-abscissae}{important-abscissae}{NWstb7-Othl-9}\nwindexuse{*quantile-print-format*}{*quantile-print-format*}{NWstb7-Othl-9}\nwendcode{}\nwbegindocs{99}\nwdocspar

Let us now tackle shading. Recall that if {\tt{}l-point} is non-{\tt{}nil},
we want to shade to the left, and if {\tt{}r-point} is non-{\tt{}nil}, we
want to shade to the right, or if both are non-{\tt{}nil}, we want to
shade between. If both are {\tt{}nil}, then we must skip shading.
\nwenddocs{}\nwbegincode{100}\sublabel{NWstb7-Othl-A}\nwmargintag{{\nwtagstyle{}\subpageref{NWstb7-Othl-A}}}\moddef{Other useful methods for distribution prototype~{\nwtagstyle{}\subpageref{NWstb7-Othl-1}}}\plusendmoddef
(defmeth dist-plot-proto :shade-under-plot ()
  "Shades the region under the curve determined by l-point and r-point.
If both are non-nil, shades between, otherwise to the left or right
as the case may be. Does nothing if both l-point, r-point are
nil. Note that a must be < b if both are non-nil."
  (when (or (send self :l-point) (send self :r-point))
     (let ((x (mapcar #'(lambda(x) (send self :linestart-coordinate 0 x))
                      (iseq (send self :num-lines))))
           (y (mapcar #'(lambda(x) (send self :linestart-coordinate 1 x))
                        (iseq (send self :num-lines))))
           (a (send self :l-point))
           (b (send self :r-point)))
\nwidentuses{\\{{dist-plot-proto}{dist-plot-proto}}\\{{:l-point}{:coll-point}}\\{{l-point}{l-point}}\\{{:r-point}{:colr-point}}\\{{r-point}{r-point}}\\{{:shade-under-plot}{:colshade-under-plot}}}\nwindexuse{dist-plot-proto}{dist-plot-proto}{NWstb7-Othl-A}\nwindexuse{:l-point}{:coll-point}{NWstb7-Othl-A}\nwindexuse{l-point}{l-point}{NWstb7-Othl-A}\nwindexuse{:r-point}{:colr-point}{NWstb7-Othl-A}\nwindexuse{r-point}{r-point}{NWstb7-Othl-A}\nwindexuse{:shade-under-plot}{:colshade-under-plot}{NWstb7-Othl-A}\nwendcode{}\nwbegindocs{101}\nwdocspar
So how are we going to handle shading between two points? Well, the
picture\footnote{With perhaps the faint hope that $\mbox{Art} +
  \mbox{Literate Programming} = \mbox{Art of Computer Programming}\ldots$}
in figure~\ref{fig:shading} illustrates the issues.

\vspace{.5in}
\begin{figure}[htbp]
 \begin{center}
    \leavevmode

    \psset{xunit=2cm,yunit=2cm}
    \psplot[plotpoints=100]{-3}{3}{2.718281828 x x mul neg exp}
    \psline{-}(-3,0)(3,0)
    \psline{-}(-1,0)(-1,.3679)
    \psline{-}(1,0)(1,.3679)
    \uput[0](-1.15,-0.1) {$a$}
    \uput[0](-.85,-0.1) {$x_{m+1}$}
    \psline{-}(-.7,0) (-.7,.05)
    \uput[0](-1.5,-0.1) {$x_{m}$}
    \psline{-}(-1.35,0) (-1.35,.05)
    \uput[0](.85,-0.1) {$b$}
    \uput[0](.6,-0.1) {$x_n$}
    \psline{-}(.75,0) (.75,.05)
    \uput[0](1.2,-0.1) {$x_{n+1}$}
    \psline{-}(1.35,0) (1.35,.05)
  \end{center}
  \caption{The Normal density function.}
  \label{fig:shading}
\end{figure}
%\vspace{0.5cm}

Suppose we wish to shade between $a$ and $b$ where $a< b$. Let
$x_1,x_2,\ldots,x_k$ denote the sequence of $x$-values which were used
in plotting the density. Set $x_0=-\infty$ and $x_{k+1}=\infty$. Then,
we need to find indices $m$ and $n$ such that $x_m < a\leq x_{m+1}$
and $x_n \leq b < x_{n+1}$. We need to shade the polygonal region
$(a,0), (a,f(a)), (x_{m+1},f(x_{m+1})),\ldots,
(x_n,f(x_n)), (b,f(b)), (b,0)$. Also, since $a$ and $b$ will be in
real coordinates, we have to convert them to canvas coordinates. There
is a subtle issue one has to watch for when shading: the shading must
be done {\em after\/} adjusting the plot to the newly redrawn density.

\nwenddocs{}\nwbegincode{102}\sublabel{NWstb7-Othl-B}\nwmargintag{{\nwtagstyle{}\subpageref{NWstb7-Othl-B}}}\moddef{Other useful methods for distribution prototype~{\nwtagstyle{}\subpageref{NWstb7-Othl-1}}}\plusendmoddef
        (cond
         ((and a b) ; We need to shade between.
          (let ((m+1 (position a x :test #'<=))
                (n (position b x :test #'>= :from-end t)))
            (unless (and m+1 n (>= n m+1))
                 (error "Bad left and right end points."))
            (let ((v-list (list
                           (send self :real-to-canvas a 0)
                           (send self :real-to-canvas a 
                                 (send self :dens-at a))))
                  (middle 
                   (select (mapcar #'(lambda(x y)
                            (send self :real-to-canvas x y)) x y)
                           (iseq m+1 n)))
                  (end (list
                        (send self :real-to-canvas b 
                              (send self :dens-at b))
                        (send self :real-to-canvas b 0)))
                  (dc (send self :draw-color)))
              (setf v-list (append v-list middle))
              (setf v-list (append v-list end))
              (send self :draw-color (send self :shade-color))
              (send self :paint-poly v-list)
              (send self :draw-color dc))))
\nwidentuses{\\{{:dens}{:coldens}}\\{{dens}{dens}}\\{{:dens-at}{:coldens-at}}\\{{:shade-color}{:colshade-color}}\\{{shade-color}{shade-color}}}\nwindexuse{:dens}{:coldens}{NWstb7-Othl-B}\nwindexuse{dens}{dens}{NWstb7-Othl-B}\nwindexuse{:dens-at}{:coldens-at}{NWstb7-Othl-B}\nwindexuse{:shade-color}{:colshade-color}{NWstb7-Othl-B}\nwindexuse{shade-color}{shade-color}{NWstb7-Othl-B}\nwendcode{}\nwbegindocs{103}\nwdocspar
Next, we need to handle shading to the left or shading to the
right. In light of the above discussion, these two tasks are
straight-forward. 
\nwenddocs{}\nwbegincode{104}\sublabel{NWstb7-Othl-C}\nwmargintag{{\nwtagstyle{}\subpageref{NWstb7-Othl-C}}}\moddef{Other useful methods for distribution prototype~{\nwtagstyle{}\subpageref{NWstb7-Othl-1}}}\plusendmoddef
         (a ; We need to shade to the left.
          (let ((m+1 (position a x :test #'>= :from-end t)))
            (when m+1
              (let ((v-list 
                     (list 
                       (send self :real-to-canvas (select x 0) 0.0)))
                    (middle 
                     (select (mapcar #'(lambda(x y)
                               (send self :real-to-canvas x y)) x y)
                       (iseq m+1)))
                     (end (list
                           (send self :real-to-canvas a 
                                 (send self :dens-at a))
                           (send self :real-to-canvas a 0.0)))
                    (dc (send self :draw-color)))
                (setf v-list (append v-list middle))
                (setf v-list (append v-list end))
                (send self :draw-color (send self :shade-color))
                (send self :paint-poly v-list)
                (send self :draw-color dc)))))
         (b ; We need to shade to the right.
          (let ((n (position b x :test #'<)))
            (when n
              (let ((v-list 
                     (list
                       (send self :real-to-canvas b 0)
                       (send self :real-to-canvas b (send self
                       :dens-at b))))
                    (end
                     (select (mapcar #'(lambda(x y)
                                (send self :real-to-canvas x y)) x y)
                             (iseq n (1- (send self :num-lines)))))
                    (dc (send self :draw-color)))
                (setf v-list (append v-list end))
                (send self :draw-color (send self :shade-color))
                (send self :paint-poly v-list)
                (send self :draw-color dc)))))))))
\nwindexdefn{:shade-under-plot}{:colshade-under-plot}{NWstb7-Othl-C}\eatline
\nwidentdefs{\\{{:shade-under-plot}{:colshade-under-plot}}}\nwidentuses{\\{{:dens}{:coldens}}\\{{dens}{dens}}\\{{:dens-at}{:coldens-at}}\\{{:shade-color}{:colshade-color}}\\{{shade-color}{shade-color}}}\nwindexuse{:dens}{:coldens}{NWstb7-Othl-C}\nwindexuse{dens}{dens}{NWstb7-Othl-C}\nwindexuse{:dens-at}{:coldens-at}{NWstb7-Othl-C}\nwindexuse{:shade-color}{:colshade-color}{NWstb7-Othl-C}\nwindexuse{shade-color}{shade-color}{NWstb7-Othl-C}\nwendcode{}\nwbegindocs{105}\nwdocspar
That concludes the {\tt{}:shade-under-plot} method.

Next, the {\tt{}:redraw-content} method, which redraws the contents of
the plot. Basically, we have to clear everything in the plot, redraw
the density, adjust the axes, redraw the axes, shade under the plot,
display the parameters of the distribution and the answer. Since the
{\tt{}:redraw-background} method is responsible for drawing the
background, we write a new {\tt{}:redraw-background} method for
{\tt{}dist-plot-proto}. 
\nwenddocs{}\nwbegincode{106}\sublabel{NWstb7-Dise-1}\nwmargintag{{\nwtagstyle{}\subpageref{NWstb7-Dise-1}}}\moddef{Distribution prototype redrawing methods~{\nwtagstyle{}\subpageref{NWstb7-Dise-1}}}\endmoddef
(defmeth dist-plot-proto :redraw-background ()
  (call-next-method)
  (send self :display-params)
  (send self :display-answer))
\nwindexdefn{:redraw-background}{:colredraw-background}{NWstb7-Dise-1}\eatline
\nwidentdefs{\\{{:redraw-background}{:colredraw-background}}}\nwidentuses{\\{{answer}{answer}}\\{{:display-answer}{:coldisplay-answer}}\\{{:display-params}{:coldisplay-params}}\\{{dist-plot-proto}{dist-plot-proto}}\\{{params}{params}}\\{{:redraw}{:colredraw}}}\nwindexuse{answer}{answer}{NWstb7-Dise-1}\nwindexuse{:display-answer}{:coldisplay-answer}{NWstb7-Dise-1}\nwindexuse{:display-params}{:coldisplay-params}{NWstb7-Dise-1}\nwindexuse{dist-plot-proto}{dist-plot-proto}{NWstb7-Dise-1}\nwindexuse{params}{params}{NWstb7-Dise-1}\nwindexuse{:redraw}{:colredraw}{NWstb7-Dise-1}\nwalsodefined{\\{NWstb7-Dise-2}}\nwused{\\{NWstb7-DisU-1}}\nwendcode{}\nwbegindocs{107}\nwdocspar
And, at last, our {\tt{}:redraw-content} method.
\nwenddocs{}\nwbegincode{108}\sublabel{NWstb7-Dise-2}\nwmargintag{{\nwtagstyle{}\subpageref{NWstb7-Dise-2}}}\moddef{Distribution prototype redrawing methods~{\nwtagstyle{}\subpageref{NWstb7-Dise-1}}}\plusendmoddef
(defmeth dist-plot-proto :redraw-content ()
  "Method args: none
Redraws the content of the plot and the background."
  (call-next-method)
  (send self :clear-lines :draw nil)
  (let* ((x (rseq (send self :xmin) (send self :xmax)
                  (send self :num-points)))
         (y (mapcar #'(lambda(w) (send self :dens-at w)) x)))
    (send self :add-lines x y :draw nil))
  (send self :adjust-to-data :draw nil)
  (send self :shade-under-plot)
  (send self :highlight-important-abscissae))
\nwindexdefn{:redraw-content}{:colredraw-content}{NWstb7-Dise-2}\eatline
\nwidentdefs{\\{{:redraw-content}{:colredraw-content}}}\nwidentuses{\\{{:dens}{:coldens}}\\{{dens}{dens}}\\{{:dens-at}{:coldens-at}}\\{{dist-plot-proto}{dist-plot-proto}}\\{{:highlight-important-abscissae}{:colhighlight-important-abscissae}}\\{{important-abscissae}{important-abscissae}}\\{{:num-points}{:colnum-points}}\\{{num-points}{num-points}}\\{{:redraw}{:colredraw}}\\{{:shade-under-plot}{:colshade-under-plot}}\\{{:xmax}{:colxmax}}\\{{:xmin}{:colxmin}}}\nwindexuse{:dens}{:coldens}{NWstb7-Dise-2}\nwindexuse{dens}{dens}{NWstb7-Dise-2}\nwindexuse{:dens-at}{:coldens-at}{NWstb7-Dise-2}\nwindexuse{dist-plot-proto}{dist-plot-proto}{NWstb7-Dise-2}\nwindexuse{:highlight-important-abscissae}{:colhighlight-important-abscissae}{NWstb7-Dise-2}\nwindexuse{important-abscissae}{important-abscissae}{NWstb7-Dise-2}\nwindexuse{:num-points}{:colnum-points}{NWstb7-Dise-2}\nwindexuse{num-points}{num-points}{NWstb7-Dise-2}\nwindexuse{:redraw}{:colredraw}{NWstb7-Dise-2}\nwindexuse{:shade-under-plot}{:colshade-under-plot}{NWstb7-Dise-2}\nwindexuse{:xmax}{:colxmax}{NWstb7-Dise-2}\nwindexuse{:xmin}{:colxmin}{NWstb7-Dise-2}\nwendcode{}\nwbegindocs{109}               

\section{The Button Overlay}
\label{sec:button-proto}
The button overlay is really quite simple. The prototype has six
slots, {\tt{}llx}, the $x$ coordinate of the left lower corner of the
button, {\tt{}ury}, the $y$ coordinate of the upper right corner of the
button, {\tt{}urx} the $x$ coordinate of the upper right corner and
{\tt{}lly} the $y$ coordinate of the lower left corner. The {\tt{}title}
slot holds the text to be displayed in the button, while {\tt{}action} is
a function that is invoked when the button is pressed with a mouse.
\nwenddocs{}\nwbegincode{110}\sublabel{NWstb7-ButZ-1}\nwmargintag{{\nwtagstyle{}\subpageref{NWstb7-ButZ-1}}}\moddef{Button overlay prototype definition~{\nwtagstyle{}\subpageref{NWstb7-ButZ-1}}}\endmoddef
(defproto button-overlay-proto '(llx ury urx lly title action) ()
  graph-overlay-proto
  "The button overlay prototype. Title is the title displayed on the
  button, and action is the function that is called when the mouse is
  clicked in the box. The slots llx and ury hold the lower left and
  upper right coordinates of the box.")
\nwindexdefn{:button-overlay-proto}{:colbutton-overlay-proto}{NWstb7-ButZ-1}\eatline
\nwidentdefs{\\{{:button-overlay-proto}{:colbutton-overlay-proto}}}\nwused{\\{NWstb7-Cod4-1}\\{NWstb7-ButJ-1}}\nwendcode{}\nwbegindocs{111}\nwdocspar

\subsection{The Overlay Methods}
\label{sec:overlay-methods}
The {\tt{}:isnew} method which basically records the given arguments in
the respective slots and invokes the inherited {\tt{}:isnew} method of
{\tt{}graph-overlay-proto}.
\nwenddocs{}\nwbegincode{112}\sublabel{NWstb7-ButW-1}\nwmargintag{{\nwtagstyle{}\subpageref{NWstb7-ButW-1}}}\moddef{Button overlay prototype methods~{\nwtagstyle{}\subpageref{NWstb7-ButW-1}}}\endmoddef
(defmeth button-overlay-proto :isnew (llx ury title action)
  "Method args: (llx ury title action)
Title is a string, action is a function that is invoked when
the mouse is clicked in the box, llx and ury are the coordinates of
the lower left and upper right x and y respectively. "
  (setf (slot-value 'action) action)
  (setf (slot-value 'title) title)
  (setf (slot-value 'llx) llx)
  (setf (slot-value 'ury) ury)
  (call-next-method))
\nwindexdefn{:isnew}{:colisnew}{NWstb7-Disa-6}\eatline
\nwidentdefs{\\{{:isnew}{:colisnew}}}\nwalsodefined{\\{NWstb7-ButW-2}\\{NWstb7-ButW-3}}\nwused{\\{NWstb7-Cod4-1}\\{NWstb7-ButJ-1}}\nwendcode{}\nwbegindocs{113}\nwdocspar

The {\tt{}:redraw} method below ensures that the button is drawn
properly.  It also calculates {\tt{}urx} and {\tt{}lly} for later use in the
{\tt{}:do-click} method. 
\nwenddocs{}\nwbegincode{114}\sublabel{NWstb7-ButW-2}\nwmargintag{{\nwtagstyle{}\subpageref{NWstb7-ButW-2}}}\moddef{Button overlay prototype methods~{\nwtagstyle{}\subpageref{NWstb7-ButW-1}}}\plusendmoddef
(defmeth button-overlay-proto :redraw ()
  "Method args: none.
This method redraws the overlay."
  (let* ((graph (send self :graph))
         (em (send graph :text-width "m"))
         (gap (round (* .5 em)))
         (title (slot-value 'title))
         (llx (slot-value 'llx))
         (ury (slot-value 'ury))
         (tw (send graph :text-width title))
         (wid (+ gap tw gap))
         (ht (+ gap (send graph :text-ascent) (send graph :text-descent)
                gap)))
    (setf (slot-value 'urx) (+ llx wid))
    (setf (slot-value 'lly) (+ ury ht))
    (send graph :draw-string title (+ llx gap)  (+ ury (- ht gap)))
    (send graph :frame-rect llx ury wid ht)))
\nwindexdefn{:redraw}{:colredraw}{NWstb7-ButW-2}\eatline
\nwidentdefs{\\{{:redraw}{:colredraw}}}\nwendcode{}\nwbegindocs{115}\nwdocspar

The final method for {\tt{}button-overlay-proto} invokes the function in
the {\tt{}action} slot when the mouse is clicked on the button.
\nwenddocs{}\nwbegincode{116}\sublabel{NWstb7-ButW-3}\nwmargintag{{\nwtagstyle{}\subpageref{NWstb7-ButW-3}}}\moddef{Button overlay prototype methods~{\nwtagstyle{}\subpageref{NWstb7-ButW-1}}}\plusendmoddef
(defmeth button-overlay-proto :do-click (x y m1 m2)
  "Method args: x y m1 m2
This method invokes the function in the action slot when the mouse is
clicked on the button."
  (let ((llx (slot-value 'llx))
        (ury (slot-value 'ury))
        (urx (slot-value 'urx))
        (lly (slot-value 'lly)))
    (when (and (< llx x urx) (< ury y lly))
          (funcall (slot-value 'action))
          t)))
\nwindexdefn{:do-click}{:coldo-click}{NWstb7-ButW-3}\eatline
\nwidentdefs{\\{{:do-click}{:coldo-click}}}\nwendcode{}\nwbegindocs{117}\nwdocspar

\section{Various Distributions}
\label{sec:variousdists}
So it is time now to add various distributions along with the
distribution specific dialogs. The following two functions
{\tt{}quantile-answer} and {\tt{}probability-answer} return standard
answers to questions.
\nwenddocs{}\nwbegincode{118}\sublabel{NWstb7-UtiH-6}\nwmargintag{{\nwtagstyle{}\subpageref{NWstb7-UtiH-6}}}\moddef{Utility functions~{\nwtagstyle{}\subpageref{NWstb7-UtiH-1}}}\plusendmoddef
(defun quantile-answer (probability quantile)
"Method args: (probability quantile)
Returns a string for a quantile answer."
  (let ((fstr (concatenate 'string
               "ANSWER: The " *probability-print-format* "-quantile is "
               *quantile-print-format* ".")))
     (format nil fstr probability quantile)))
\nwindexdefn{:quantile-answer}{:colquantile-answer}{NWstb7-UtiH-6}\eatline
\nwidentdefs{\\{{:quantile-answer}{:colquantile-answer}}}\nwidentuses{\\{{answer}{answer}}\\{{*probability-print-format*}{*probability-print-format*}}\\{{*quantile-print-format*}{*quantile-print-format*}}}\nwindexuse{answer}{answer}{NWstb7-UtiH-6}\nwindexuse{*probability-print-format*}{*probability-print-format*}{NWstb7-UtiH-6}\nwindexuse{*quantile-print-format*}{*quantile-print-format*}{NWstb7-UtiH-6}\nwendcode{}\nwbegindocs{119}\nwdocspar

\nwenddocs{}\nwbegincode{120}\sublabel{NWstb7-UtiH-7}\nwmargintag{{\nwtagstyle{}\subpageref{NWstb7-UtiH-7}}}\moddef{Utility functions~{\nwtagstyle{}\subpageref{NWstb7-UtiH-1}}}\plusendmoddef
(defun probability-answer (probability x1 x2)
"Method args: (probability x1 x2)
Returns a string for a probability answer."
  (if (and x2 (not x1))
     (let ((fstr (concatenate 'string
                "ANSWER: The probability to the right of "
                *quantile-print-format* " is "
                *probability-print-format* ".")))
                (format nil fstr x2 probability))
    (if (and x1 x2)
       (let ((fstr (concatenate 'string
                   "ANSWER: The probability between "
                   *quantile-print-format* " and "
                   *quantile-print-format* " is "
                   *probability-print-format* ".")))
             (format nil fstr x1 x2 probability))
     (let ((fstr (concatenate 'string
                   "ANSWER: The probability to the left of "
                   *quantile-print-format* " is "
                  *probability-print-format* ".")))
          (format nil fstr x1 probability)))))
\nwindexdefn{:probability-answer}{:colprobability-answer}{NWstb7-UtiH-7}\eatline
\nwidentdefs{\\{{:probability-answer}{:colprobability-answer}}}\nwidentuses{\\{{answer}{answer}}\\{{*probability-print-format*}{*probability-print-format*}}\\{{*quantile-print-format*}{*quantile-print-format*}}}\nwindexuse{answer}{answer}{NWstb7-UtiH-7}\nwindexuse{*probability-print-format*}{*probability-print-format*}{NWstb7-UtiH-7}\nwindexuse{*quantile-print-format*}{*quantile-print-format*}{NWstb7-UtiH-7}\nwendcode{}\nwbegindocs{121}\nwdocspar

Let us begin with the most famous of them all, the normal
distribution.

\subsection{The Normal Distribution}
\label{sec:normal}
Let us begin by defining the normal density, normal cdf, and inverse
of the normal cdf functions. We have to watch the names we give these
functions since \xls/ has the densities built into it for common
distributions.  However, they are standard ones. 

\nwenddocs{}\nwbegincode{122}\sublabel{NWstb7-NorJ-1}\nwmargintag{{\nwtagstyle{}\subpageref{NWstb7-NorJ-1}}}\moddef{Normal distribution~{\nwtagstyle{}\subpageref{NWstb7-NorJ-1}}}\endmoddef
(defun gaussian-density (x mu sigma)
  "Method args: (x mu sigma)
Returns the normal density with mean mu and std. dev. sigma at x."
  (/ (normal-dens (/ (- x mu) sigma)) sigma))
\nwindexdefn{gaussian-density}{gaussian-density}{NWstb7-NorJ-1}\eatline
\nwidentdefs{\\{{gaussian-density}{gaussian-density}}}\nwidentuses{\\{{dens}{dens}}}\nwindexuse{dens}{dens}{NWstb7-NorJ-1}\nwalsodefined{\\{NWstb7-NorJ-2}\\{NWstb7-NorJ-3}\\{NWstb7-NorJ-4}\\{NWstb7-NorJ-5}\\{NWstb7-NorJ-6}}\nwused{\\{NWstb7-Cod4-1}\\{NWstb7-DisI-1}}\nwendcode{}\nwbegindocs{123}\nwdocspar
\nwenddocs{}\nwbegincode{124}\sublabel{NWstb7-NorJ-2}\nwmargintag{{\nwtagstyle{}\subpageref{NWstb7-NorJ-2}}}\moddef{Normal distribution~{\nwtagstyle{}\subpageref{NWstb7-NorJ-1}}}\plusendmoddef
(defun gaussian-cdf (x mu sigma)
  "Method args: (x mu sigma)
Returns the normal cdf with mean mu and std. dev. sigma at x."
  (normal-cdf (/ (- x mu) sigma)))
\nwindexdefn{gaussian-cdf}{gaussian-cdf}{NWstb7-NorJ-2}\eatline
\nwidentdefs{\\{{gaussian-cdf}{gaussian-cdf}}}\nwidentuses{\\{{cdf}{cdf}}}\nwindexuse{cdf}{cdf}{NWstb7-NorJ-2}\nwendcode{}\nwbegindocs{125}\nwdocspar
\nwenddocs{}\nwbegincode{126}\sublabel{NWstb7-NorJ-3}\nwmargintag{{\nwtagstyle{}\subpageref{NWstb7-NorJ-3}}}\moddef{Normal distribution~{\nwtagstyle{}\subpageref{NWstb7-NorJ-1}}}\plusendmoddef
(defun gaussian-icdf (x mu sigma)
  "Method args: (x mu sigma)
Returns the x-th normal quantile with mean mu and std. dev. sigma."
  (+ (* sigma (normal-quant x)) mu))
\nwindexdefn{gaussian-cdf}{gaussian-cdf}{NWstb7-NorJ-2}\eatline
\nwidentdefs{\\{{gaussian-cdf}{gaussian-cdf}}}\nwidentuses{\\{{icdf}{icdf}}}\nwindexuse{icdf}{icdf}{NWstb7-NorJ-3}\nwendcode{}\nwbegindocs{127}\nwdocspar

Here is the dialog for finding normal-quantiles. The code is pretty
standard, except for the fact that before installing the text-items in
the dialog, we make all dialog-items have the maximum width, so that
they line up nicely in the dialog. 
\nwenddocs{}\nwbegincode{128}\sublabel{NWstb7-NorJ-4}\nwmargintag{{\nwtagstyle{}\subpageref{NWstb7-NorJ-4}}}\moddef{Normal distribution~{\nwtagstyle{}\subpageref{NWstb7-NorJ-1}}}\plusendmoddef
(defun normal-quant-dialog (dist)
  "Dialog for the quantiles of the Normal Distribution."
  (let* ((params (send dist :params))
         (prompt (send text-item-proto :new
                  (format nil 
                    "To find Normal Distribution Quantiles,~%~
                     complete all fields and press OK.~%")))
         (mean-label (send text-item-proto :new "Mean"))
         (mean-val (send edit-text-item-proto :new
                         (format nil "~a" (select params 0)) :text-length 10))
         (sd-label (send text-item-proto :new "Std. Dev."))
         (sd-val (send edit-text-item-proto :new
                       (format nil "~a" (select params 1)) :text-length 10))
         (prob-label (send text-item-proto :new "Probability"))
         (prob-val (send edit-text-item-proto :new "" :text-length 10))
         (olist (list mean-label mean-val sd-label sd-val 
                      prob-label prob-val))
         (mwid (max (mapcar #'(lambda(x) (send x :width)) olist)))
         (ok (send modal-button-proto :new "OK"
                   :action
                   #'(lambda ()
                      (let* ((inputs (get-numbers-from 
                                      (list mean-val sd-val prob-val)))
                             (params (select inputs '(0 1)))
                             (prob (select inputs 2)))
                         (send dist :params params)
                         (send dist :l-point (send dist :icdf-at prob))
                         (send dist :important-abscissae 
                               (list (send dist :l-point)))
                         (send dist :r-point nil)
                         (send dist :answer
                              (quantile-answer prob (send dist :l-point)))
                         (send dist :adjust-to-data)
                         (send dist :redraw)))))
         (cancel (send button-item-proto :new "Cancel"
                       :action
                       #'(lambda()
                           (send (send cancel :dialog)
                                 :modal-dialog-return nil)))))
    (dolist (x olist)
            (send x :width mwid))
    (send (send modal-dialog-proto
                :new (list prompt
                           (list mean-label sd-label)
                           (list mean-val sd-val)
                           (list prob-label prob-val)
                           (list ok cancel))
                :title "Normal Quantile Dialog") :modal-dialog)))
\nwindexdefn{normal-quantile-dialog}{normal-quantile-dialog}{NWstb7-NorJ-4}\eatline
\nwidentdefs{\\{{normal-quantile-dialog}{normal-quantile-dialog}}}\nwidentuses{\\{{:answer}{:colanswer}}\\{{answer}{answer}}\\{{get-numbers-from}{get-numbers-from}}\\{{:icdf}{:colicdf}}\\{{icdf}{icdf}}\\{{:icdf-at}{:colicdf-at}}\\{{:important-abscissae}{:colimportant-abscissae}}\\{{important-abscissae}{important-abscissae}}\\{{:l-point}{:coll-point}}\\{{l-point}{l-point}}\\{{:params}{:colparams}}\\{{params}{params}}\\{{:redraw}{:colredraw}}\\{{:r-point}{:colr-point}}\\{{r-point}{r-point}}\\{{:width}{:colwidth}}}\nwindexuse{:answer}{:colanswer}{NWstb7-NorJ-4}\nwindexuse{answer}{answer}{NWstb7-NorJ-4}\nwindexuse{get-numbers-from}{get-numbers-from}{NWstb7-NorJ-4}\nwindexuse{:icdf}{:colicdf}{NWstb7-NorJ-4}\nwindexuse{icdf}{icdf}{NWstb7-NorJ-4}\nwindexuse{:icdf-at}{:colicdf-at}{NWstb7-NorJ-4}\nwindexuse{:important-abscissae}{:colimportant-abscissae}{NWstb7-NorJ-4}\nwindexuse{important-abscissae}{important-abscissae}{NWstb7-NorJ-4}\nwindexuse{:l-point}{:coll-point}{NWstb7-NorJ-4}\nwindexuse{l-point}{l-point}{NWstb7-NorJ-4}\nwindexuse{:params}{:colparams}{NWstb7-NorJ-4}\nwindexuse{params}{params}{NWstb7-NorJ-4}\nwindexuse{:redraw}{:colredraw}{NWstb7-NorJ-4}\nwindexuse{:r-point}{:colr-point}{NWstb7-NorJ-4}\nwindexuse{r-point}{r-point}{NWstb7-NorJ-4}\nwindexuse{:width}{:colwidth}{NWstb7-NorJ-4}\nwendcode{}\nwbegindocs{129}\nwdocspar

Next, the dialog for finding normal probabilities. 
\nwenddocs{}\nwbegincode{130}\sublabel{NWstb7-NorJ-5}\nwmargintag{{\nwtagstyle{}\subpageref{NWstb7-NorJ-5}}}\moddef{Normal distribution~{\nwtagstyle{}\subpageref{NWstb7-NorJ-1}}}\plusendmoddef
(defun normal-prob-dialog (dist)
  "Dialog for the probabilities of the Normal Distribution."
  (let* ((params (send dist :params))
         (prompt (send text-item-proto :new
                  (format nil 
                    "To find Normal Distribution Probabilities,~%~
                     complete appropriate fields and press OK.~%~
                     For probability to the left, use Left Point;~%~
                     for probability to the right, use Right Point;~%~
                     for probability between, use both.~%")))
         (mean-label (send text-item-proto :new "Mean"))
         (mean-val (send edit-text-item-proto :new
                         (format nil "~a" (select params 0)) :text-length 10))
         (sd-label (send text-item-proto :new "Std. Dev."))
         (sd-val (send edit-text-item-proto :new
                       (format nil "~a" (select params 1)) :text-length 10))
         (l-label (send text-item-proto :new "Left Point"))
         (l-val (send edit-text-item-proto :new "" :text-length 10))
         (r-label (send text-item-proto :new "Right Point"))
         (r-val (send edit-text-item-proto :new "" :text-length 10))
         (olist (list mean-label mean-val sd-label sd-val
                      l-label l-val r-label r-val))
         (mwid (max (mapcar #'(lambda(x) (send x :width)) olist)))
         (ok (send modal-button-proto :new "OK"
                   :action
                   #'(lambda ()
                      (let* ((inputs (get-values-from 
                                      (list mean-val sd-val l-val r-val)))
                             (params (select inputs '(0 1)))
                             (lx (select inputs 2))
                             (rx (select inputs 3))
                             (between (and rx lx))
                             (left (and lx (not rx)))
                             (prob (if between
                                      (- (send dist :cdf-at rx)
                                         (send dist :cdf-at lx))
                                     (if left
                                        (send dist :cdf-at lx)
                                       (- 1 (send dist :cdf-at rx))))))
                         (send dist :params params)
                         (send dist :l-point lx)
                         (send dist :r-point rx)
                         (send dist :important-abscissae
                               (if between
                                   (list lx rx)
                                 (if left
                                     (list lx)
                                   (list rx))))
                         (send dist :answer
                              (probability-answer prob lx rx))
                         (send dist :redraw)))))
         (cancel (send button-item-proto :new "Cancel"
                       :action
                       #'(lambda()
                           (send (send cancel :dialog)
                                 :modal-dialog-return nil)))))
    (dolist (x olist)
            (send x :width mwid))
    (send (send modal-dialog-proto
                :new (list prompt
                           (list mean-label sd-label)
                           (list mean-val sd-val)
                           (list l-label l-val)
                           (list r-label r-val)
                           (list ok cancel))
                :title "Normal Probability Dialog") :modal-dialog)))
\nwindexdefn{normal-proba-dialog}{normal-proba-dialog}{NWstb7-NorJ-5}\eatline
\nwidentdefs{\\{{normal-proba-dialog}{normal-proba-dialog}}}\nwidentuses{\\{{:answer}{:colanswer}}\\{{answer}{answer}}\\{{:cdf}{:colcdf}}\\{{cdf}{cdf}}\\{{:cdf-at}{:colcdf-at}}\\{{get-values-from}{get-values-from}}\\{{:important-abscissae}{:colimportant-abscissae}}\\{{important-abscissae}{important-abscissae}}\\{{:l-point}{:coll-point}}\\{{l-point}{l-point}}\\{{:params}{:colparams}}\\{{params}{params}}\\{{:redraw}{:colredraw}}\\{{:r-point}{:colr-point}}\\{{r-point}{r-point}}\\{{:width}{:colwidth}}}\nwindexuse{:answer}{:colanswer}{NWstb7-NorJ-5}\nwindexuse{answer}{answer}{NWstb7-NorJ-5}\nwindexuse{:cdf}{:colcdf}{NWstb7-NorJ-5}\nwindexuse{cdf}{cdf}{NWstb7-NorJ-5}\nwindexuse{:cdf-at}{:colcdf-at}{NWstb7-NorJ-5}\nwindexuse{get-values-from}{get-values-from}{NWstb7-NorJ-5}\nwindexuse{:important-abscissae}{:colimportant-abscissae}{NWstb7-NorJ-5}\nwindexuse{important-abscissae}{important-abscissae}{NWstb7-NorJ-5}\nwindexuse{:l-point}{:coll-point}{NWstb7-NorJ-5}\nwindexuse{l-point}{l-point}{NWstb7-NorJ-5}\nwindexuse{:params}{:colparams}{NWstb7-NorJ-5}\nwindexuse{params}{params}{NWstb7-NorJ-5}\nwindexuse{:redraw}{:colredraw}{NWstb7-NorJ-5}\nwindexuse{:r-point}{:colr-point}{NWstb7-NorJ-5}\nwindexuse{r-point}{r-point}{NWstb7-NorJ-5}\nwindexuse{:width}{:colwidth}{NWstb7-NorJ-5}\nwendcode{}\nwbegindocs{131}\nwdocspar

We write one last function that creates an instance of
{\tt{}dist-plot-proto} for the normal distribution.

\nwenddocs{}\nwbegincode{132}\sublabel{NWstb7-NorJ-6}\nwmargintag{{\nwtagstyle{}\subpageref{NWstb7-NorJ-6}}}\moddef{Normal distribution~{\nwtagstyle{}\subpageref{NWstb7-NorJ-1}}}\plusendmoddef
(defun normal-distribution ()
  (send dist-plot-proto :new
        :dens #'gaussian-density
        :cdf #'gaussian-cdf
        :icdf #'gaussian-icdf
        :params (list 0 1)
        :params-print-format "Parameters: Mean = ~,3f, Std. Dev. = ~,3f"
        :prob-dialog #'normal-prob-dialog
        :quant-dialog #'normal-quant-dialog
        :num-points 50
        :title "Normal Distribution"))
\nwindexdefn{normal-distribution}{normal-distribution}{NWstb7-NorJ-6}\eatline
\nwidentdefs{\\{{normal-distribution}{normal-distribution}}}\nwidentuses{\\{{:cdf}{:colcdf}}\\{{cdf}{cdf}}\\{{:dens}{:coldens}}\\{{dens}{dens}}\\{{dist-plot-proto}{dist-plot-proto}}\\{{:icdf}{:colicdf}}\\{{icdf}{icdf}}\\{{:num-points}{:colnum-points}}\\{{num-points}{num-points}}\\{{:params}{:colparams}}\\{{params}{params}}\\{{:params-print-format}{:colparams-print-format}}\\{{params-print-format}{params-print-format}}}\nwindexuse{:cdf}{:colcdf}{NWstb7-NorJ-6}\nwindexuse{cdf}{cdf}{NWstb7-NorJ-6}\nwindexuse{:dens}{:coldens}{NWstb7-NorJ-6}\nwindexuse{dens}{dens}{NWstb7-NorJ-6}\nwindexuse{dist-plot-proto}{dist-plot-proto}{NWstb7-NorJ-6}\nwindexuse{:icdf}{:colicdf}{NWstb7-NorJ-6}\nwindexuse{icdf}{icdf}{NWstb7-NorJ-6}\nwindexuse{:num-points}{:colnum-points}{NWstb7-NorJ-6}\nwindexuse{num-points}{num-points}{NWstb7-NorJ-6}\nwindexuse{:params}{:colparams}{NWstb7-NorJ-6}\nwindexuse{params}{params}{NWstb7-NorJ-6}\nwindexuse{:params-print-format}{:colparams-print-format}{NWstb7-NorJ-6}\nwindexuse{params-print-format}{params-print-format}{NWstb7-NorJ-6}\nwendcode{}\nwbegindocs{133}\nwdocspar

\subsection{Student's $t$-distribution}
\label{sec:t}
This is pretty easy. We don't have to worry about the density as we
did for the normal, since we are not going to muck around with
non-central $t$ distributions for now. Here is the dialog for finding
$t$-quantiles.

\nwenddocs{}\nwbegincode{134}\sublabel{NWstb7-T*dE-1}\nwmargintag{{\nwtagstyle{}\subpageref{NWstb7-T*dE-1}}}\moddef{T distribution~{\nwtagstyle{}\subpageref{NWstb7-T*dE-1}}}\endmoddef
(defun t-quant-dialog (dist)
  "Dialog for the quantiles of the T Distribution."
  (let* ((params (send dist :params))
         (prompt (send text-item-proto :new
                  (format nil 
                    "To find T-Distribution Quantiles,~%~
                     complete all fields and press OK.~%")))
         (df-label (send text-item-proto :new "Degrees of Freedom"))
         (df-val (send edit-text-item-proto :new
                       (format nil "~a" (select params 0)) :text-length 10))
         (prob-label (send text-item-proto :new "Probability"))
         (prob-val (send edit-text-item-proto :new "" :text-length 10))
         (olist (list df-label df-val prob-label prob-val))
         (mwid (max (mapcar #'(lambda(x) (send x :width)) olist)))
         (ok (send modal-button-proto :new "OK"
                   :action
                   #'(lambda ()
                      (let* ((inputs (get-numbers-from 
                                      (list df-val prob-val)))
                             (params (select inputs '(0)))
                             (prob (select inputs 1)))
                         (send dist :params params)
                         (send dist :l-point (send dist :icdf-at prob))
                         (send dist :important-abscissae 
                               (list (send dist :l-point)))
                         (send dist :r-point nil)
                         (send dist :answer
                              (quantile-answer prob (send dist :l-point)))
                         (send dist :adjust-to-data)
                         (send dist :redraw)))))
         (cancel (send button-item-proto :new "Cancel"
                       :action
                       #'(lambda()
                           (send (send cancel :dialog)
                                 :modal-dialog-return nil)))))
    (dolist (x olist)
            (send x :width mwid))
    (send (send modal-dialog-proto
                :new (list prompt
                           (list df-label prob-label)
                           (list df-val prob-val)
                           (list ok cancel))
                :title "T Quantile Dialog") :modal-dialog)))
\nwindexdefn{t-quant-dialog}{t-quant-dialog}{NWstb7-T*dE-1}\eatline
\nwidentdefs{\\{{t-quant-dialog}{t-quant-dialog}}}\nwidentuses{\\{{:answer}{:colanswer}}\\{{answer}{answer}}\\{{get-numbers-from}{get-numbers-from}}\\{{:icdf}{:colicdf}}\\{{icdf}{icdf}}\\{{:icdf-at}{:colicdf-at}}\\{{:important-abscissae}{:colimportant-abscissae}}\\{{important-abscissae}{important-abscissae}}\\{{:l-point}{:coll-point}}\\{{l-point}{l-point}}\\{{:params}{:colparams}}\\{{params}{params}}\\{{:redraw}{:colredraw}}\\{{:r-point}{:colr-point}}\\{{r-point}{r-point}}\\{{:width}{:colwidth}}}\nwindexuse{:answer}{:colanswer}{NWstb7-T*dE-1}\nwindexuse{answer}{answer}{NWstb7-T*dE-1}\nwindexuse{get-numbers-from}{get-numbers-from}{NWstb7-T*dE-1}\nwindexuse{:icdf}{:colicdf}{NWstb7-T*dE-1}\nwindexuse{icdf}{icdf}{NWstb7-T*dE-1}\nwindexuse{:icdf-at}{:colicdf-at}{NWstb7-T*dE-1}\nwindexuse{:important-abscissae}{:colimportant-abscissae}{NWstb7-T*dE-1}\nwindexuse{important-abscissae}{important-abscissae}{NWstb7-T*dE-1}\nwindexuse{:l-point}{:coll-point}{NWstb7-T*dE-1}\nwindexuse{l-point}{l-point}{NWstb7-T*dE-1}\nwindexuse{:params}{:colparams}{NWstb7-T*dE-1}\nwindexuse{params}{params}{NWstb7-T*dE-1}\nwindexuse{:redraw}{:colredraw}{NWstb7-T*dE-1}\nwindexuse{:r-point}{:colr-point}{NWstb7-T*dE-1}\nwindexuse{r-point}{r-point}{NWstb7-T*dE-1}\nwindexuse{:width}{:colwidth}{NWstb7-T*dE-1}\nwalsodefined{\\{NWstb7-T*dE-2}\\{NWstb7-T*dE-3}}\nwused{\\{NWstb7-Cod4-1}\\{NWstb7-DisI-1}}\nwendcode{}\nwbegindocs{135}\nwdocspar

Here is the dialog for finding $t$-probabilities. 

\nwenddocs{}\nwbegincode{136}\sublabel{NWstb7-T*dE-2}\nwmargintag{{\nwtagstyle{}\subpageref{NWstb7-T*dE-2}}}\moddef{T distribution~{\nwtagstyle{}\subpageref{NWstb7-T*dE-1}}}\plusendmoddef
(defun t-prob-dialog (dist)
  "Dialog for the probabilities of the T Distribution."
  (let* ((params (send dist :params))
         (prompt (send text-item-proto :new
                  (format nil 
                    "To find T-Distribution Probabilities,~%~
                     complete appropriate fields and press OK.~%~
                     For probability to the left, use Left Point;~%~
                     for probability to the right, use Right Point;~%~
                     for probability between, use both.~%")))
         (df-label (send text-item-proto :new "Degrees of Freedom"))
         (df-val (send edit-text-item-proto :new
                       (format nil "~a" (select params 0)) :text-length 10))
         (l-label (send text-item-proto :new "Left Point"))
         (l-val (send edit-text-item-proto :new "" :text-length 10))
         (r-label (send text-item-proto :new "Right Point"))
         (r-val (send edit-text-item-proto :new "" :text-length 10))
         (olist (list df-label df-val l-label l-val r-label r-val))
         (mwid (max (mapcar #'(lambda(x) (send x :width)) olist)))
         (ok (send modal-button-proto :new "OK"
                   :action
                   #'(lambda ()
                      (let* ((inputs (get-values-from 
                                      (list df-val l-val r-val)))
                             (params (select inputs '(0)))
                             (lx (select inputs 1))
                             (rx (select inputs 2))
                             (between (and rx lx))
                             (left (and lx (not rx)))
                             (prob (if between
                                      (- (send dist :cdf-at rx)
                                         (send dist :cdf-at lx))
                                     (if left
                                        (send dist :cdf-at lx)
                                       (- 1 (send dist :cdf-at rx))))))
                         (send dist :params params)
                         (send dist :l-point lx)
                         (send dist :r-point rx)
                         (send dist :important-abscissae
                               (if between
                                   (list lx rx)
                                 (if left
                                     (list lx)
                                   (list rx))))
                         (send dist :answer
                              (probability-answer prob lx rx))
                         (send dist :redraw)))))
         (cancel (send button-item-proto :new "Cancel"
                       :action
                       #'(lambda()
                           (send (send cancel :dialog)
                                 :modal-dialog-return nil)))))
    (dolist (x olist)
            (send x :width mwid))
    (send (send modal-dialog-proto
                :new (list prompt
                           (list df-label df-val)
                           (list l-label l-val)
                           (list r-label r-val)
                           (list ok cancel))
                :title "T Probability Dialog") :modal-dialog)))
\nwindexdefn{t-prob-dialog}{t-prob-dialog}{NWstb7-T*dE-2}\eatline
\nwidentdefs{\\{{t-prob-dialog}{t-prob-dialog}}}\nwidentuses{\\{{:answer}{:colanswer}}\\{{answer}{answer}}\\{{:cdf}{:colcdf}}\\{{cdf}{cdf}}\\{{:cdf-at}{:colcdf-at}}\\{{get-values-from}{get-values-from}}\\{{:important-abscissae}{:colimportant-abscissae}}\\{{important-abscissae}{important-abscissae}}\\{{:l-point}{:coll-point}}\\{{l-point}{l-point}}\\{{:params}{:colparams}}\\{{params}{params}}\\{{:redraw}{:colredraw}}\\{{:r-point}{:colr-point}}\\{{r-point}{r-point}}\\{{:width}{:colwidth}}}\nwindexuse{:answer}{:colanswer}{NWstb7-T*dE-2}\nwindexuse{answer}{answer}{NWstb7-T*dE-2}\nwindexuse{:cdf}{:colcdf}{NWstb7-T*dE-2}\nwindexuse{cdf}{cdf}{NWstb7-T*dE-2}\nwindexuse{:cdf-at}{:colcdf-at}{NWstb7-T*dE-2}\nwindexuse{get-values-from}{get-values-from}{NWstb7-T*dE-2}\nwindexuse{:important-abscissae}{:colimportant-abscissae}{NWstb7-T*dE-2}\nwindexuse{important-abscissae}{important-abscissae}{NWstb7-T*dE-2}\nwindexuse{:l-point}{:coll-point}{NWstb7-T*dE-2}\nwindexuse{l-point}{l-point}{NWstb7-T*dE-2}\nwindexuse{:params}{:colparams}{NWstb7-T*dE-2}\nwindexuse{params}{params}{NWstb7-T*dE-2}\nwindexuse{:redraw}{:colredraw}{NWstb7-T*dE-2}\nwindexuse{:r-point}{:colr-point}{NWstb7-T*dE-2}\nwindexuse{r-point}{r-point}{NWstb7-T*dE-2}\nwindexuse{:width}{:colwidth}{NWstb7-T*dE-2}\nwendcode{}\nwbegindocs{137}\nwdocspar

And a function that creates an instance of {\tt{}dist-plot-proto} for the
$t$-distribution.

\nwenddocs{}\nwbegincode{138}\sublabel{NWstb7-T*dE-3}\nwmargintag{{\nwtagstyle{}\subpageref{NWstb7-T*dE-3}}}\moddef{T distribution~{\nwtagstyle{}\subpageref{NWstb7-T*dE-1}}}\plusendmoddef
(defun t-distribution ()
  (send dist-plot-proto :new
        :dens #'t-dens
        :cdf #'t-cdf
        :icdf #'t-quant
        :params (list 15)
        :params-print-format "Parameter: Degrees of Freedom = ~,1f"
        :prob-dialog #'t-prob-dialog
        :quant-dialog #'t-quant-dialog
        :num-points 50
        :title "T-Distribution"))
\nwindexdefn{t-distribution}{t-distribution}{NWstb7-T*dE-3}\eatline
\nwidentdefs{\\{{t-distribution}{t-distribution}}}\nwidentuses{\\{{:cdf}{:colcdf}}\\{{cdf}{cdf}}\\{{:dens}{:coldens}}\\{{dens}{dens}}\\{{dist-plot-proto}{dist-plot-proto}}\\{{:icdf}{:colicdf}}\\{{icdf}{icdf}}\\{{:num-points}{:colnum-points}}\\{{num-points}{num-points}}\\{{:params}{:colparams}}\\{{params}{params}}\\{{:params-print-format}{:colparams-print-format}}\\{{params-print-format}{params-print-format}}}\nwindexuse{:cdf}{:colcdf}{NWstb7-T*dE-3}\nwindexuse{cdf}{cdf}{NWstb7-T*dE-3}\nwindexuse{:dens}{:coldens}{NWstb7-T*dE-3}\nwindexuse{dens}{dens}{NWstb7-T*dE-3}\nwindexuse{dist-plot-proto}{dist-plot-proto}{NWstb7-T*dE-3}\nwindexuse{:icdf}{:colicdf}{NWstb7-T*dE-3}\nwindexuse{icdf}{icdf}{NWstb7-T*dE-3}\nwindexuse{:num-points}{:colnum-points}{NWstb7-T*dE-3}\nwindexuse{num-points}{num-points}{NWstb7-T*dE-3}\nwindexuse{:params}{:colparams}{NWstb7-T*dE-3}\nwindexuse{params}{params}{NWstb7-T*dE-3}\nwindexuse{:params-print-format}{:colparams-print-format}{NWstb7-T*dE-3}\nwindexuse{params-print-format}{params-print-format}{NWstb7-T*dE-3}\nwendcode{}\nwbegindocs{139}\nwdocspar

\subsection{Chi-square distribution}
\label{sec:chisq}
This is exactly like the $t$-distribution.

\nwenddocs{}\nwbegincode{140}\sublabel{NWstb7-ChiN-1}\nwmargintag{{\nwtagstyle{}\subpageref{NWstb7-ChiN-1}}}\moddef{Chi-square distribution~{\nwtagstyle{}\subpageref{NWstb7-ChiN-1}}}\endmoddef
(defun chisq-quant-dialog (dist)
  "Dialog for the quantiles of the Chi-square Distribution."
  (let* ((params (send dist :params))
         (prompt (send text-item-proto :new
                  (format nil 
                    "To find Chi-square Distribution Quantiles,~%~
                     complete all fields and press OK.~%")))
         (df-label (send text-item-proto :new "Degrees of Freedom"))
         (df-val (send edit-text-item-proto :new
                       (format nil "~a" (select params 0)) :text-length 10))
         (prob-label (send text-item-proto :new "Probability"))
         (prob-val (send edit-text-item-proto :new "" :text-length 10))
         (olist (list df-label df-val prob-label prob-val))
         (mwid (max (mapcar #'(lambda(x) (send x :width)) olist)))
         (ok (send modal-button-proto :new "OK"
                   :action
                   #'(lambda ()
                      (let* ((inputs (get-numbers-from 
                                      (list df-val prob-val)))
                             (params (select inputs '(0)))
                             (prob (select inputs 1)))
                         (send dist :params params)
                         (send dist :l-point (send dist :icdf-at prob))
                         (send dist :important-abscissae 
                               (list (send dist :l-point)))
                         (send dist :r-point nil)
                         (send dist :answer
                              (quantile-answer prob (send dist :l-point)))
                         (send dist :adjust-to-data)
                         (send dist :redraw)))))
         (cancel (send button-item-proto :new "Cancel"
                       :action
                       #'(lambda()
                           (send (send cancel :dialog)
                                 :modal-dialog-return nil)))))
    (dolist (x olist)
            (send x :width mwid))
    (send (send modal-dialog-proto
                :new (list prompt
                           (list df-label prob-label)
                           (list df-val prob-val)
                           (list ok cancel))
                :title "Chi-square Quantile Dialog") :modal-dialog)))
\nwindexdefn{chisq-quant-dialog}{chisq-quant-dialog}{NWstb7-ChiN-1}\eatline
\nwidentdefs{\\{{chisq-quant-dialog}{chisq-quant-dialog}}}\nwidentuses{\\{{:answer}{:colanswer}}\\{{answer}{answer}}\\{{get-numbers-from}{get-numbers-from}}\\{{:icdf}{:colicdf}}\\{{icdf}{icdf}}\\{{:icdf-at}{:colicdf-at}}\\{{:important-abscissae}{:colimportant-abscissae}}\\{{important-abscissae}{important-abscissae}}\\{{:l-point}{:coll-point}}\\{{l-point}{l-point}}\\{{:params}{:colparams}}\\{{params}{params}}\\{{:redraw}{:colredraw}}\\{{:r-point}{:colr-point}}\\{{r-point}{r-point}}\\{{:width}{:colwidth}}}\nwindexuse{:answer}{:colanswer}{NWstb7-ChiN-1}\nwindexuse{answer}{answer}{NWstb7-ChiN-1}\nwindexuse{get-numbers-from}{get-numbers-from}{NWstb7-ChiN-1}\nwindexuse{:icdf}{:colicdf}{NWstb7-ChiN-1}\nwindexuse{icdf}{icdf}{NWstb7-ChiN-1}\nwindexuse{:icdf-at}{:colicdf-at}{NWstb7-ChiN-1}\nwindexuse{:important-abscissae}{:colimportant-abscissae}{NWstb7-ChiN-1}\nwindexuse{important-abscissae}{important-abscissae}{NWstb7-ChiN-1}\nwindexuse{:l-point}{:coll-point}{NWstb7-ChiN-1}\nwindexuse{l-point}{l-point}{NWstb7-ChiN-1}\nwindexuse{:params}{:colparams}{NWstb7-ChiN-1}\nwindexuse{params}{params}{NWstb7-ChiN-1}\nwindexuse{:redraw}{:colredraw}{NWstb7-ChiN-1}\nwindexuse{:r-point}{:colr-point}{NWstb7-ChiN-1}\nwindexuse{r-point}{r-point}{NWstb7-ChiN-1}\nwindexuse{:width}{:colwidth}{NWstb7-ChiN-1}\nwalsodefined{\\{NWstb7-ChiN-2}\\{NWstb7-ChiN-3}}\nwused{\\{NWstb7-Cod4-1}\\{NWstb7-DisI-1}}\nwendcode{}\nwbegindocs{141}\nwdocspar

Then the dialog for finding Chi-square probabilities. 

\nwenddocs{}\nwbegincode{142}\sublabel{NWstb7-ChiN-2}\nwmargintag{{\nwtagstyle{}\subpageref{NWstb7-ChiN-2}}}\moddef{Chi-square distribution~{\nwtagstyle{}\subpageref{NWstb7-ChiN-1}}}\plusendmoddef
(defun chisq-prob-dialog (dist)
  "Dialog for the probabilities of the Chi-square Distribution."
  (let* ((params (send dist :params))
         (prompt (send text-item-proto :new
                  (format nil 
                    "To find Chisq-Distribution Probabilities,~%~
                     complete appropriate fields and press OK.~%~
                     For probability to the left, use Left Point;~%~
                     for probability to the right, use Right Point;~%~
                     for probability between, use both.~%")))
         (df-label (send text-item-proto :new "Degrees of Freedom"))
         (df-val (send edit-text-item-proto :new
                       (format nil "~a" (select params 0)) :text-length 10))
         (l-label (send text-item-proto :new "Left Point"))
         (l-val (send edit-text-item-proto :new "" :text-length 10))
         (r-label (send text-item-proto :new "Right Point"))
         (r-val (send edit-text-item-proto :new "" :text-length 10))
         (olist (list df-label df-val l-label l-val r-label r-val))
         (mwid (max (mapcar #'(lambda(x) (send x :width)) olist)))
         (ok (send modal-button-proto :new "OK"
                   :action
                   #'(lambda ()
                      (let* ((inputs (get-values-from 
                                      (list df-val l-val r-val)))
                             (params (select inputs '(0)))
                             (lx (select inputs 1))
                             (rx (select inputs 2))
                             (between (and rx lx))
                             (left (and lx (not rx)))
                             (prob (if between
                                      (- (send dist :cdf-at rx)
                                         (send dist :cdf-at lx))
                                     (if left
                                        (send dist :cdf-at lx)
                                       (- 1 (send dist :cdf-at rx))))))
                         (send dist :params params)
                         (send dist :l-point lx)
                         (send dist :r-point rx)
                         (send dist :important-abscissae
                               (if between
                                   (list lx rx)
                                 (if left
                                     (list lx)
                                   (list rx))))
                         (send dist :answer
                              (probability-answer prob lx rx))
                         (send dist :redraw)))))
         (cancel (send button-item-proto :new "Cancel"
                       :action
                       #'(lambda()
                           (send (send cancel :dialog)
                                 :modal-dialog-return nil)))))
    (dolist (x olist)
            (send x :width mwid))
    (send (send modal-dialog-proto
                :new (list prompt
                           (list df-label df-val)
                           (list l-label l-val)
                           (list r-label r-val)
                           (list ok cancel))
                :title "Chi-square Probability Dialog") :modal-dialog)))
\nwindexdefn{chisq-prob-dialog}{chisq-prob-dialog}{NWstb7-ChiN-2}\eatline
\nwidentdefs{\\{{chisq-prob-dialog}{chisq-prob-dialog}}}\nwidentuses{\\{{:answer}{:colanswer}}\\{{answer}{answer}}\\{{:cdf}{:colcdf}}\\{{cdf}{cdf}}\\{{:cdf-at}{:colcdf-at}}\\{{get-values-from}{get-values-from}}\\{{:important-abscissae}{:colimportant-abscissae}}\\{{important-abscissae}{important-abscissae}}\\{{:l-point}{:coll-point}}\\{{l-point}{l-point}}\\{{:params}{:colparams}}\\{{params}{params}}\\{{:redraw}{:colredraw}}\\{{:r-point}{:colr-point}}\\{{r-point}{r-point}}\\{{:width}{:colwidth}}}\nwindexuse{:answer}{:colanswer}{NWstb7-ChiN-2}\nwindexuse{answer}{answer}{NWstb7-ChiN-2}\nwindexuse{:cdf}{:colcdf}{NWstb7-ChiN-2}\nwindexuse{cdf}{cdf}{NWstb7-ChiN-2}\nwindexuse{:cdf-at}{:colcdf-at}{NWstb7-ChiN-2}\nwindexuse{get-values-from}{get-values-from}{NWstb7-ChiN-2}\nwindexuse{:important-abscissae}{:colimportant-abscissae}{NWstb7-ChiN-2}\nwindexuse{important-abscissae}{important-abscissae}{NWstb7-ChiN-2}\nwindexuse{:l-point}{:coll-point}{NWstb7-ChiN-2}\nwindexuse{l-point}{l-point}{NWstb7-ChiN-2}\nwindexuse{:params}{:colparams}{NWstb7-ChiN-2}\nwindexuse{params}{params}{NWstb7-ChiN-2}\nwindexuse{:redraw}{:colredraw}{NWstb7-ChiN-2}\nwindexuse{:r-point}{:colr-point}{NWstb7-ChiN-2}\nwindexuse{r-point}{r-point}{NWstb7-ChiN-2}\nwindexuse{:width}{:colwidth}{NWstb7-ChiN-2}\nwendcode{}\nwbegindocs{143}\nwdocspar
And, as usual, 

\nwenddocs{}\nwbegincode{144}\sublabel{NWstb7-ChiN-3}\nwmargintag{{\nwtagstyle{}\subpageref{NWstb7-ChiN-3}}}\moddef{Chi-square distribution~{\nwtagstyle{}\subpageref{NWstb7-ChiN-1}}}\plusendmoddef
(defun chisq-distribution ()
  (send dist-plot-proto :new
        :dens #'chisq-dens
        :cdf #'chisq-cdf
        :icdf #'chisq-quant
        :params (list 15)
        :params-print-format "Parameter: Degrees of Freedom = ~,1f"
        :prob-dialog #'chisq-prob-dialog
        :quant-dialog #'chisq-quant-dialog
        :num-points 50
        :title "Chi-Square Distribution"))
\nwindexdefn{chisq-distribution}{chisq-distribution}{NWstb7-ChiN-3}\eatline
\nwidentdefs{\\{{chisq-distribution}{chisq-distribution}}}\nwidentuses{\\{{:cdf}{:colcdf}}\\{{cdf}{cdf}}\\{{:dens}{:coldens}}\\{{dens}{dens}}\\{{dist-plot-proto}{dist-plot-proto}}\\{{:icdf}{:colicdf}}\\{{icdf}{icdf}}\\{{:num-points}{:colnum-points}}\\{{num-points}{num-points}}\\{{:params}{:colparams}}\\{{params}{params}}\\{{:params-print-format}{:colparams-print-format}}\\{{params-print-format}{params-print-format}}}\nwindexuse{:cdf}{:colcdf}{NWstb7-ChiN-3}\nwindexuse{cdf}{cdf}{NWstb7-ChiN-3}\nwindexuse{:dens}{:coldens}{NWstb7-ChiN-3}\nwindexuse{dens}{dens}{NWstb7-ChiN-3}\nwindexuse{dist-plot-proto}{dist-plot-proto}{NWstb7-ChiN-3}\nwindexuse{:icdf}{:colicdf}{NWstb7-ChiN-3}\nwindexuse{icdf}{icdf}{NWstb7-ChiN-3}\nwindexuse{:num-points}{:colnum-points}{NWstb7-ChiN-3}\nwindexuse{num-points}{num-points}{NWstb7-ChiN-3}\nwindexuse{:params}{:colparams}{NWstb7-ChiN-3}\nwindexuse{params}{params}{NWstb7-ChiN-3}\nwindexuse{:params-print-format}{:colparams-print-format}{NWstb7-ChiN-3}\nwindexuse{params-print-format}{params-print-format}{NWstb7-ChiN-3}\nwendcode{}\nwbegindocs{145}\nwdocspar

\subsection{The $F$ Distribution}
\label{sec:f}
We wont bother describing the code for the $F$-distribution.

\nwenddocs{}\nwbegincode{146}\sublabel{NWstb7-F*dE-1}\nwmargintag{{\nwtagstyle{}\subpageref{NWstb7-F*dE-1}}}\moddef{F distribution~{\nwtagstyle{}\subpageref{NWstb7-F*dE-1}}}\endmoddef
(defun f-quant-dialog (dist)
  "Dialog for the quantiles of the F Distribution."
  (let* ((params (send dist :params))
         (prompt (send text-item-proto :new
                  (format nil 
                    "To find F Distribution Quantiles,~%~
                     complete all fields and press OK.~%")))
         (ndf-label (send text-item-proto :new "Numerator df"))
         (ndf-val (send edit-text-item-proto :new
                         (format nil "~a" (select params 0)) :text-length 10))
         (ddf-label (send text-item-proto :new "Denominator df"))
         (ddf-val (send edit-text-item-proto :new
                       (format nil "~a" (select params 1)) :text-length 10))
         (prob-label (send text-item-proto :new "Probability"))
         (prob-val (send edit-text-item-proto :new "" :text-length 10))
         (olist (list ndf-label ndf-val ddf-label ddf-val
                      prob-label prob-val))
         (mwid (max (mapcar #'(lambda(x) (send x :width)) olist)))
         (ok (send modal-button-proto :new "OK"
                   :action
                   #'(lambda ()
                      (let* ((inputs (get-numbers-from 
                                      (list ndf-val ddf-val prob-val)))
                             (params (select inputs '(0 1)))
                             (prob (select inputs 2)))
                         (send dist :params params)
                         (send dist :l-point (send dist :icdf-at prob))
                         (send dist :important-abscissae 
                               (list (send dist :l-point)))
                         (send dist :r-point nil)
                         (send dist :answer
                              (quantile-answer prob (send dist :l-point)))
                         (send dist :adjust-to-data)
                         (send dist :redraw)))))
         (cancel (send button-item-proto :new "Cancel"
                       :action
                       #'(lambda()
                           (send (send cancel :dialog)
                                 :modal-dialog-return nil)))))
    (dolist (x olist)
            (send x :width mwid))
    (send (send modal-dialog-proto
                :new (list prompt
                           (list ndf-label ddf-label)
                           (list ndf-val ddf-val)
                           (list prob-label prob-val)
                           (list ok cancel))
                :title "F Quantile Dialog") :modal-dialog)))
\nwindexdefn{f-quant-dialog}{f-quant-dialog}{NWstb7-F*dE-1}\eatline
\nwidentdefs{\\{{f-quant-dialog}{f-quant-dialog}}}\nwidentuses{\\{{:answer}{:colanswer}}\\{{answer}{answer}}\\{{get-numbers-from}{get-numbers-from}}\\{{:icdf}{:colicdf}}\\{{icdf}{icdf}}\\{{:icdf-at}{:colicdf-at}}\\{{:important-abscissae}{:colimportant-abscissae}}\\{{important-abscissae}{important-abscissae}}\\{{:l-point}{:coll-point}}\\{{l-point}{l-point}}\\{{:params}{:colparams}}\\{{params}{params}}\\{{:redraw}{:colredraw}}\\{{:r-point}{:colr-point}}\\{{r-point}{r-point}}\\{{:width}{:colwidth}}}\nwindexuse{:answer}{:colanswer}{NWstb7-F*dE-1}\nwindexuse{answer}{answer}{NWstb7-F*dE-1}\nwindexuse{get-numbers-from}{get-numbers-from}{NWstb7-F*dE-1}\nwindexuse{:icdf}{:colicdf}{NWstb7-F*dE-1}\nwindexuse{icdf}{icdf}{NWstb7-F*dE-1}\nwindexuse{:icdf-at}{:colicdf-at}{NWstb7-F*dE-1}\nwindexuse{:important-abscissae}{:colimportant-abscissae}{NWstb7-F*dE-1}\nwindexuse{important-abscissae}{important-abscissae}{NWstb7-F*dE-1}\nwindexuse{:l-point}{:coll-point}{NWstb7-F*dE-1}\nwindexuse{l-point}{l-point}{NWstb7-F*dE-1}\nwindexuse{:params}{:colparams}{NWstb7-F*dE-1}\nwindexuse{params}{params}{NWstb7-F*dE-1}\nwindexuse{:redraw}{:colredraw}{NWstb7-F*dE-1}\nwindexuse{:r-point}{:colr-point}{NWstb7-F*dE-1}\nwindexuse{r-point}{r-point}{NWstb7-F*dE-1}\nwindexuse{:width}{:colwidth}{NWstb7-F*dE-1}\nwalsodefined{\\{NWstb7-F*dE-2}\\{NWstb7-F*dE-3}}\nwused{\\{NWstb7-Cod4-1}\\{NWstb7-DisI-1}}\nwendcode{}\nwbegindocs{147}\nwdocspar

\nwenddocs{}\nwbegincode{148}\sublabel{NWstb7-F*dE-2}\nwmargintag{{\nwtagstyle{}\subpageref{NWstb7-F*dE-2}}}\moddef{F distribution~{\nwtagstyle{}\subpageref{NWstb7-F*dE-1}}}\plusendmoddef
(defun f-prob-dialog (dist)
  "Dialog for the probabilities of the F Distribution."
  (let* ((params (send dist :params))
         (prompt (send text-item-proto :new
                  (format nil 
                    "To find F istribution Probabilities,~%~
                     complete appropriate fields and press OK.~%~
                     For probability to the left, use Left Point;~%~
                     for probability to the right, use Right Point;~%~
                     for probability between, use both.~%")))
         (ndf-label (send text-item-proto :new "Numerator df"))
         (ndf-val (send edit-text-item-proto :new
                         (format nil "~a" (select params 0)) :text-length 10))
         (ddf-label (send text-item-proto :new "Denominator df"))
         (ddf-val (send edit-text-item-proto :new
                       (format nil "~a" (select params 1)) :text-length 10))
         (l-label (send text-item-proto :new "Left Point"))
         (l-val (send edit-text-item-proto :new "" :text-length 10))
         (r-label (send text-item-proto :new "Right Point"))
         (r-val (send edit-text-item-proto :new "" :text-length 10))
         (olist (list mean-label mean-val sd-label sd-val
                      l-label l-val r-label r-val))
         (mwid (max (mapcar #'(lambda(x) (send x :width)) olist)))
         (ok (send modal-button-proto :new "OK"
                   :action
                   #'(lambda ()
                      (let* ((inputs (get-values-from 
                                      (list ndf-val ddf-val l-val r-val)))
                             (params (select inputs '(0 1)))
                             (lx (select inputs 2))
                             (rx (select inputs 3))
                             (between (and rx lx))
                             (left (and lx (not rx)))
                             (prob (if between
                                      (- (send dist :cdf-at rx)
                                         (send dist :cdf-at lx))
                                     (if left
                                        (send dist :cdf-at lx)
                                       (- 1 (send dist :cdf-at rx))))))
                         (send dist :params params)
                         (send dist :l-point lx)
                         (send dist :r-point rx)
                         (send dist :important-abscissae
                               (if between
                                   (list lx rx)
                                 (if left
                                     (list lx)
                                   (list rx))))
                         (send dist :answer
                              (probability-answer prob lx rx))
                         (send dist :redraw)))))
         (cancel (send button-item-proto :new "Cancel"
                       :action
                       #'(lambda()
                           (send (send cancel :dialog)
                                 :modal-dialog-return nil)))))
    (dolist (x olist)
            (send x :width mwid))
    (send (send modal-dialog-proto
                :new (list prompt
                           (list ndf-label ddf-label)
                           (list ndf-val ddf-val)
                           (list l-label l-val)
                           (list r-label r-val)
                           (list ok cancel))
                :title "F Probability Dialog") :modal-dialog)))
\nwindexdefn{f-prob-dialog}{f-prob-dialog}{NWstb7-F*dE-2}\eatline
\nwidentdefs{\\{{f-prob-dialog}{f-prob-dialog}}}\nwidentuses{\\{{:answer}{:colanswer}}\\{{answer}{answer}}\\{{:cdf}{:colcdf}}\\{{cdf}{cdf}}\\{{:cdf-at}{:colcdf-at}}\\{{get-values-from}{get-values-from}}\\{{:important-abscissae}{:colimportant-abscissae}}\\{{important-abscissae}{important-abscissae}}\\{{:l-point}{:coll-point}}\\{{l-point}{l-point}}\\{{:params}{:colparams}}\\{{params}{params}}\\{{:redraw}{:colredraw}}\\{{:r-point}{:colr-point}}\\{{r-point}{r-point}}\\{{:width}{:colwidth}}}\nwindexuse{:answer}{:colanswer}{NWstb7-F*dE-2}\nwindexuse{answer}{answer}{NWstb7-F*dE-2}\nwindexuse{:cdf}{:colcdf}{NWstb7-F*dE-2}\nwindexuse{cdf}{cdf}{NWstb7-F*dE-2}\nwindexuse{:cdf-at}{:colcdf-at}{NWstb7-F*dE-2}\nwindexuse{get-values-from}{get-values-from}{NWstb7-F*dE-2}\nwindexuse{:important-abscissae}{:colimportant-abscissae}{NWstb7-F*dE-2}\nwindexuse{important-abscissae}{important-abscissae}{NWstb7-F*dE-2}\nwindexuse{:l-point}{:coll-point}{NWstb7-F*dE-2}\nwindexuse{l-point}{l-point}{NWstb7-F*dE-2}\nwindexuse{:params}{:colparams}{NWstb7-F*dE-2}\nwindexuse{params}{params}{NWstb7-F*dE-2}\nwindexuse{:redraw}{:colredraw}{NWstb7-F*dE-2}\nwindexuse{:r-point}{:colr-point}{NWstb7-F*dE-2}\nwindexuse{r-point}{r-point}{NWstb7-F*dE-2}\nwindexuse{:width}{:colwidth}{NWstb7-F*dE-2}\nwendcode{}\nwbegindocs{149}\nwdocspar

\nwenddocs{}\nwbegincode{150}\sublabel{NWstb7-F*dE-3}\nwmargintag{{\nwtagstyle{}\subpageref{NWstb7-F*dE-3}}}\moddef{F distribution~{\nwtagstyle{}\subpageref{NWstb7-F*dE-1}}}\plusendmoddef
(defun f-distribution ()
  (send dist-plot-proto :new
        :dens #'f-dens
        :cdf #'f-cdf
        :icdf #'f-quant
        :params (list 20 20)
        :params-print-format "Parameters: Num. df. = ~,1f, Den. df. = ~,1f"
        :prob-dialog #'f-prob-dialog
        :quant-dialog #'f-quant-dialog
        :num-points 50
        :title "FDistribution"))
\nwindexdefn{f-distribution}{f-distribution}{NWstb7-F*dE-3}\eatline
\nwidentdefs{\\{{f-distribution}{f-distribution}}}\nwidentuses{\\{{:cdf}{:colcdf}}\\{{cdf}{cdf}}\\{{:dens}{:coldens}}\\{{dens}{dens}}\\{{dist-plot-proto}{dist-plot-proto}}\\{{:icdf}{:colicdf}}\\{{icdf}{icdf}}\\{{:num-points}{:colnum-points}}\\{{num-points}{num-points}}\\{{:params}{:colparams}}\\{{params}{params}}\\{{:params-print-format}{:colparams-print-format}}\\{{params-print-format}{params-print-format}}}\nwindexuse{:cdf}{:colcdf}{NWstb7-F*dE-3}\nwindexuse{cdf}{cdf}{NWstb7-F*dE-3}\nwindexuse{:dens}{:coldens}{NWstb7-F*dE-3}\nwindexuse{dens}{dens}{NWstb7-F*dE-3}\nwindexuse{dist-plot-proto}{dist-plot-proto}{NWstb7-F*dE-3}\nwindexuse{:icdf}{:colicdf}{NWstb7-F*dE-3}\nwindexuse{icdf}{icdf}{NWstb7-F*dE-3}\nwindexuse{:num-points}{:colnum-points}{NWstb7-F*dE-3}\nwindexuse{num-points}{num-points}{NWstb7-F*dE-3}\nwindexuse{:params}{:colparams}{NWstb7-F*dE-3}\nwindexuse{params}{params}{NWstb7-F*dE-3}\nwindexuse{:params-print-format}{:colparams-print-format}{NWstb7-F*dE-3}\nwindexuse{params-print-format}{params-print-format}{NWstb7-F*dE-3}\nwendcode{}\nwbegindocs{151}\nwdocspar

\section{The Main Function}
\label{sec:main-func}
The main function {\tt{}stbl} installs the statistical tables in a menu.
\nwenddocs{}\nwbegincode{152}\sublabel{NWstb7-MaiD-1}\nwmargintag{{\nwtagstyle{}\subpageref{NWstb7-MaiD-1}}}\moddef{Main function~{\nwtagstyle{}\subpageref{NWstb7-MaiD-1}}}\endmoddef
(defun stbl ()
"Method args: none
 Installs statistical tables in a menu."
  (let ((menu (send menu-proto :new "Tables")))
    (send menu :append-items
          (send menu-item-proto :new "Normal Distribution"
                :action #'normal-distribution)
          (send menu-item-proto :new "T Distribution"
                :action #'t-distribution)
          (send menu-item-proto :new "Chi-square Distribution"
                :action #'chisq-distribution)
          (send menu-item-proto :new "F Distribution"
                :action #'F-distribution))
    (send menu :install)))
\nwindexdefn{stbl}{stbl}{NWstb7-MaiD-1}\eatline
\nwidentdefs{\\{{stbl}{stbl}}}\nwused{\\{NWstb7-Cod4-1}\\{NWstb7-StaN-1}}\nwendcode{}\nwbegindocs{153}\nwdocspar
Oh!, we have to invoke the function!
\nwenddocs{}\nwbegincode{154}\sublabel{NWstb7-InvK-1}\nwmargintag{{\nwtagstyle{}\subpageref{NWstb7-InvK-1}}}\moddef{Invoke main function~{\nwtagstyle{}\subpageref{NWstb7-InvK-1}}}\endmoddef
(stbl)
\nwidentuses{\\{{stbl}{stbl}}}\nwindexuse{stbl}{stbl}{NWstb7-InvK-1}\nwused{\\{NWstb7-Cod4-1}\\{NWstb7-StaN-1}}\nwendcode{}\nwbegindocs{155}\nwdocspar

\section{Other Auxiliary Programs}
\label{sec:programs}

\nwenddocs{}\nwbegincode{156}\sublabel{NWstb7-ProG-1}\nwmargintag{{\nwtagstyle{}\subpageref{NWstb7-ProG-1}}}\moddef{Program hist.lsp~{\nwtagstyle{}\subpageref{NWstb7-ProG-1}}}\endmoddef
\LA{}Copyright for code~{\nwtagstyle{}\subpageref{NWstb7-CopI-1}}\RA{}
(def z (histogram (normal-rand 1000)))
(defmeth z :close () (send self :remove) (exit))
\nosublabel{NWstb7-ProG-1-u1}\nwindexdefn{hist.lsp}{hist.lsp}{NWstb7-ProG-1}\eatline
\nwidentdefs{\\{{hist.lsp}{hist.lsp}}}\nwused{\\{NWstb7-Cod4-1}}\nwendcode{}\nwbegindocs{157}\nwdocspar

\nwenddocs{}\nwbegincode{158}\sublabel{NWstb7-ProI-1}\nwmargintag{{\nwtagstyle{}\subpageref{NWstb7-ProI-1}}}\moddef{Program smhist.lsp~{\nwtagstyle{}\subpageref{NWstb7-ProI-1}}}\endmoddef
\LA{}Copyright for code~{\nwtagstyle{}\subpageref{NWstb7-CopI-1}}\RA{}
(def l (normal-rand 1000))
(def z (histogram l))
(send z :add-lines (kernel-dens l :type 'g))
(defmeth z :close () (send self :remove) (exit))
\nwindexdefn{smhist.lsp}{smhist.lsp}{NWstb7-ProI-1}\eatline
\nwidentdefs{\\{{smhist.lsp}{smhist.lsp}}}\nwidentuses{\\{{dens}{dens}}}\nwindexuse{dens}{dens}{NWstb7-ProI-1}\nwused{\\{NWstb7-Cod4-1}}\nwendcode{}\nwbegindocs{159}\nwdocspar

\nwenddocs{}\nwbegincode{160}\sublabel{NWstb7-ProE-1}\nwmargintag{{\nwtagstyle{}\subpageref{NWstb7-ProE-1}}}\moddef{Program 68.lsp~{\nwtagstyle{}\subpageref{NWstb7-ProE-1}}}\endmoddef
\LA{}Copyright for code~{\nwtagstyle{}\subpageref{NWstb7-CopI-1}}\RA{}
(require "dists")
(def z 
  (send dist-plot-proto :new
        :dens #'gaussian-density
        :cdf #'gaussian-cdf
        :icdf #'gaussian-icdf
        :params (list 0 1)
        :params-print-format "Parameters: Mean = ~d, Std. Dev. = ~d"
        :prob-dialog #'normal-prob-dialog
        :quant-dialog #'normal-quant-dialog
        :num-points 50
        :title "Normal Distribution"))
(defmeth z :close () (send self :remove) (exit))
(send z :l-point -1)
(send z :r-point 1)
(send z :important-abscissae '(-1 1))
(send z :answer "The shaded area is 68%")
(send z :redraw)
\nwindexdefn{68.lsp}{68.lsp}{NWstb7-ProE-1}\eatline
\nwidentdefs{\\{{68.lsp}{68.lsp}}}\nwidentuses{\\{{:answer}{:colanswer}}\\{{answer}{answer}}\\{{:cdf}{:colcdf}}\\{{cdf}{cdf}}\\{{:dens}{:coldens}}\\{{dens}{dens}}\\{{dist-plot-proto}{dist-plot-proto}}\\{{:icdf}{:colicdf}}\\{{icdf}{icdf}}\\{{:important-abscissae}{:colimportant-abscissae}}\\{{important-abscissae}{important-abscissae}}\\{{:l-point}{:coll-point}}\\{{l-point}{l-point}}\\{{:num-points}{:colnum-points}}\\{{num-points}{num-points}}\\{{:params}{:colparams}}\\{{params}{params}}\\{{:params-print-format}{:colparams-print-format}}\\{{params-print-format}{params-print-format}}\\{{:redraw}{:colredraw}}\\{{:r-point}{:colr-point}}\\{{r-point}{r-point}}}\nwindexuse{:answer}{:colanswer}{NWstb7-ProE-1}\nwindexuse{answer}{answer}{NWstb7-ProE-1}\nwindexuse{:cdf}{:colcdf}{NWstb7-ProE-1}\nwindexuse{cdf}{cdf}{NWstb7-ProE-1}\nwindexuse{:dens}{:coldens}{NWstb7-ProE-1}\nwindexuse{dens}{dens}{NWstb7-ProE-1}\nwindexuse{dist-plot-proto}{dist-plot-proto}{NWstb7-ProE-1}\nwindexuse{:icdf}{:colicdf}{NWstb7-ProE-1}\nwindexuse{icdf}{icdf}{NWstb7-ProE-1}\nwindexuse{:important-abscissae}{:colimportant-abscissae}{NWstb7-ProE-1}\nwindexuse{important-abscissae}{important-abscissae}{NWstb7-ProE-1}\nwindexuse{:l-point}{:coll-point}{NWstb7-ProE-1}\nwindexuse{l-point}{l-point}{NWstb7-ProE-1}\nwindexuse{:num-points}{:colnum-points}{NWstb7-ProE-1}\nwindexuse{num-points}{num-points}{NWstb7-ProE-1}\nwindexuse{:params}{:colparams}{NWstb7-ProE-1}\nwindexuse{params}{params}{NWstb7-ProE-1}\nwindexuse{:params-print-format}{:colparams-print-format}{NWstb7-ProE-1}\nwindexuse{params-print-format}{params-print-format}{NWstb7-ProE-1}\nwindexuse{:redraw}{:colredraw}{NWstb7-ProE-1}\nwindexuse{:r-point}{:colr-point}{NWstb7-ProE-1}\nwindexuse{r-point}{r-point}{NWstb7-ProE-1}\nwnotused{Program\ 68.lsp}\nwendcode{}\nwbegindocs{161}\nwdocspar

\nwenddocs{}\nwbegincode{162}\sublabel{NWstb7-ProE.2-1}\nwmargintag{{\nwtagstyle{}\subpageref{NWstb7-ProE.2-1}}}\moddef{Program 95.lsp~{\nwtagstyle{}\subpageref{NWstb7-ProE.2-1}}}\endmoddef
\LA{}Copyright for code~{\nwtagstyle{}\subpageref{NWstb7-CopI-1}}\RA{}
(require "dists")
(def z 
  (send dist-plot-proto :new
        :dens #'gaussian-density
        :cdf #'gaussian-cdf
        :icdf #'gaussian-icdf
        :params (list 0 1)
        :params-print-format "Parameters: Mean = ~d, Std. Dev. = ~d"
        :prob-dialog #'normal-prob-dialog
        :quant-dialog #'normal-quant-dialog
        :num-points 50
        :title "Normal Distribution"))
(defmeth z :close () (send self :remove) (exit))
(send z :l-point -2)
(send z :r-point 2)
(send z :important-abscissae '(-2 2))
(send z :answer "The shaded area is 95%")
(send z :redraw)
\nwindexdefn{95.lsp}{95.lsp}{NWstb7-ProE.2-1}\eatline
\nwidentdefs{\\{{95.lsp}{95.lsp}}}\nwidentuses{\\{{:answer}{:colanswer}}\\{{answer}{answer}}\\{{:cdf}{:colcdf}}\\{{cdf}{cdf}}\\{{:dens}{:coldens}}\\{{dens}{dens}}\\{{dist-plot-proto}{dist-plot-proto}}\\{{:icdf}{:colicdf}}\\{{icdf}{icdf}}\\{{:important-abscissae}{:colimportant-abscissae}}\\{{important-abscissae}{important-abscissae}}\\{{:l-point}{:coll-point}}\\{{l-point}{l-point}}\\{{:num-points}{:colnum-points}}\\{{num-points}{num-points}}\\{{:params}{:colparams}}\\{{params}{params}}\\{{:params-print-format}{:colparams-print-format}}\\{{params-print-format}{params-print-format}}\\{{:redraw}{:colredraw}}\\{{:r-point}{:colr-point}}\\{{r-point}{r-point}}}\nwindexuse{:answer}{:colanswer}{NWstb7-ProE.2-1}\nwindexuse{answer}{answer}{NWstb7-ProE.2-1}\nwindexuse{:cdf}{:colcdf}{NWstb7-ProE.2-1}\nwindexuse{cdf}{cdf}{NWstb7-ProE.2-1}\nwindexuse{:dens}{:coldens}{NWstb7-ProE.2-1}\nwindexuse{dens}{dens}{NWstb7-ProE.2-1}\nwindexuse{dist-plot-proto}{dist-plot-proto}{NWstb7-ProE.2-1}\nwindexuse{:icdf}{:colicdf}{NWstb7-ProE.2-1}\nwindexuse{icdf}{icdf}{NWstb7-ProE.2-1}\nwindexuse{:important-abscissae}{:colimportant-abscissae}{NWstb7-ProE.2-1}\nwindexuse{important-abscissae}{important-abscissae}{NWstb7-ProE.2-1}\nwindexuse{:l-point}{:coll-point}{NWstb7-ProE.2-1}\nwindexuse{l-point}{l-point}{NWstb7-ProE.2-1}\nwindexuse{:num-points}{:colnum-points}{NWstb7-ProE.2-1}\nwindexuse{num-points}{num-points}{NWstb7-ProE.2-1}\nwindexuse{:params}{:colparams}{NWstb7-ProE.2-1}\nwindexuse{params}{params}{NWstb7-ProE.2-1}\nwindexuse{:params-print-format}{:colparams-print-format}{NWstb7-ProE.2-1}\nwindexuse{params-print-format}{params-print-format}{NWstb7-ProE.2-1}\nwindexuse{:redraw}{:colredraw}{NWstb7-ProE.2-1}\nwindexuse{:r-point}{:colr-point}{NWstb7-ProE.2-1}\nwindexuse{r-point}{r-point}{NWstb7-ProE.2-1}\nwnotused{Program\ 95.lsp}\nwendcode{}\nwbegindocs{163}\nwdocspar

\nwenddocs{}\nwbegincode{164}\sublabel{NWstb7-ProE.3-1}\nwmargintag{{\nwtagstyle{}\subpageref{NWstb7-ProE.3-1}}}\moddef{Program 99.lsp~{\nwtagstyle{}\subpageref{NWstb7-ProE.3-1}}}\endmoddef
\LA{}Copyright for code~{\nwtagstyle{}\subpageref{NWstb7-CopI-1}}\RA{}
(require "dists")
(def z 
  (send dist-plot-proto :new
        :dens #'gaussian-density
        :cdf #'gaussian-cdf
        :icdf #'gaussian-icdf
        :params (list 0 1)
        :params-print-format "Parameters: Mean = ~d, Std. Dev. = ~d"
        :prob-dialog #'normal-prob-dialog
        :quant-dialog #'normal-quant-dialog
        :num-points 50
        :title "Normal Distribution"))
(defmeth z :close () (send self :remove) (exit))
(send z :l-point -3)
(send z :r-point 3)
(send z :important-abscissae '(-3 3))
(send z :answer "The shaded area is 99.7%")
(send z :redraw)
\nwindexdefn{99.lsp}{99.lsp}{NWstb7-ProE.3-1}\eatline
\nwidentdefs{\\{{99.lsp}{99.lsp}}}\nwidentuses{\\{{:answer}{:colanswer}}\\{{answer}{answer}}\\{{:cdf}{:colcdf}}\\{{cdf}{cdf}}\\{{:dens}{:coldens}}\\{{dens}{dens}}\\{{dist-plot-proto}{dist-plot-proto}}\\{{:icdf}{:colicdf}}\\{{icdf}{icdf}}\\{{:important-abscissae}{:colimportant-abscissae}}\\{{important-abscissae}{important-abscissae}}\\{{:l-point}{:coll-point}}\\{{l-point}{l-point}}\\{{:num-points}{:colnum-points}}\\{{num-points}{num-points}}\\{{:params}{:colparams}}\\{{params}{params}}\\{{:params-print-format}{:colparams-print-format}}\\{{params-print-format}{params-print-format}}\\{{:redraw}{:colredraw}}\\{{:r-point}{:colr-point}}\\{{r-point}{r-point}}}\nwindexuse{:answer}{:colanswer}{NWstb7-ProE.3-1}\nwindexuse{answer}{answer}{NWstb7-ProE.3-1}\nwindexuse{:cdf}{:colcdf}{NWstb7-ProE.3-1}\nwindexuse{cdf}{cdf}{NWstb7-ProE.3-1}\nwindexuse{:dens}{:coldens}{NWstb7-ProE.3-1}\nwindexuse{dens}{dens}{NWstb7-ProE.3-1}\nwindexuse{dist-plot-proto}{dist-plot-proto}{NWstb7-ProE.3-1}\nwindexuse{:icdf}{:colicdf}{NWstb7-ProE.3-1}\nwindexuse{icdf}{icdf}{NWstb7-ProE.3-1}\nwindexuse{:important-abscissae}{:colimportant-abscissae}{NWstb7-ProE.3-1}\nwindexuse{important-abscissae}{important-abscissae}{NWstb7-ProE.3-1}\nwindexuse{:l-point}{:coll-point}{NWstb7-ProE.3-1}\nwindexuse{l-point}{l-point}{NWstb7-ProE.3-1}\nwindexuse{:num-points}{:colnum-points}{NWstb7-ProE.3-1}\nwindexuse{num-points}{num-points}{NWstb7-ProE.3-1}\nwindexuse{:params}{:colparams}{NWstb7-ProE.3-1}\nwindexuse{params}{params}{NWstb7-ProE.3-1}\nwindexuse{:params-print-format}{:colparams-print-format}{NWstb7-ProE.3-1}\nwindexuse{params-print-format}{params-print-format}{NWstb7-ProE.3-1}\nwindexuse{:redraw}{:colredraw}{NWstb7-ProE.3-1}\nwindexuse{:r-point}{:colr-point}{NWstb7-ProE.3-1}\nwindexuse{r-point}{r-point}{NWstb7-ProE.3-1}\nwnotused{Program\ 99.lsp}\nwendcode{}\nwbegindocs{165}\nwdocspar

\section{Miscellany}
\label{sec:misc}
It is often the case that one might want some of the functions we have
designed here for other purposes. So let us make it easy to extract
various sections of the code as separate files. I have not bothered
with packages, but that would be easy to fix.

\nwenddocs{}\nwbegincode{166}\sublabel{NWstb7-UtiE-1}\nwmargintag{{\nwtagstyle{}\subpageref{NWstb7-UtiE-1}}}\moddef{Utilities file~{\nwtagstyle{}\subpageref{NWstb7-UtiE-1}}}\endmoddef
\LA{}Copyright for code~{\nwtagstyle{}\subpageref{NWstb7-CopI-1}}\RA{}
(provide "utils")
\LA{}Utility functions~{\nwtagstyle{}\subpageref{NWstb7-UtiH-1}}\RA{}
\LA{}Additional methods for built-in prototypes~{\nwtagstyle{}\subpageref{NWstb7-Addg-1}}\RA{}
\nwnotused{Utilities\ file}\nwendcode{}\nwbegindocs{167}\nwdocspar

\nwenddocs{}\nwbegincode{168}\sublabel{NWstb7-ButJ-1}\nwmargintag{{\nwtagstyle{}\subpageref{NWstb7-ButJ-1}}}\moddef{Button overlay file~{\nwtagstyle{}\subpageref{NWstb7-ButJ-1}}}\endmoddef
\LA{}Copyright for code~{\nwtagstyle{}\subpageref{NWstb7-CopI-1}}\RA{}
(provide "button")
\LA{}Button overlay prototype definition~{\nwtagstyle{}\subpageref{NWstb7-ButZ-1}}\RA{}
\LA{}Button overlay prototype methods~{\nwtagstyle{}\subpageref{NWstb7-ButW-1}}\RA{}
\nwnotused{Button\ overlay\ file}\nwendcode{}\nwbegindocs{169}\nwdocspar

\nwenddocs{}\nwbegincode{170}\sublabel{NWstb7-DisN-1}\nwmargintag{{\nwtagstyle{}\subpageref{NWstb7-DisN-1}}}\moddef{Distribution proto file~{\nwtagstyle{}\subpageref{NWstb7-DisN-1}}}\endmoddef
\LA{}Copyright for code~{\nwtagstyle{}\subpageref{NWstb7-CopI-1}}\RA{}
(require "utils")
(require "button")
(provide "distproto")
\LA{}Implementation constants~{\nwtagstyle{}\subpageref{NWstb7-ImpO-1}}\RA{}
\LA{}Distribution prototype definition~{\nwtagstyle{}\subpageref{NWstb7-DisX-1}}\RA{}
\LA{}Distribution prototype methods~{\nwtagstyle{}\subpageref{NWstb7-DisU-1}}\RA{}
\nwnotused{Distribution\ proto\ file}\nwendcode{}\nwbegindocs{171}\nwdocspar

\nwenddocs{}\nwbegincode{172}\sublabel{NWstb7-DisI-1}\nwmargintag{{\nwtagstyle{}\subpageref{NWstb7-DisI-1}}}\moddef{Distributions file~{\nwtagstyle{}\subpageref{NWstb7-DisI-1}}}\endmoddef
\LA{}Copyright for code~{\nwtagstyle{}\subpageref{NWstb7-CopI-1}}\RA{}
(require "distproto")
(provide "dists")
\LA{}Normal distribution~{\nwtagstyle{}\subpageref{NWstb7-NorJ-1}}\RA{}
\LA{}T distribution~{\nwtagstyle{}\subpageref{NWstb7-T*dE-1}}\RA{}
\LA{}Chi-square distribution~{\nwtagstyle{}\subpageref{NWstb7-ChiN-1}}\RA{}
\LA{}F distribution~{\nwtagstyle{}\subpageref{NWstb7-F*dE-1}}\RA{}
\nwnotused{Distributions\ file}\nwendcode{}\nwbegindocs{173}\nwdocspar

\nwenddocs{}\nwbegincode{174}\sublabel{NWstb7-StaN-1}\nwmargintag{{\nwtagstyle{}\subpageref{NWstb7-StaN-1}}}\moddef{Statistical tables file~{\nwtagstyle{}\subpageref{NWstb7-StaN-1}}}\endmoddef
\LA{}Copyright for code~{\nwtagstyle{}\subpageref{NWstb7-CopI-1}}\RA{}
(require "dists")
(provide "stbls")
\LA{}Main function~{\nwtagstyle{}\subpageref{NWstb7-MaiD-1}}\RA{}
\LA{}Invoke main function~{\nwtagstyle{}\subpageref{NWstb7-InvK-1}}\RA{}
\nwnotused{Statistical\ tables\ file}\nwendcode{}\nwbegindocs{175}\nwdocspar

\section{The Makefile}
\label{sec:makefile}
Our {\tt{}Makefile} must be able to extract each of the individual files,
the hyper-document and \LaTeX{} the hyper-document. It is pretty
simple.

\nwenddocs{}\nwbegincode{176}\sublabel{NWstb7-Mak8-1}\nwmargintag{{\nwtagstyle{}\subpageref{NWstb7-Mak8-1}}}\moddef{Makefile~{\nwtagstyle{}\subpageref{NWstb7-Mak8-1}}}\endmoddef
all:    stbl.nw
        noweave -index -delay stbl.nw > stbl.tex
        notangle -R'Hyper-document' stbl.nw > hyperdoc.tex
        notangle -R'Utilities file' stbl.nw > utils.lsp
        notangle -R'Button overlay file' stbl.nw > button.lsp
        notangle -R'Distribution proto file' stbl.nw > distproto.lsp
        notangle -R'Distributions file' stbl.nw > dists.lsp
        notangle -R'Statistical tables file' stbl.nw > stbls.lsp
        notangle -R'Program hist.lsp' stbl.nw > hist.lsp
        notangle -R'Program smhist.lsp' stbl.nw > smhist.lsp
        notangle -R'Program 68.lsp' stbl.nw > 68.lsp
        notangle -R'Program 95.lsp' stbl.nw > 95.lsp
        notangle -R'Program 99.lsp' stbl.nw > 99.lsp
        notangle -R'Readme file' stbl.nw > README

\nwidentuses{\\{{hist.lsp}{hist.lsp}}\\{{68.lsp}{68.lsp}}\\{{95.lsp}{95.lsp}}\\{{99.lsp}{99.lsp}}\\{{smhist.lsp}{smhist.lsp}}\\{{stbl}{stbl}}}\nwindexuse{hist.lsp}{hist.lsp}{NWstb7-Mak8-1}\nwindexuse{68.lsp}{68.lsp}{NWstb7-Mak8-1}\nwindexuse{95.lsp}{95.lsp}{NWstb7-Mak8-1}\nwindexuse{99.lsp}{99.lsp}{NWstb7-Mak8-1}\nwindexuse{smhist.lsp}{smhist.lsp}{NWstb7-Mak8-1}\nwindexuse{stbl}{stbl}{NWstb7-Mak8-1}\nwused{\\{NWstb7-LitG-1}}\nwendcode{}\nwbegindocs{177}\nwdocspar
\section{The Readme file}
\label{sec:readme}
We shall just refer them to the literate programming introduction.

\nwenddocs{}\nwbegincode{178}\sublabel{NWstb7-ReaB-1}\nwmargintag{{\nwtagstyle{}\subpageref{NWstb7-ReaB-1}}}\moddef{Readme file~{\nwtagstyle{}\subpageref{NWstb7-ReaB-1}}}\endmoddef
Please look at the introduction section of stbl.ps and read it in its
entirety. 

It contains a description of the program and how to use it. In
addition there are some pictures that will give you a better idea of
what the code can do.
\nwidentuses{\\{{stbl}{stbl}}}\nwindexuse{stbl}{stbl}{NWstb7-ReaB-1}\nwused{\\{NWstb7-LitG-1}}\nwendcode{}

\nwixlogsorted{c}{{Additional methods for built-in prototypes}{NWstb7-Addg-1}{\nwixu{NWstb7-Cod4-1}\nwixd{NWstb7-Addg-1}\nwixd{NWstb7-Addg-2}\nwixd{NWstb7-Addg-3}\nwixu{NWstb7-UtiE-1}}}%
\nwixlogsorted{c}{{Button overlay file}{NWstb7-ButJ-1}{\nwixd{NWstb7-ButJ-1}}}%
\nwixlogsorted{c}{{Button overlay prototype definition}{NWstb7-ButZ-1}{\nwixu{NWstb7-Cod4-1}\nwixd{NWstb7-ButZ-1}\nwixu{NWstb7-ButJ-1}}}%
\nwixlogsorted{c}{{Button overlay prototype methods}{NWstb7-ButW-1}{\nwixu{NWstb7-Cod4-1}\nwixd{NWstb7-ButW-1}\nwixd{NWstb7-ButW-2}\nwixd{NWstb7-ButW-3}\nwixu{NWstb7-ButJ-1}}}%
\nwixlogsorted{c}{{Chi-square distribution}{NWstb7-ChiN-1}{\nwixu{NWstb7-Cod4-1}\nwixd{NWstb7-ChiN-1}\nwixd{NWstb7-ChiN-2}\nwixd{NWstb7-ChiN-3}\nwixu{NWstb7-DisI-1}}}%
\nwixlogsorted{c}{{Code}{NWstb7-Cod4-1}{\nwixu{NWstb7-HypK-1}\nwixd{NWstb7-Cod4-1}}}%
\nwixlogsorted{c}{{Copyright for code}{NWstb7-CopI-1}{\nwixd{NWstb7-CopI-1}\nwixu{NWstb7-Cod4-1}\nwixu{NWstb7-ProG-1}\nwixu{NWstb7-ProI-1}\nwixu{NWstb7-ProE-1}\nwixu{NWstb7-ProE.2-1}\nwixu{NWstb7-ProE.3-1}\nwixu{NWstb7-UtiE-1}\nwixu{NWstb7-ButJ-1}\nwixu{NWstb7-DisN-1}\nwixu{NWstb7-DisI-1}\nwixu{NWstb7-StaN-1}}}%
\nwixlogsorted{c}{{Distribution proto file}{NWstb7-DisN-1}{\nwixd{NWstb7-DisN-1}}}%
\nwixlogsorted{c}{{Distribution prototype accessor and modifier methods}{NWstb7-Disq-1}{\nwixu{NWstb7-DisU-1}\nwixd{NWstb7-Disq-1}\nwixd{NWstb7-Disq-2}\nwixd{NWstb7-Disq-3}\nwixd{NWstb7-Disq-4}\nwixd{NWstb7-Disq-5}\nwixd{NWstb7-Disq-6}\nwixd{NWstb7-Disq-7}\nwixd{NWstb7-Disq-8}\nwixd{NWstb7-Disq-9}\nwixd{NWstb7-Disq-A}\nwixd{NWstb7-Disq-B}\nwixd{NWstb7-Disq-C}\nwixd{NWstb7-Disq-D}\nwixd{NWstb7-Disq-E}\nwixd{NWstb7-Disq-F}}}%
\nwixlogsorted{c}{{Distribution prototype definition}{NWstb7-DisX-1}{\nwixu{NWstb7-Cod4-1}\nwixd{NWstb7-DisX-1}\nwixu{NWstb7-DisN-1}}}%
\nwixlogsorted{c}{{Distribution prototype :isnew method}{NWstb7-Disa-1}{\nwixu{NWstb7-DisU-1}\nwixd{NWstb7-Disa-1}\nwixd{NWstb7-Disa-2}\nwixd{NWstb7-Disa-3}\nwixd{NWstb7-Disa-4}\nwixd{NWstb7-Disa-5}\nwixd{NWstb7-Disa-6}}}%
\nwixlogsorted{c}{{Distribution prototype methods}{NWstb7-DisU-1}{\nwixu{NWstb7-Cod4-1}\nwixd{NWstb7-DisU-1}\nwixu{NWstb7-DisN-1}}}%
\nwixlogsorted{c}{{Distribution prototype redrawing methods}{NWstb7-Dise-1}{\nwixu{NWstb7-DisU-1}\nwixd{NWstb7-Dise-1}\nwixd{NWstb7-Dise-2}}}%
\nwixlogsorted{c}{{Distributions file}{NWstb7-DisI-1}{\nwixd{NWstb7-DisI-1}}}%
\nwixlogsorted{c}{{F distribution}{NWstb7-F*dE-1}{\nwixu{NWstb7-Cod4-1}\nwixd{NWstb7-F*dE-1}\nwixd{NWstb7-F*dE-2}\nwixd{NWstb7-F*dE-3}\nwixu{NWstb7-DisI-1}}}%
\nwixlogsorted{c}{{Hyper-document}{NWstb7-HypE-1}{\nwixu{NWstb7-HypK-1}\nwixd{NWstb7-HypE-1}}}%
\nwixlogsorted{c}{{Hyper-document stuff}{NWstb7-HypK-1}{\nwixu{NWstb7-LitG-1}\nwixd{NWstb7-HypK-1}}}%
\nwixlogsorted{c}{{Implementation constants}{NWstb7-ImpO-1}{\nwixu{NWstb7-Cod4-1}\nwixd{NWstb7-ImpO-1}\nwixd{NWstb7-ImpO-2}\nwixd{NWstb7-ImpO-3}\nwixd{NWstb7-ImpO-4}\nwixu{NWstb7-DisN-1}}}%
\nwixlogsorted{c}{{Invoke main function}{NWstb7-InvK-1}{\nwixu{NWstb7-Cod4-1}\nwixd{NWstb7-InvK-1}\nwixu{NWstb7-StaN-1}}}%
\nwixlogsorted{c}{{Literate Program}{NWstb7-LitG-1}{\nwixd{NWstb7-LitG-1}}}%
\nwixlogsorted{c}{{Main function}{NWstb7-MaiD-1}{\nwixu{NWstb7-Cod4-1}\nwixd{NWstb7-MaiD-1}\nwixu{NWstb7-StaN-1}}}%
\nwixlogsorted{c}{{Makefile}{NWstb7-Mak8-1}{\nwixu{NWstb7-LitG-1}\nwixd{NWstb7-Mak8-1}}}%
\nwixlogsorted{c}{{Normal distribution}{NWstb7-NorJ-1}{\nwixu{NWstb7-Cod4-1}\nwixd{NWstb7-NorJ-1}\nwixd{NWstb7-NorJ-2}\nwixd{NWstb7-NorJ-3}\nwixd{NWstb7-NorJ-4}\nwixd{NWstb7-NorJ-5}\nwixd{NWstb7-NorJ-6}\nwixu{NWstb7-DisI-1}}}%
\nwixlogsorted{c}{{Other useful methods for distribution prototype}{NWstb7-Othl-1}{\nwixu{NWstb7-DisU-1}\nwixd{NWstb7-Othl-1}\nwixd{NWstb7-Othl-2}\nwixd{NWstb7-Othl-3}\nwixd{NWstb7-Othl-4}\nwixd{NWstb7-Othl-5}\nwixd{NWstb7-Othl-6}\nwixd{NWstb7-Othl-7}\nwixd{NWstb7-Othl-8}\nwixd{NWstb7-Othl-9}\nwixd{NWstb7-Othl-A}\nwixd{NWstb7-Othl-B}\nwixd{NWstb7-Othl-C}}}%
\nwixlogsorted{c}{{Program hist.lsp}{NWstb7-ProG-1}{\nwixu{NWstb7-Cod4-1}\nwixd{NWstb7-ProG-1}}}%
\nwixlogsorted{c}{{Program 68.lsp}{NWstb7-ProE-1}{\nwixd{NWstb7-ProE-1}}}%
\nwixlogsorted{c}{{Program 95.lsp}{NWstb7-ProE.2-1}{\nwixd{NWstb7-ProE.2-1}}}%
\nwixlogsorted{c}{{Program 99.lsp}{NWstb7-ProE.3-1}{\nwixd{NWstb7-ProE.3-1}}}%
\nwixlogsorted{c}{{Program smhist.lsp}{NWstb7-ProI-1}{\nwixu{NWstb7-Cod4-1}\nwixd{NWstb7-ProI-1}}}%
\nwixlogsorted{c}{{Readme file}{NWstb7-ReaB-1}{\nwixu{NWstb7-LitG-1}\nwixd{NWstb7-ReaB-1}}}%
\nwixlogsorted{c}{{Statistical tables file}{NWstb7-StaN-1}{\nwixd{NWstb7-StaN-1}}}%
\nwixlogsorted{c}{{T distribution}{NWstb7-T*dE-1}{\nwixu{NWstb7-Cod4-1}\nwixd{NWstb7-T*dE-1}\nwixd{NWstb7-T*dE-2}\nwixd{NWstb7-T*dE-3}\nwixu{NWstb7-DisI-1}}}%
\nwixlogsorted{c}{{Utilities file}{NWstb7-UtiE-1}{\nwixd{NWstb7-UtiE-1}}}%
\nwixlogsorted{c}{{Utility functions}{NWstb7-UtiH-1}{\nwixu{NWstb7-Cod4-1}\nwixd{NWstb7-UtiH-1}\nwixd{NWstb7-UtiH-2}\nwixd{NWstb7-UtiH-3}\nwixd{NWstb7-UtiH-4}\nwixd{NWstb7-UtiH-5}\nwixd{NWstb7-UtiH-6}\nwixd{NWstb7-UtiH-7}\nwixu{NWstb7-UtiE-1}}}%
\nwixlogsorted{i}{{Copyright}{Copyright}}%
\nwixlogsorted{i}{{FSF}{FSF}}%
\nwixlogsorted{i}{{GNU}{GNU}}%
\nwixlogsorted{i}{{:answer}{:colanswer}}%
\nwixlogsorted{i}{{answer}{answer}}%
\nwixlogsorted{i}{{:answer-display-loc}{:colanswer-display-loc}}%
\nwixlogsorted{i}{{answer-display-loc}{answer-display-loc}}%
\nwixlogsorted{i}{{:button-overlay-proto}{:colbutton-overlay-proto}}%
\nwixlogsorted{i}{{:cdf}{:colcdf}}%
\nwixlogsorted{i}{{cdf}{cdf}}%
\nwixlogsorted{i}{{:cdf-at}{:colcdf-at}}%
\nwixlogsorted{i}{{chisq-distribution}{chisq-distribution}}%
\nwixlogsorted{i}{{chisq-prob-dialog}{chisq-prob-dialog}}%
\nwixlogsorted{i}{{chisq-quant-dialog}{chisq-quant-dialog}}%
\nwixlogsorted{i}{{:dens}{:coldens}}%
\nwixlogsorted{i}{{dens}{dens}}%
\nwixlogsorted{i}{{:dens-at}{:coldens-at}}%
\nwixlogsorted{i}{{:display-answer}{:coldisplay-answer}}%
\nwixlogsorted{i}{{:display-params}{:coldisplay-params}}%
\nwixlogsorted{i}{{dist-plot-proto}{dist-plot-proto}}%
\nwixlogsorted{i}{{:do-click}{:coldo-click}}%
\nwixlogsorted{i}{{:draw-vert-arrow}{:coldraw-vert-arrow}}%
\nwixlogsorted{i}{{f-distribution}{f-distribution}}%
\nwixlogsorted{i}{{f-prob-dialog}{f-prob-dialog}}%
\nwixlogsorted{i}{{f-quant-dialog}{f-quant-dialog}}%
\nwixlogsorted{i}{{gaussian-cdf}{gaussian-cdf}}%
\nwixlogsorted{i}{{gaussian-density}{gaussian-density}}%
\nwixlogsorted{i}{{get-numbers-from}{get-numbers-from}}%
\nwixlogsorted{i}{{get-values-from}{get-values-from}}%
\nwixlogsorted{i}{{:highlight-important-abscissae}{:colhighlight-important-abscissae}}%
\nwixlogsorted{i}{{hist.lsp}{hist.lsp}}%
\nwixlogsorted{i}{{:icdf}{:colicdf}}%
\nwixlogsorted{i}{{icdf}{icdf}}%
\nwixlogsorted{i}{{:icdf-at}{:colicdf-at}}%
\nwixlogsorted{i}{{:important-abscissae}{:colimportant-abscissae}}%
\nwixlogsorted{i}{{important-abscissae}{important-abscissae}}%
\nwixlogsorted{i}{{:isnew}{:colisnew}}%
\nwixlogsorted{i}{{:l-point}{:coll-point}}%
\nwixlogsorted{i}{{l-point}{l-point}}%
\nwixlogsorted{i}{{68.lsp}{68.lsp}}%
\nwixlogsorted{i}{{95.lsp}{95.lsp}}%
\nwixlogsorted{i}{{99.lsp}{99.lsp}}%
\nwixlogsorted{i}{{*max-probability*}{*max-probability*}}%
\nwixlogsorted{i}{{:max-probability}{:colmax-probability}}%
\nwixlogsorted{i}{{max-probability}{max-probability}}%
\nwixlogsorted{i}{{*min-probability*}{*min-probability*}}%
\nwixlogsorted{i}{{:min-probability}{:colmin-probability}}%
\nwixlogsorted{i}{{min-probability}{min-probability}}%
\nwixlogsorted{i}{{new-xlispstat}{new-xlispstat}}%
\nwixlogsorted{i}{{nonzero-probability-p}{nonzero-probability-p}}%
\nwixlogsorted{i}{{normal-distribution}{normal-distribution}}%
\nwixlogsorted{i}{{normal-proba-dialog}{normal-proba-dialog}}%
\nwixlogsorted{i}{{normal-quantile-dialog}{normal-quantile-dialog}}%
\nwixlogsorted{i}{{*num-points*}{*num-points*}}%
\nwixlogsorted{i}{{:num-points}{:colnum-points}}%
\nwixlogsorted{i}{{num-points}{num-points}}%
\nwixlogsorted{i}{{:params}{:colparams}}%
\nwixlogsorted{i}{{params}{params}}%
\nwixlogsorted{i}{{:params-display-loc}{:colparams-display-loc}}%
\nwixlogsorted{i}{{params-display-loc}{params-display-loc}}%
\nwixlogsorted{i}{{:params-print-format}{:colparams-print-format}}%
\nwixlogsorted{i}{{params-print-format}{params-print-format}}%
\nwixlogsorted{i}{{:probability-answer}{:colprobability-answer}}%
\nwixlogsorted{i}{{probability-p}{probability-p}}%
\nwixlogsorted{i}{{*probability-print-format*}{*probability-print-format*}}%
\nwixlogsorted{i}{{:quantile-answer}{:colquantile-answer}}%
\nwixlogsorted{i}{{*quantile-print-format*}{*quantile-print-format*}}%
\nwixlogsorted{i}{{:redraw}{:colredraw}}%
\nwixlogsorted{i}{{:redraw-background}{:colredraw-background}}%
\nwixlogsorted{i}{{:redraw-content}{:colredraw-content}}%
\nwixlogsorted{i}{{:r-point}{:colr-point}}%
\nwixlogsorted{i}{{r-point}{r-point}}%
\nwixlogsorted{i}{{*shade-color*}{*shade-color*}}%
\nwixlogsorted{i}{{:shade-color}{:colshade-color}}%
\nwixlogsorted{i}{{shade-color}{shade-color}}%
\nwixlogsorted{i}{{:shade-under-plot}{:colshade-under-plot}}%
\nwixlogsorted{i}{{smhist.lsp}{smhist.lsp}}%
\nwixlogsorted{i}{{stbl}{stbl}}%
\nwixlogsorted{i}{{strict-probability-p}{strict-probability-p}}%
\nwixlogsorted{i}{{t-distribution}{t-distribution}}%
\nwixlogsorted{i}{{t-prob-dialog}{t-prob-dialog}}%
\nwixlogsorted{i}{{t-quant-dialog}{t-quant-dialog}}%
\nwixlogsorted{i}{{val-of}{val-of}}%
\nwixlogsorted{i}{{:width}{:colwidth}}%
\nwixlogsorted{i}{{:xmax}{:colxmax}}%
\nwixlogsorted{i}{{:xmin}{:colxmin}}%
\nwbegindocs{179}\nwdocspar

\section{Discussion}
\label{sec:discussion}
We devote this section to a discussion of problems and issues arising
in the development of a full-fledged hyper-text. It is incomplete as
of now.

\begin{thebibliography}{99}

\bibitem{moore}{Moore, David S., and McCabe, George P., {\it
    Introduction to the Practice of Statistics}, Second Edition,
edition, W. H. Freeman \& Co., (1993).}

\end{thebibliography}

\section*{List of code chunks}
This list is generated automatically. The numeral is that of the first
definition of the chunk.
\nowebchunks
\begin{multicols}{2}[\section*{Index}
Here is a list of the identifiers used, and where they appear.
Underlined entries indicate the place of definition.
This index is generated automatically.]
\nowebindex
\end{multicols}
\end{document}









\nwenddocs{}
